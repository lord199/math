
%\documentclass[oneside,final,14pt]{extreport}
\documentclass[12pt, a4paper, openany]{book}
\usepackage[left=1cm,right=1cm,top=2cm,bottom=2cm,bindingoffset=0cm]{geometry}
%\usepackage[koi8-r]{inputenc}
\usepackage[russianb]{babel}
\usepackage{vmargin}
\setpapersize{A4}
\usepackage[T2A]{fontenc}
\usepackage[utf8x]{inputenc}  % more recent versions (at least>=2004-17-10)
%\usepackage[russian]{babel}
\setmarginsrb{2cm}{2cm}{2cm}{2cm}{0pt}{5mm}{0pt}{0mm}
\usepackage{indentfirst}
\usepackage{nicefrac} % For comparison
%\usepackage{xfrac}    % Works better with other fonts
%\usepackage[unicode, pdftex]{hyperref}
\usepackage{lettrine}
\usepackage[usenames]{color}
\usepackage{colortbl}
\usepackage{mathtext}
\usepackage{txfonts}
\pagestyle{plain}
\usepackage{pxfonts}
\usepackage[pagestyles]{titlesec}
\usepackage{xfrac}    % Works better with other fonts
\usepackage[colorlinks=true,linkcolor=black,urlcolor=black,bookmarksopen=true]{hyperref}

\usepackage{fancyhdr} % пакет для установки колонтитулов
\pagestyle{fancy} % смена стиля оформления страниц
\fancyhf{} % очистка текущих значений
\fancyhead[C]{\thepage} % установка верхнего колонтитула
\renewcommand{\headrulewidth}{0pt} % убрать разделительную линию


% Настройка вертикальных и горизонтальных отступов
\titlespacing{\chapter}{0pt}{5pt}{5pt}
\titlespacing{\section}{\parindent}{4mm}{4mm}
\titlespacing{\subsection}{\parindent}{3mm}{3mm}

\newcommand{\anonsection}[1]{ \section*{#1} \addcontentsline{toc}{section}{\numberline {}#1}} 

\makeatletter %%%%% <---- Starting chapter without a pagebreak
\renewcommand\chapter{\par%
	\thispagestyle{plain}% \global\@topnum\z@
	\@afterindentfalse \secdef\@chapter\@schapter}
\makeatother %%%%% <---- Starting chapter without a pagebreak
\titleformat{\chapter}[display]
{\normalfont\bfseries}{}{0pt}{\Large}

\newpagestyle{mystyle}{
	\sethead[\thepage][][]{}{}{\thepage}	
}

\pagestyle{mystyle}

\sloppy
\begin{document}

	
	\begin{titlepage}
		
		\begin{center}
			%\vfill
			
			%\vfill
			\topskip0pt
			\vspace*{\fill}
			\begin{flushleft}
		{\large	\textit{Математическое просвещение}			}
			\end{flushleft}
			\ \\
\ \\
			\ \\
\ \\
			{\large\bf ВАЦЛАВ СЕРПИНСКИЙ\\}
			\ \\
			\ \\
			{\Huge\bf 250 задач по элементарной теории чисел\\}
			\ \\
			\ \\
			\ \\
			\textit{Перевод с польского И.Г. Мельникова}

			\vspace*{\fill}    
			
			\vfill
			
И З Д А Т Е Л Ь С Т В О

«П Р О С В Е Щ Е Н И Е»

М о с к в а 1968
		\end{center}
		
	\end{titlepage}
	
	\thispagestyle{empty} % выключаем отображение номера для этой страницы
	
	\newpage
	
\setcounter{secnumdepth}{0}  
	
	\section[Выдающийся польский математик Вацлав Серпинский (к 85-летию со дня рождения)]{\center ВЫДАЮЩИИСЯ ПОЛЬСКИЙ МАТЕМАТИК ВАЦЛАВ СЕРПИНСКИЙ}
		\begin{center}
\textit{(К 85-летию со дня рождения)}
\end{center}
	
	Четырнадцатого марта 1882 г. в Варшаве в семье врача Константина Серпинского родился мальчик, которому дали два имени: Вацлав Франциск. Этому мальчику суждено было стать одним из крупнейших польских математиков.
	Образование Вацлав Серпинский получил в Варшаве. Здесь он окончил гимназию и университет.

	Незаурядные способности Серпинского обнаружились рано, повышенный же интерес к математике наметился лишь в последних классах гимназии под влиянием двух его соучеников, владевших некоторыми разделами высшей математики, и прекрасного учителя математики Влодзимежа Влодарского. Последний был очень высокого мнения о математических способностях Серпинского. В гимназии у Серпинского было еще несколько замечательных учителей. Так, его учителем французского языка был К. Аппель, впоследствии профессор Варшавского университета.

	Среди сверстников Серпинского по гимназии было немало способных людей. Из класса, в котором учился Серпинский, вышло заметное число ученых и деятелей культуры, из коих отметим известного астронома Тадеуша Банахевича.

	Уже в школьные годы Серпинский проявлял большой интерес к общественным делам. Вместе с несколькими своими друзьями он организовал тайную школу для мальчиков из рабочей среды. Эта хорошо законспирированная школа на протяжении ряда лет успешно готовила своих учащихся к экзамену за четыре класса гимназии.
	
	В 1900 г. Серпинский поступил на физико-математический факультет Варшавского университета, который в ту пору представлял собой молодое учебное заведение с преподаванием на русском языке, существовавшее всего около трех десятилетий\footnote{Этот университет был создан на базе Главной школы, существовавшей в Варшаве в 1862—1869 гг. С начала XIX столетия до 1832 г. в Варшаве был польский университет.}.
	
	Следует заметить, что поляки, имевшие возможность учиться в старинных польских университетах (в Краковском и Львовском), охотно шли и в новый университет. Трудности, которые испытывал университет в первые годы своего существования (отсутствие традиций, хороших научно-педагогических кадров и др.), вскоре были преодолены.
	
	Активная и разнообразная деятельность работавших здесь математиков М. А. Андреевского, Н. Н. Алексеева и Н. Я. Сонина на первой стадии, а затем В. А. Анисимова, Н. Н. Зинина и Г. Ф. Вороного позволила уже к концу XIX в. приблизить уровень преподавания математики в Варшавском университете к уровню преподавания математики в университетах Петербурга, Москвы, Казани, Харькова, Дерпта (Тарту)\footnote{Ср.: С. Е. Белозеров. Математика в Ростовском университете. Ист.-мат. исслед., вып. VI. М., Гостехиздат, 1953, стр. 247—352.}.
	
	
	Наибольшее влияние на Серпинского оказал питомец Петербургского университета профессор математики, впоследствии член-корреспондент Российской Академии наук, Георгий Федосеевич Вороной (1868—1908). Деятельность Вороного в Варшавском университете началась в 1894 г. и продолжалась там с небольшими перерывами до его безвременной смерти. Вороной — первоклассный ученый, на трудах которого лежит печать гениальности. Вместе с Германом Минковским он является создателем геометрии чисел. Глубокие и важные результаты были получены им в аналитической теории чисел, а также в теории алгебраических чисел. Его проблематика разрабатывалась в нашей стране Б. А. Венковым (1900—1962), Б. Н. Делоне (род. в 1890 г.), Д. К. Фаддеевым (род. в 1907 г.) и др., а также зарубежными математиками. Г. Ф. Вороной принадлежал к Петербургской школе теории чисел, и в ней он занимал одно из самых видных мест.
	
	Серпинский прослушал несколько лекционных курсов у Вороного и выполнил свою первую научную работу по аналитической теории чисел в духе идей и методов Вороного на тему, предложенную последним для конкурсных студенческих сочинений. Подробный отзыв Вороного на эту работу был напечатан в VI выпуске «Варшавских университетских известий» за 1904 г. Вороной ходатайствовал о присуждении Серпинскому золотой медали и об оставлении его при университете для подготовки к профессорскому званию. Имя Вороного Серпинский вспоминает всегда с большой теплотой\footnote{Вороной умер 20 ноября 1908 г. Спустя три дня Серпинский — доцент Львовского университета — одну из своих лекции по расписанию заменил лекцией о Вороном. Эта лекция опубликована в журнале «Wiadomosci Matematyczne», т. 13, 1909, стр. 1—4.}.
	
	Сохранился диплом Серпинского об окончании университета. Ниже мы воспроизводим текст этого интересного документа, подписанного ректором Варшавского университета П. А. Зиловым и за декана физико-математического факультета профессором Н. Н. Зининым (сыном известного русского химика академика Н. Н. Зинина).
	

	
	\begin{center}
	ДИПЛОМ
	\end{center}
	
\hangindent=1.5cm \hangafter=0  Совет Императорского Варшавского Университета сим объявляет, что Вацлав Франциск (2-х имен) Константинович Серпинский, поступив в число студентов Варшавского Университета в начале 1900/1901 учебного года, выслушал в течение 1900/1901, 1901/2, 1902/3, и 1903/4 учебных годов полный курс наук, преподаваемых на четырех курсах математического отделения Физико-Математического Факультета сего Университета, и на окончательных испытаниях оказал следующие познания: в Геометрии, Анализе, Теории чисел, Теории вероятностей, Механике, Астрономии, Геодезии, Математической физике, Опытной физике, Физической географии и Химии — отличные (5); в Русском языке и сочинении — хорошо (4). Письменный его ответ оценен баллом 5 (отлично).

\hangindent=1.5cm \hangafter=0 Представленное же им сочинение, под девизом «Summa» на тему: «О суммировании ряда $ \sum_{n<\alphaup}^{n\geqslant\betaup} \tauup(n)f(n) $ при условии, что $\tauup(n) $ представляет число разложений $n $ на сумму квадратов двух целых чисел» в заседании Совета 27 Мая 1904 года, награждено золотою медалью. Посему он, Серпинский, согласно примечанию к § 96 Университетского Устава признан Физико-Математическим Факультетом достойным ученой степени Кандидата и, на основании п. 3 л. А § 48 Университетского устава, утвержден в этой степени Советом Университета 19 Июня 1904 года. Вследствие сего, г. Серпинскому предоставляются все права и преимущества, законами Российской империи со степенью Кандидата соединямые. В удостоверении чего дан сей диплом от Совета Императорского Варшавского Университета, с приложением Университетской печати. Г. Варшава, Апреля 1 дня, 1905 года.
	
После окончания университета Серпинский преподавал математику в двух гимназиях Варшавы. Учительская деятельность его была непродолжительной, так как в 1905 г. после забастовки учащейся молодежи, к которой он примкнул, ему пришлось покинуть Варшаву. Серпинский поступил на философское отделение Ягеллонского университета в Кракове, где работали два известных польских математика: Станислав Заремба и Казимир Жоравский; первый был специалистом в теории дифференциальных уравнений, второй — в области геометрии. Уже в 1906 г. Серпинский сдал экзамены по математике, астрономии и философии, обязательные для соискателя докторской степени, и на основании диссертации «О суммировании ряда $ \sum_{(m^2+n^2)\leqslant x} f(m^2+n^2) $» получил ученую степень доктора философии. По возвращении в Варшаву Серпинский преподает математику в частных средних школах, в учительской семинарии и на курсах, игравших роль польского университета (Варшавский университет в 1905—1908 гг. был закрыт), и значительную часть своего времени посвящает научно-исследовательской работе. В 1906 г. появилась его первая печатная работа (на польском языке) под названием «Об одной задаче из теории асимптотических функций». По своей проблематике и методу эта работа примыкает к работе Вороного с таким же названием, опубликованной в 1903 г. в журнале Крелле на французском языке. Интересно отметить, что этот же мемуар Вороного был одним из отправных пунктов для выдающихся исследований академика И. М. Виноградова.

В упомянутой работе Серпинский вывел формулу, позволяющую приближенно вычислять число точек $A(n)$ с целочисленными координатами $x$, $y$ в круге $x^2+y^2\leqslant n$. Формула Серпинского{\footnote{Запись $f(t) = O(g(t))$ означает, что для всех достаточно больших $t$ выполняется неравенство $|f(t)| < Kg(t)$. где $K$ — некоторая постоянная. Приведенная теорема Серпинского была снова доказана в 1913 г. известным немецким математиком Э. Ландау.}} имеет вид:

$$
A(n) = \pi n + O(\sqrt[3]{n})
$$
	
	В другой работе, напечатанной в 1909 г., он предложил новую асимптотическую формулу, дающую число целых точек в шаре $x^2+y^2 + z^2\leqslant n$.
	
	Обе эти работы и многие другие исследования Серпинского выполнены в стиле Петербургской школы, характерными чертами которого являются четкая постановка конкретных вопросов и доведение решения задачи до «алгорифма» — формулы, удобной для вычисления.
	
	В 1907 г. Серпинский опубликовал опять только одну работу, на этот раз из анализа и на французском языке. Начиная с 1908 г. число его печатных работ быстро растет, тематика их становится весьма разнообразной, они появляются на языках польском и французском, причем последним Серпинский пользуется все чаще и чаще. В 1948 г. в списке печатных работ Серпинского значилось 512 мемуаров и 15 монографий и учебников. Выдающийся вклад Серпинского в науку был высоко оценен его соотечественниками и математиками всего мира. VI математический съезд польские математики провели осенью 1948 г., совместив его с 40-летием университетской деятельности Серпинского{\footnote{„VI Polski zjazd matematyczny. Jubileusz 40-lecia dzialalnosci na katedrze uniwersyteckiej profesora Waclawa Sierpinskiego, Warszawa, 23. 9. 1948“. Warszavva, 1949, 94 стр.}}. Много теплых слов было сказано здесь в адрес Серпинского. От математиков Советского Союза юбиляра поздравил А. Н. Колмогоров. Он сказал: «От имени Академии наук СССР и Московского математического общества я приветствую профессора Серпинского с сорокалетием научной деятельности.
	
	Советские математики высоко ценят научные работы профессора Серпинского и его заслуги как создателя польской математической школы, занявшей видное место среди мировых научных школ.
	
	Позвольте пожелать Вам, Вацлав Константинович, долгих лет дальнейшей продуктивной работы».
	
	Обилие работ Серпинского, почти фантастическое число их, не позволяет задерживаться здесь на отдельных работах и вынуждает характеризовать его научное творчество в самых общих чертах. Лишь в виде исключения мы остановимся здесь на характеристике четырех из девяти работ, опубликованных Серпинским в 1908 г.
	
	Эти ранние работы Серпинского, как и его первая печатная работа, примечательны в том отношении, что в них сразу же раскрывается математическое дарование автора и его весьма высокая научная квалификация.
	
	В большой работе «О суммировании ряда $ \sum \tauup(n)f(n)$...», в основу которой Серпинский положил свое студенческое сочинение, среди различных арифметических результатов мы встречаем оценки для сумм вида
	
	$$
	\sum_{n=1}^{x} \tauup(n^2),\ \ \  	\sum_{n=1}^{x} \tauup^2(n),\ \ \ \sum_{n=1}^{x} \tauup_8(n),
	$$
	
\noindent	где $\tauup(n)$ и $\tauup_8(n)$ обозначают соответственно число разложений $n$ на 2 и 8 квадратов.

В другой работе под названием «Об одном случае ошибочного применения правила умножения вероятностей» Серпинский показывает, что вероятность того, что два натуральных числа, не превосходящих $n$, являются взаимно простыми, равна

$$
\frac{1}{n^2} \sum_{k=1}^{n} \muup(k) \left[ \frac{n}{k} \right]^2
$$

\noindent (где символ $\muup$ означает функцию Мёбиуса, а квадратные скобки — целую часть), вопреки тому, что сообщает П. Бахман в своей книге «Die analytische Zahlehtheorie» (Leipzig, 1894, стр. 430).

Новый классический результат Серпинский получает в работе «О разложении целых чисел на разность двух квадратов». Здесь он показал, что число различных представлений натурального числа $n$ в виде разности двух квадратов равно удвоенной разности между числом четных и числом нечетных делителей $n$.

В годы учебы Серпинского в университетах еще не изучались вопросы теории множеств. О трудах основоположника теории множеств Георга Кантора (1845—1918) многие математики либо ничего не знали, либо имели лишь смутное представление. Открыв совершенно самостоятельно в 1907 г. один любопытнейший факт из теории множеств, Серпинский написал о нем в Гёттинген Банахевичу. Последний сразу же ответил телеграммой, текст которой содержал одно лишь слово «Кантор», и вскоре прислал соответствующую литературу. С этого времени одним из главных предметов занятий Серпинского становится теория множеств с ее выходами в топологию, теорию функций действительного переменного, математическую логику и другие области математики.

Первая работа Серпинского по теории множеств была опубликована в 1908 г. под названием «Об одной теореме Кантора»; в ней Серпинский дал найденное им независимо от Кантора доказательство известной ныне каждому студенту теоремы о том, что положение точки на плоскости может быть определено одним действительным числом, из чего уже легко следует эквивалентность множеств точек прямой и плоскости, и вообще пространств любого числа измерений.

В дальнейшем Серпинский получил большое количество важных и глубоких результатов, относящихся как к абстрактной теории множеств, так и к ее топологическим приложениям (в связи с исследованием проблемы размерности), а особенно — к проблематике, пограничной между собственно теорией множеств и математической логикой. Здесь в первую очередь следует отметить изучение (самим Серпинским, а затем и его многочисленными учениками) обширного класса предложений, эквивалентных знаменитой континуум-гипотезе Кантора и так называемой аксиоме выбора теории множеств, и геометрических следствий этой аксиомы, носящих зачастую внешне парадоксальный характер{\footnote{Подробнее об этой проблематике см. А. Френкель и И. Бар-Хиллел. Основания теории множеств, пер. с англ., М., «Мир», 1966, гл. II; первоначальные сведения можно также найти в книжке Серпинского «О теории множеств», русский перевод которой в 1966 г. положил начало серии «Математическое просвещение». — \textit{Прим. ред.}}}.

Перу Серпинского принадлежит более десятка капитальных трудов по теории множеств, теории функций и топологии, в том числе ставшие уже классическими монографии «Lemons sur les nombres transfinis» («Лекции о трансфинитных числах»), опубликованная в 1950 г. в Париже, и «Cardinal and ordinal numbers» («Кардинальные и порядковые числа»), вышедшая в 1958 г. в Варшаве.

Начало деятельности Серпинского в математике было весьма удачным, и он вскоре приобрел известность. С осени 1908 г. Серпинский работает во Львовском университете, куда его пригласил тогдашний ректор, специалист по теории аналитических функций И. Пужина. Уже в следующем учебном году он прочитал курс лекций под названием «Теория множеств», который, как свидетельствует чешский историк математики Гвидо Феттер, был первым в мире самостоятельным университетским курсом теории множеств. К этим лекциям студенты проявили особый интерес. Для некоторых из них этот курс определил область, в которой позднее они прославились как видные исследователи. Среди первых учеников Серпинского были студенты О. Никодым, теперь профессор одного из американских университетов, и С. Ружевич, позднее профессор Львовского университета и ректор Академии внешней торговли, убитый немецко-фашистскими захватчиками в 1941 г. вместе с несколькими десятками профессоров Львова.

В 1910 г. Львовский университет присвоил Серпинскому звание профессора, а спустя год Краковская Академия наук наградила его за труды, опубликованные в 1909—1910 гг. на польском языке. Деятельность Серпинского привлекает внимание молодых талантливых математиков. В 1913 г. во Львов прибыли С. Мазуркевич, чтобы пройти у него докторантуру, и 3. Янишевский, уже получивший степень доктора в Парижском университете за работу по топологии. Круг учеников и сотрудников Серпинского, проявляющих интерес к теоретико-множественной тематике, заметно расширяется, и здесь во Львове, городе, входящем тогда в состав Австро-Венгрии, зарождается новая математическая школа — Польская.

Примерно в это же время в России возникает новая математическая школа — Московская школа теории функций действительного переменного.

Идеи теоретико-множественной математики проникли в русскую литературу уже в самом начале XX века. К этому времени относится первый курс лекций по теории функций действительного переменного, прочитанный в Московском университете Б. К. Млодзеевским, опубликование в 1907 г. И. И. Жегалкиным магистерской диссертации «Трансфинитные числа» и, наконец, появление в 1911 г. знаменитой работы Д. Ф. Егорова «О последовательности измеримых функций».

Решающее значение для возникновения новой математической школы имела деятельность в области теории функций Н. Н. Лузина (1883—1950), ученика Д. Ф. Егорова по Московскому университету. Свои первые работы (они появляются уже в 1911 г.) Лузин присылает из Гёттингена и Парижа, где на протяжении четырех лет (1910—1914) он слушает лекции крупнейших математиков и ведет интенсивную научную работу.

Появление в 1915 г. докторской диссертации Лузина «Интеграл и тригонометрический ряд» оказало сильное влияние на дальнейшее развитие теории функций. По-видимому, с этого времени Лузин становится признанным главой Московской математической школы, сразу заявившей о себе выдающимися исследованиями самого Лузина и его учеников: П. С. Александрова, Д. Е . Меньшова, М. Я. Суслина и А. Я. Хинчина.

Полный расцвет школы Лузина приходится на советский период, когда, наряду с работами Лузина и его первых учеников, появляются работы А. Н. Колмогорова, М. А. Лаврентьева, П. С. Новикова, П. С. Урысона, Л. В. Келдыш и других исследователей.

Влияние Московской школы на развитие математики в нашей стране становилось все более и более ощутительным в связи с проникновением идей теории множеств в функциональный анализ, теорию вероятностей, теорию дифференциальных уравнений и в другие отрасли математики.

Влияние Московской школы сказалось и на развитии математики за рубежом. «Особенно значительным было влияние на польскую математику, где возникла сильная школа теории функций (В. Серпинский, Г. Штейнгауз, С. Мазуркевич, А. Райхман, А. Зыгмунд, С. Банах и др.); оно сказывалось и на творчестве французских, английских, немецких и японских ученых»{\footnote{А .П. Юшкевич, Математика. В кн.: «История естествознания в России», т. 2, М., изд-во АН СССР, 1960, стр. 194.}}.

Научная и литературная деятельность Серпинского уже в самом начале получает в Польше высокое признание. В 1913 г. Краковская Академия наук присуждает Серпинскому премию за «Очерк теории множеств», а в 1917 г. — за монографию «Теория чисел». Обе книги были опубликованы в Варшаве на польском языке: первая в 1912 г., вторая в 1914 г.

В начале первой мировой войны Серпинский был интернирован (как гражданин г. Львова, входящего тогда в состав Австро-Венгрии) и направлен в г. Вятку. Но здесь Серпинский пробыл сравнительно недолго, так как Д. Ф. Егорову и Б. К. Млодзеевскому после больших хлопот и усилий удалось получить для него разрешение на жительство в Москве.

В 1915 г. Серпинский приехал в Москву, где на протяжении почти трех лет он продолжал свою научную и литературную деятельность и имел полезные контакты с русскими математиками. Возможность общения с Серпинским радовала многих московских математиков. Так, П. С. Александров (в то время студент Лузина) замечает, что он был счастлив, когда осенью 1915 г. ему довелось докладывать о своей первой научной работе в присутствии Серпинского{\footnote{См: P. S. Aleksandrow. О wspolpracy polskiej i radzieckiej szkoly matematycznej, Wiadomosci matematyczne, VI, 1963, стр. 177.
}}. С чувством большой благодарности Серпинский вспоминает о внимании и заботе, которые были проявлены к нему Егоровым, Млодзеевским и другими московскими математиками. Но совершенно особое значение для Серпинского имела возникшая здесь большая дружба между ним и Лузиным{\footnote{Сохранилась фотография, на которой сняты Егоров, Лузин и Серпинский. См.: «Ист.-мат исслед.», вып. VIII. М., 1955 стр. 70.}}. Эта дружба, основанная на общности научных интересов, закрепленная совместными исследованиями и результатами, служила источником вдохновения для обоих математиков на протяжении нескольких десятилетий, до самой смерти Лузина.

В Вятке и в Москве Серпинского не покидала мысль о создании польского университета в Варшаве. Здесь он подготовил первый том «Математического анализа» в двух частях и опубликовал его в Москве на польском языке в 1916—1917 гг. Эта книга была переиздана в 1923 г. в Варшаве и на протяжении ряда лет служила учебным пособием для польских студентов. За 1915—1918 гг. Серпинским было опубликовано 36 работ (из коих четыре совместно с Лузиным), т. е. столько же, сколько за четыре предыдущих года.
	
	\newpage
	\tableofcontents
	
	\thispagestyle{empty} % 
	
	\newpage
	
	\setcounter{secnumdepth}{0}  
	
	\phantomsection
	
		\section*{Описание}
	
	{\bf Название:} 250 задач по элементарной теории чисе
	
{\bf Автор:} Вацлав Франциск Серпинский

{\bf Переводчик (с польского):} И.Г. Мельников
	
{\bf Издательство:} Москва: Просвещение, 1968
	
		{\bf Редактор:} Ю. А. Гастев
	
		{\bf Художественный редактор:} В. С. Эрденко
	
		{\bf Технический редактор:} Н. Ф. Макарова
	
		{\bf Корректоры:} К. А. Иванова
	
		{\bf Аннотация:} Сборник задач по элементарной теории чисел (от совсем простых до довольно трудных), с решениями и комментариями. Может быть использована в работе школьных и студенческих математических кружков.
		\thispagestyle{empty} % выключаем отображение номера для этой страницы

	
\end{document}


