
%\documentclass[oneside,final,14pt]{extreport}
\documentclass[12pt, a4paper, openany]{book}
\usepackage[left=1cm,right=1cm,top=2cm,bottom=2cm,bindingoffset=0cm]{geometry}
%\usepackage[koi8-r]{inputenc}
%\usepackage[russianb]{babel}
\usepackage{vmargin}

\setpapersize{A4}
\usepackage[T2A]{fontenc}
\usepackage[utf8x]{inputenc}
\usepackage[english, russian]{babel}
\setmarginsrb{2cm}{2cm}{2cm}{2cm}{0pt}{5mm}{0pt}{0mm}
\usepackage{indentfirst}
\usepackage{nicefrac} % For comparison
%\usepackage{xfrac}    % Works better with other fonts
%\usepackage[unicode, pdftex]{hyperref}
\usepackage{lettrine}
\usepackage[usenames]{color}
\usepackage{colortbl}
\usepackage{mathtext}
\usepackage{epigraph}
\usepackage{amsmath, amsfonts, amssymb, mathrsfs}
%\usepackage{mathptmx}
%\usepackage{txfonts}
\usepackage{pxfonts}
\usepackage[pagestyles]{titlesec}
\usepackage{ebgaramond}
\usepackage{awesomebox}
\usepackage{enumitem}

    \usepackage{etoolbox}
\makeatletter
\newlength\epitextskip
\pretocmd{\@epitext}{\em}{}{}
\apptocmd{\@epitext}{\em}{}{}
\patchcmd{\epigraph}{\@epitext{#1}\\}{\@epitext{#1}\\[\epitextskip]}{}{}
\makeatother

\setlength\epigraphrule{0pt}
\setlength\epitextskip{2ex}
\setlength\epigraphwidth{.8\textwidth}

\usepackage{xfrac}    % Works better with other fonts
\usepackage[colorlinks=true,linkcolor=black,urlcolor=black,bookmarksopen=true]{hyperref}

\usepackage{fancyhdr} % пакет для установки колонтитулов
\pagestyle{fancy} % смена стиля оформления страниц
\fancyhf{} % очистка текущих значений
\fancyhead[C]{\thepage} % установка верхнего колонтитула
\renewcommand{\headrulewidth}{0pt} % убрать разделительную линию


% Настройка вертикальных и горизонтальных отступов
\titlespacing{\chapter}{0pt}{5pt}{5pt}
\titlespacing{\section}{\parindent}{4mm}{4mm}
\titlespacing{\subsection}{\parindent}{3mm}{3mm}


% Настройка задачи со зведочкой
\newcounter{namedlistcounter}  % number the items
\newenvironment{withdot}
{\begin{list}
		{\arabic{namedlistcounter}*.} % labeling 
		{\usecounter{namedlistcounter}   % set counter
			\setlength{\leftmargin}{3em}} % set spacing 
	}
	{\end{list}}


\newcommand{\anonsection}[1]{ \section*{#1} \addcontentsline{toc}{section}{\numberline {}#1}} 

\makeatletter %%%%% <---- Starting chapter without a pagebreak
\renewcommand\chapter{\par%
	\thispagestyle{plain}% \global\@topnum\z@
	\@afterindentfalse \secdef\@chapter\@schapter}
\makeatother %%%%% <---- Starting chapter without a pagebreak
\titleformat{\chapter}[display]
{\normalfont\bfseries}{}{0pt}{\Large}

\newpagestyle{mystyle}{
	\sethead[\thepage][][]{}{}{\thepage}	
}

\renewcommand{\rmdefault}{cmr}

\pagestyle{mystyle}
\sloppy
\begin{document}

	
	\begin{titlepage}
		
		\begin{center}
			%\vfill
			
			%\vfill
			\topskip0pt
			\vspace*{\fill}
			\begin{flushleft}
		{\large	\textit{Математическое просвещение}			}
			\end{flushleft}
			\ \\
\ \\
			\ \\
\ \\
			{\large\bf ВАЦЛАВ СЕРПИНСКИЙ\\}
			\ \\
			\ \\
			{\Huge\bf 250 задач по элементарной теории чисел\\}
			\ \\
			\ \\
			\ \\
			\textit{Перевод с польского И.Г. Мельникова}

			\vspace*{\fill}    
			
			\vfill
			
И З Д А Т Е Л Ь С Т В О

«П Р О С В Е Щ Е Н И Е»

М о с к в а 1968
		\end{center}
		
	\end{titlepage}
	
	\thispagestyle{empty} % выключаем отображение номера для этой страницы
	
	\newpage
	
\setcounter{secnumdepth}{0}  
	
	\section[Выдающийся польский математик Вацлав Серпинский (к 85-летию со дня рождения)]{\center ВЫДАЮЩИИСЯ ПОЛЬСКИЙ МАТЕМАТИК ВАЦЛАВ СЕРПИНСКИЙ}
		\begin{center}
\textit{(К 85-летию со дня рождения)}
\end{center}
	
	Четырнадцатого марта 1882 г. в Варшаве в семье врача Константина Серпинского родился мальчик, которому дали два имени: Вацлав Франциск. Этому мальчику суждено было стать одним из крупнейших польских математиков.
	Образование Вацлав Серпинский получил в Варшаве. Здесь он окончил гимназию и университет.

	Незаурядные способности Серпинского обнаружились рано, повышенный же интерес к математике наметился лишь в последних классах гимназии под влиянием двух его соучеников, владевших некоторыми разделами высшей математики, и прекрасного учителя математики Влодзимежа Влодарского. Последний был очень высокого мнения о математических способностях Серпинского. В гимназии у Серпинского было еще несколько замечательных учителей. Так, его учителем французского языка был К. Аппель, впоследствии профессор Варшавского университета.

	Среди сверстников Серпинского по гимназии было немало способных людей. Из класса, в котором учился Серпинский, вышло заметное число ученых и деятелей культуры, из коих отметим известного астронома Тадеуша Банахевича.

	Уже в школьные годы Серпинский проявлял большой интерес к общественным делам. Вместе с несколькими своими друзьями он организовал тайную школу для мальчиков из рабочей среды. Эта хорошо законспирированная школа на протяжении ряда лет успешно готовила своих учащихся к экзамену за четыре класса гимназии.
	
	В 1900 г. Серпинский поступил на физико-математический факультет Варшавского университета, который в ту пору представлял собой молодое учебное заведение с преподаванием на русском языке, существовавшее всего около трех десятилетий\footnote{Этот университет был создан на базе Главной школы, существовавшей в Варшаве в 1862—1869 гг. С начала XIX столетия до 1832 г. в Варшаве был польский университет.}.
	
	Следует заметить, что поляки, имевшие возможность учиться в старинных польских университетах (в Краковском и Львовском), охотно шли и в новый университет. Трудности, которые испытывал университет в первые годы своего существования (отсутствие традиций, хороших научно-педагогических кадров и др.), вскоре были преодолены.
	
	Активная и разнообразная деятельность работавших здесь математиков М. А. Андреевского, Н. Н. Алексеева и Н. Я. Сонина на первой стадии, а затем В. А. Анисимова, Н. Н. Зинина и Г. Ф. Вороного позволила уже к концу XIX в. приблизить уровень преподавания математики в Варшавском университете к уровню преподавания математики в университетах Петербурга, Москвы, Казани, Харькова, Дерпта (Тарту)\footnote{Ср.: С. Е. Белозеров. Математика в Ростовском университете. Ист.-мат. исслед., вып. VI. М., Гостехиздат, 1953, стр. 247—352.}.
	
	
	Наибольшее влияние на Серпинского оказал питомец Петербургского университета профессор математики, впоследствии член-корреспондент Российской Академии наук, Георгий Федосеевич Вороной (1868—1908). Деятельность Вороного в Варшавском университете началась в 1894 г. и продолжалась там с небольшими перерывами до его безвременной смерти. Вороной — первоклассный ученый, на трудах которого лежит печать гениальности. Вместе с Германом Минковским он является создателем геометрии чисел. Глубокие и важные результаты были получены им в аналитической теории чисел, а также в теории алгебраических чисел. Его проблематика разрабатывалась в нашей стране Б. А. Венковым (1900—1962), Б. Н. Делоне (род. в 1890 г.), Д. К. Фаддеевым (род. в 1907 г.) и др., а также зарубежными математиками. Г. Ф. Вороной принадлежал к Петербургской школе теории чисел, и в ней он занимал одно из самых видных мест.
	
	Серпинский прослушал несколько лекционных курсов у Вороного и выполнил свою первую научную работу по аналитической теории чисел в духе идей и методов Вороного на тему, предложенную последним для конкурсных студенческих сочинений. Подробный отзыв Вороного на эту работу был напечатан в VI выпуске «Варшавских университетских известий» за 1904 г. Вороной ходатайствовал о присуждении Серпинскому золотой медали и об оставлении его при университете для подготовки к профессорскому званию. Имя Вороного Серпинский вспоминает всегда с большой теплотой\footnote{Вороной умер 20 ноября 1908 г. Спустя три дня Серпинский — доцент Львовского университета — одну из своих лекции по расписанию заменил лекцией о Вороном. Эта лекция опубликована в журнале «Wiadomosci Matematyczne», т. 13, 1909, стр. 1—4.}.
	
	Сохранился диплом Серпинского об окончании университета. Ниже мы воспроизводим текст этого интересного документа, подписанного ректором Варшавского университета П. А. Зиловым и за декана физико-математического факультета профессором Н. Н. Зининым (сыном известного русского химика академика Н. Н. Зинина).
	

	
	\begin{center}
	ДИПЛОМ
	\end{center}
	
\hangindent=1.5cm \hangafter=0  Совет Императорского Варшавского Университета сим объявляет, что Вацлав Франциск (2-х имен) Константинович Серпинский, поступив в число студентов Варшавского Университета в начале 1900/1901 учебного года, выслушал в течение 1900/1901, 1901/2, 1902/3, и 1903/4 учебных годов полный курс наук, преподаваемых на четырех курсах математического отделения Физико-Математического Факультета сего Университета, и на окончательных испытаниях оказал следующие познания: в Геометрии, Анализе, Теории чисел, Теории вероятностей, Механике, Астрономии, Геодезии, Математической физике, Опытной физике, Физической географии и Химии — отличные (5); в Русском языке и сочинении — хорошо (4). Письменный его ответ оценен баллом 5 (отлично).

\hangindent=1.5cm \hangafter=0 Представленное же им сочинение, под девизом «Summa» на тему: «О суммировании ряда $ \sum_{n<\alpha}^{n\geqslant\beta} \tau(n)f(n) $ при условии, что $\tau(n) $ представляет число разложений $n $ на сумму квадратов двух целых чисел» в заседании Совета 27 Мая 1904 года, награждено золотою медалью. Посему он, Серпинский, согласно примечанию к § 96 Университетского Устава признан Физико-Математическим Факультетом достойным ученой степени Кандидата и, на основании п. 3 л. А § 48 Университетского устава, утвержден в этой степени Советом Университета 19 Июня 1904 года. Вследствие сего, г. Серпинскому предоставляются все права и преимущества, законами Российской империи со степенью Кандидата соединямые. В удостоверении чего дан сей диплом от Совета Императорского Варшавского Университета, с приложением Университетской печати. Г. Варшава, Апреля 1 дня, 1905 года.
	
После окончания университета Серпинский преподавал математику в двух гимназиях Варшавы. Учительская деятельность его была непродолжительной, так как в 1905 г. после забастовки учащейся молодежи, к которой он примкнул, ему пришлось покинуть Варшаву. Серпинский поступил на философское отделение Ягеллонского университета в Кракове, где работали два известных польских математика: Станислав Заремба и Казимир Жоравский; первый был специалистом в теории дифференциальных уравнений, второй — в области геометрии. Уже в 1906 г. Серпинский сдал экзамены по математике, астрономии и философии, обязательные для соискателя докторской степени, и на основании диссертации «О суммировании ряда $ \sum_{(m^2+n^2)\leqslant x} f(m^2+n^2) $» получил ученую степень доктора философии. По возвращении в Варшаву Серпинский преподает математику в частных средних школах, в учительской семинарии и на курсах, игравших роль польского университета (Варшавский университет в 1905—1908 гг. был закрыт), и значительную часть своего времени посвящает научно-исследовательской работе. В 1906 г. появилась его первая печатная работа (на польском языке) под названием «Об одной задаче из теории асимптотических функций». По своей проблематике и методу эта работа примыкает к работе Вороного с таким же названием, опубликованной в 1903 г. в журнале Крелле на французском языке. Интересно отметить, что этот же мемуар Вороного был одним из отправных пунктов для выдающихся исследований академика И. М. Виноградова.

В упомянутой работе Серпинский вывел формулу, позволяющую приближенно вычислять число точек $A(n)$ с целочисленными координатами $x$, $y$ в круге $x^2+y^2\leqslant n$. Формула Серпинского{\footnote{Запись $f(t) = O(g(t))$ означает, что для всех достаточно больших $t$ выполняется неравенство $|f(t)| < Kg(t)$. где $K$ — некоторая постоянная. Приведенная теорема Серпинского была снова доказана в 1913 г. известным немецким математиком Э. Ландау.}} имеет вид:

$$
A(n) = \pi n + O(\sqrt[3]{n})
$$
	
	В другой работе, напечатанной в 1909 г., он предложил новую асимптотическую формулу, дающую число целых точек в шаре $x^2+y^2 + z^2\leqslant n$.
	
	Обе эти работы и многие другие исследования Серпинского выполнены в стиле Петербургской школы, характерными чертами которого являются четкая постановка конкретных вопросов и доведение решения задачи до «алгорифма» — формулы, удобной для вычисления.
	
	В 1907 г. Серпинский опубликовал опять только одну работу, на этот раз из анализа и на французском языке. Начиная с 1908 г. число его печатных работ быстро растет, тематика их становится весьма разнообразной, они появляются на языках польском и французском, причем последним Серпинский пользуется все чаще и чаще. В 1948 г. в списке печатных работ Серпинского значилось 512 мемуаров и 15 монографий и учебников. Выдающийся вклад Серпинского в науку был высоко оценен его соотечественниками и математиками всего мира. VI математический съезд польские математики провели осенью 1948 г., совместив его с 40-летием университетской деятельности Серпинского{\footnote{„VI Polski zjazd matematyczny. Jubileusz 40-lecia dzialalnosci na katedrze uniwersyteckiej profesora Waclawa Sierpinskiego, Warszawa, 23. 9. 1948“. Warszavva, 1949, 94 стр.}}. Много теплых слов было сказано здесь в адрес Серпинского. От математиков Советского Союза юбиляра поздравил А. Н. Колмогоров. Он сказал: «От имени Академии наук СССР и Московского математического общества я приветствую профессора Серпинского с сорокалетием научной деятельности.
	
	Советские математики высоко ценят научные работы профессора Серпинского и его заслуги как создателя польской математической школы, занявшей видное место среди мировых научных школ.
	
	Позвольте пожелать Вам, Вацлав Константинович, долгих лет дальнейшей продуктивной работы».
	
	Обилие работ Серпинского, почти фантастическое число их, не позволяет задерживаться здесь на отдельных работах и вынуждает характеризовать его научное творчество в самых общих чертах. Лишь в виде исключения мы остановимся здесь на характеристике четырех из девяти работ, опубликованных Серпинским в 1908 г.
	
	Эти ранние работы Серпинского, как и его первая печатная работа, примечательны в том отношении, что в них сразу же раскрывается математическое дарование автора и его весьма высокая научная квалификация.
	
	В большой работе «О суммировании ряда $ \sum \tauup(n)f(n)$...», в основу которой Серпинский положил свое студенческое сочинение, среди различных арифметических результатов мы встречаем оценки для сумм вида
	
	$$
	\sum_{n=1}^{x} \tauup(n^2),\ \ \  	\sum_{n=1}^{x} \tauup^2(n),\ \ \ \sum_{n=1}^{x} \tauup_8(n),
	$$
	
\noindent	где $\tauup(n)$ и $\tauup_8(n)$ обозначают соответственно число разложений $n$ на 2 и 8 квадратов.

В другой работе под названием «Об одном случае ошибочного применения правила умножения вероятностей» Серпинский показывает, что вероятность того, что два натуральных числа, не превосходящих $n$, являются взаимно простыми, равна

$$
\frac{1}{n^2} \sum_{k=1}^{n} \muup(k) \left[ \frac{n}{k} \right]^2
$$

\noindent (где символ $\muup$ означает функцию Мёбиуса, а квадратные скобки — целую часть), вопреки тому, что сообщает П. Бахман в своей книге «Die analytische Zahlehtheorie» (Leipzig, 1894, стр. 430).

Новый классический результат Серпинский получает в работе «О разложении целых чисел на разность двух квадратов». Здесь он показал, что число различных представлений натурального числа $n$ в виде разности двух квадратов равно удвоенной разности между числом четных и числом нечетных делителей $n$.

В годы учебы Серпинского в университетах еще не изучались вопросы теории множеств. О трудах основоположника теории множеств Георга Кантора (1845—1918) многие математики либо ничего не знали, либо имели лишь смутное представление. Открыв совершенно самостоятельно в 1907 г. один любопытнейший факт из теории множеств, Серпинский написал о нем в Гёттинген Банахевичу. Последний сразу же ответил телеграммой, текст которой содержал одно лишь слово «Кантор», и вскоре прислал соответствующую литературу. С этого времени одним из главных предметов занятий Серпинского становится теория множеств с ее выходами в топологию, теорию функций действительного переменного, математическую логику и другие области математики.

Первая работа Серпинского по теории множеств была опубликована в 1908 г. под названием «Об одной теореме Кантора»; в ней Серпинский дал найденное им независимо от Кантора доказательство известной ныне каждому студенту теоремы о том, что положение точки на плоскости может быть определено одним действительным числом, из чего уже легко следует эквивалентность множеств точек прямой и плоскости, и вообще пространств любого числа измерений.

В дальнейшем Серпинский получил большое количество важных и глубоких результатов, относящихся как к абстрактной теории множеств, так и к ее топологическим приложениям (в связи с исследованием проблемы размерности), а особенно — к проблематике, пограничной между собственно теорией множеств и математической логикой. Здесь в первую очередь следует отметить изучение (самим Серпинским, а затем и его многочисленными учениками) обширного класса предложений, эквивалентных знаменитой континуум-гипотезе Кантора и так называемой аксиоме выбора теории множеств, и геометрических следствий этой аксиомы, носящих зачастую внешне парадоксальный характер{\footnote{Подробнее об этой проблематике см. А. Френкель и И. Бар-Хиллел. Основания теории множеств, пер. с англ., М., «Мир», 1966, гл. II; первоначальные сведения можно также найти в книжке Серпинского «О теории множеств», русский перевод которой в 1966 г. положил начало серии «Математическое просвещение». — \textit{Прим. ред.}}}.

Перу Серпинского принадлежит более десятка капитальных трудов по теории множеств, теории функций и топологии, в том числе ставшие уже классическими монографии «Lemons sur les nombres transfinis» («Лекции о трансфинитных числах»), опубликованная в 1950 г. в Париже, и «Cardinal and ordinal numbers» («Кардинальные и порядковые числа»), вышедшая в 1958 г. в Варшаве.

Начало деятельности Серпинского в математике было весьма удачным, и он вскоре приобрел известность. С осени 1908 г. Серпинский работает во Львовском университете, куда его пригласил тогдашний ректор, специалист по теории аналитических функций И. Пужина. Уже в следующем учебном году он прочитал курс лекций под названием «Теория множеств», который, как свидетельствует чешский историк математики Гвидо Феттер, был первым в мире самостоятельным университетским курсом теории множеств. К этим лекциям студенты проявили особый интерес. Для некоторых из них этот курс определил область, в которой позднее они прославились как видные исследователи. Среди первых учеников Серпинского были студенты О. Никодым, теперь профессор одного из американских университетов, и С. Ружевич, позднее профессор Львовского университета и ректор Академии внешней торговли, убитый немецко-фашистскими захватчиками в 1941 г. вместе с несколькими десятками профессоров Львова.

В 1910 г. Львовский университет присвоил Серпинскому звание профессора, а спустя год Краковская Академия наук наградила его за труды, опубликованные в 1909—1910 гг. на польском языке. Деятельность Серпинского привлекает внимание молодых талантливых математиков. В 1913 г. во Львов прибыли С. Мазуркевич, чтобы пройти у него докторантуру, и 3. Янишевский, уже получивший степень доктора в Парижском университете за работу по топологии. Круг учеников и сотрудников Серпинского, проявляющих интерес к теоретико-множественной тематике, заметно расширяется, и здесь во Львове, городе, входящем тогда в состав Австро-Венгрии, зарождается новая математическая школа — Польская.

Примерно в это же время в России возникает новая математическая школа — Московская школа теории функций действительного переменного.

Идеи теоретико-множественной математики проникли в русскую литературу уже в самом начале XX века. К этому времени относится первый курс лекций по теории функций действительного переменного, прочитанный в Московском университете Б. К. Млодзеевским, опубликование в 1907 г. И. И. Жегалкиным магистерской диссертации «Трансфинитные числа» и, наконец, появление в 1911 г. знаменитой работы Д. Ф. Егорова «О последовательности измеримых функций».

Решающее значение для возникновения новой математической школы имела деятельность в области теории функций Н. Н. Лузина (1883—1950), ученика Д. Ф. Егорова по Московскому университету. Свои первые работы (они появляются уже в 1911 г.) Лузин присылает из Гёттингена и Парижа, где на протяжении четырех лет (1910—1914) он слушает лекции крупнейших математиков и ведет интенсивную научную работу.

Появление в 1915 г. докторской диссертации Лузина «Интеграл и тригонометрический ряд» оказало сильное влияние на дальнейшее развитие теории функций. По-видимому, с этого времени Лузин становится признанным главой Московской математической школы, сразу заявившей о себе выдающимися исследованиями самого Лузина и его учеников: П. С. Александрова, Д. Е . Меньшова, М. Я. Суслина и А. Я. Хинчина.

Полный расцвет школы Лузина приходится на советский период, когда, наряду с работами Лузина и его первых учеников, появляются работы А. Н. Колмогорова, М. А. Лаврентьева, П. С. Новикова, П. С. Урысона, Л. В. Келдыш и других исследователей.

Влияние Московской школы на развитие математики в нашей стране становилось все более и более ощутительным в связи с проникновением идей теории множеств в функциональный анализ, теорию вероятностей, теорию дифференциальных уравнений и в другие отрасли математики.

Влияние Московской школы сказалось и на развитии математики за рубежом. «Особенно значительным было влияние на польскую математику, где возникла сильная школа теории функций (В. Серпинский, Г. Штейнгауз, С. Мазуркевич, А. Райхман, А. Зыгмунд, С. Банах и др.); оно сказывалось и на творчестве французских, английских, немецких и японских ученых»{\footnote{А .П. Юшкевич, Математика. В кн.: «История естествознания в России», т. 2, М., изд-во АН СССР, 1960, стр. 194.}}.

Научная и литературная деятельность Серпинского уже в самом начале получает в Польше высокое признание. В 1913 г. Краковская Академия наук присуждает Серпинскому премию за «Очерк теории множеств», а в 1917 г. — за монографию «Теория чисел». Обе книги были опубликованы в Варшаве на польском языке: первая в 1912 г., вторая в 1914 г.

В начале первой мировой войны Серпинский был интернирован (как гражданин г. Львова, входящего тогда в состав Австро-Венгрии) и направлен в г. Вятку. Но здесь Серпинский пробыл сравнительно недолго, так как Д. Ф. Егорову и Б. К. Млодзеевскому после больших хлопот и усилий удалось получить для него разрешение на жительство в Москве.

В 1915 г. Серпинский приехал в Москву, где на протяжении почти трех лет он продолжал свою научную и литературную деятельность и имел полезные контакты с русскими математиками. Возможность общения с Серпинским радовала многих московских математиков. Так, П. С. Александров (в то время студент Лузина) замечает, что он был счастлив, когда осенью 1915 г. ему довелось докладывать о своей первой научной работе в присутствии Серпинского{\footnote{См: P. S. Aleksandrow. О wspolpracy polskiej i radzieckiej szkoly matematycznej, Wiadomosci matematyczne, VI, 1963, стр. 177.
}}. С чувством большой благодарности Серпинский вспоминает о внимании и заботе, которые были проявлены к нему Егоровым, Млодзеевским и другими московскими математиками. Но совершенно особое значение для Серпинского имела возникшая здесь большая дружба между ним и Лузиным{\footnote{Сохранилась фотография, на которой сняты Егоров, Лузин и Серпинский. См.: «Ист.-мат исслед.», вып. VIII. М., 1955 стр. 70.}}. Эта дружба, основанная на общности научных интересов, закрепленная совместными исследованиями и результатами, служила источником вдохновения для обоих математиков на протяжении нескольких десятилетий, до самой смерти Лузина.

В Вятке и в Москве Серпинского не покидала мысль о создании польского университета в Варшаве. Здесь он подготовил первый том «Математического анализа» в двух частях и опубликовал его в Москве на польском языке в 1916—1917 гг. Эта книга была переиздана в 1923 г. в Варшаве и на протяжении ряда лет служила учебным пособием для польских студентов. За 1915—1918 гг. Серпинским было опубликовано 36 работ (из коих четыре совместно с Лузиным), т. е. столько же, сколько за четыре предыдущих года.

Весной 1917 г. из польских газет, выходящих в Москве, Серпинский неожиданно узнал, что Краковская Академия наук избрала его своим членом-корреспондентом. Это известие его очень обрадовало. Вскоре после Октябрьской революции Серпинский возвращается во Львов и приступает к работе в университете. Осенью 1918 г. он получает кафедру во вновь созданном Варшавском университете. В Варшаве Серпинский застал своих друзей профессоров 3. Янишевского и С. Мазуркевича, с которыми он сразу же приступил к осуществлению программы развития математики в Польше. В 1919 г. эти три математика приняли решение о создании первого в мире специализированного математического журнала «Fundamenta Mathematicae», посвященного теории множеств, топологии, теории функций действительного переменного и математической логике. Тогда многие видные математики (среди которых был А. Лебег) не верили в успех этого начинания, им казалось, что журнал, игнорирующий остальные отрасли математики, не будет жизнестойким. Первый том журнала появился в 1920 г., через несколько месяцев после смерти 3. Янишевского — одного из его основателей. Начатое дело Серпинский продолжал с Мазуркевичем до смерти последнего в 1945 г., затем нелегкие обязанности редактора он выполнял с К. Куратовским, в последние же годы Серпинский является почетным редактором журнала, а Куратовский — редактором. Этот журнал сыграл большую роль в развитии математики не только в Польше, но и во всех странах, где ею занимаются. Еще в 1935 г., когда вышел 25-й том журнала, один американский математик сказал, что история «F. М.» является одновременно историей современной теории функций действительного переменного, а в 1962 г., когда вышел 50-й том, П. С. Александров заявил, что юбилей этого журнала является праздником для математиков всего мира. Серпинский как-то заметил, что в 50 томах «F. М.» содержится 1500 работ 420 различных авторов, в том числе около 300 зарубежных, среди которых немало крупнейших математиков современности. Думается, что в этой связи уместно подчеркнуть особую ценность вклада польских математиков и, в частности, самого Серпинского, которому из упомянутых им 1500 работ принадлежат 262.

В 1921 г. Серпинский был избран действительным членом Польской Академии наук. Во многих странах мира обращают внимание на его исключительно высокую творческую активность, выдающиеся литературные, педагогические и организационные способности. Серпинский получает приглашения от многих зарубежных университетов. Он читает лекции в Страсбурге, Сорбонне, Яссах, Брюсселе, Женеве, Базеле, Праге, Будапеште, Риме и в других городах. Имя Серпинского быстро приобретает огромную популярность. В математическую литературу прочно вошли термины: «Универсальная кривая Серпинского», «Треугольная кривая Серпинского», «S-континуум» и др.

В годы второй мировой войны Серпинский не прекращал научную работу и даже преподавал в подпольном университете Варшавы. Осенью 1944 г., когда немецкие войска начали сжигать Варшаву, Серпинский вынужден был покинуть город. Заботливые друзья вывезли его в Мехувский уезд.

В феврале 1945 г., после освобождения Польши советскими войсками Серпинский пешком отправился в Краков. Здесь его во второй раз приютил Ягеллонский университет. Он приступил к чтению лекций и стал печатать статьи и книги, написанные им во время оккупации. Здесь же он спустя несколько месяцев возобновил издание журнала «F. М.», которое было прервано войной.

С осени 1945 г. Серпинский в Варшаве. И снова большой труд по восстановлению университета. Снова лекции в различных университетах Европы, Индии, Канады, США, доклады и сообщения на симпозиумах и съездах. Деятельность Серпинского получает высокую оценку в Польской Народной Республике. В 1949 г. ему была присуждена Государственная премия первой степени, а в 1951 г. была выбита медаль с барельефом Серпинского по случаю 20-летия исполнения им обязанностей председателя Варшавского научного общества. В 1952—1957 гг. Серпинский был вице-президентом Польской Академии наук. В апреле 1957 г. он принял участие в юбилейной научной сессии Академии наук СССР, посвященной 250-летию со дня рождения Л. Эйлера. В том же году он возобновил издание «Acta Arithmetica» — единственного в мире журнала, посвященного только вопросам теории чисел.

В последние 20 лет теория чисел снова занимает видное место в научной и литературной работе Серпинского. Многочисленные результаты, полученные Серпинским и его учениками, наиболее выдающимся из которых является Андрей Шинцель, заметно обогащают сокровищницу теории чисел, в особенности так называемой элементарной теории чисел.

Сейчас, как и в прошлые годы, Серпинский печатает оригинальные статьи, издает серьезные и популярные книги. Список работ, опубликованных им, содержит уже более 700 названий. Среди них свыше 30 монографий, учебников и популярных книжек.

Университеты десяти городов: Амстердама, Бордо, Вроцлава, Лакхнау (Индия), Львова, Москвы, Парижа, Праги, Софии и Тарту присвоили Серпинскому степень доктора honoris kausa. Серпинский — вице-председатель Международной Академии философии наук, почетный член Болгарской, Итальянской, Лиманской, Парижской, Румынской, Нью-Йоркской, Чехословацкой и других академий науки. Он также почетный член Лондонского математического общества и многих других научных обществ.

Вацлав Серпинский — старейший академик Польши. Он воспитал три поколения учеников, среди которых немало крупных математиков. Его непрерывная творческая деятельность на протяжении шестидесяти лет создала славу польской науке. Серпинский по праву считается отцом польской школы математиков.

	\begin{flushright}
	\textit{И. Мельников}
\end{flushright}
\newpage



 \hangindent=7.5cm \hangafter=0  {\small Элементарную теорию чисел следует считать одним из наилучших предметов для первоначального математического образования. Она требует очень мало предварительных знаний, а предмет ее понятен и близок; методы рассуждений, применяемые ею, просты, общи и немногочисленны; среди математических наук нет равной ей в обращении к естественной человеческой любознательности{\footnote{ Это высказывание Г. X. Харди (Bull. Amer. Math. Soc, 35, 1929, стр. 818) Серпинский поместил в качестве эпиграфа к своей книге «200 задач по элементарной теотеории чисел», изданной на польском языке.}}.}


	\section[Предисловие переводчика]{\center ПРЕДИСЛОВИЕ ПЕРЕВОДЧИКА}

Теория чисел зародилась давно, еще в древней Греции, но развивалась крайне медленно. Большой и устойчивый интерес к ее проблемам в значительной мере был обусловлен деятельностью П. Ферма (1601—1665). Как самостоятельная наука теория чисел получила свое первоначальное оформление лишь в XVIII в. в многочисленных работах Л. Эйлера (1707—1783). Следующей важнейшей вехой в ее истории были исследования К. Ф. Гаусса (1777—1855) и его последователей. Большое значение для развития теории чисел имели исследования П. Л. Чебышева (1821—1894) и целой плеяды русских и советских арифметиков, принадлежащих к Петербургской математической школе или продолжающих ее славные традиции. Общеизвестно мировое значение вклада в теорию чисел И. М. Виноградова, Ю. В. Линника, Л. Г. Шнирельмана и других советских математиков. Большие заслуги в развитии теории чисел имеют и современные зарубежные математики.


В настоящее время теория чисел — обширная и трудная область математики. Она развивается в различных направлениях и использует разнообразные методы и средства.

Представляется вполне естественным, чтобы факты, принадлежащие арифметике, обосновывались «элементарными» методами, т. е. при помощи лишь арифметических и элементарноалгебраических средств. В одних случаях это требование выполняется сравнительно легко. В других же случаях поиски элементарных доказательств носят затяжной характер, и не всегда им сопутствует успех. В свое время большое удивление в математическом мире вызвали элементарные решения глубоких проблем теории чисел, найденные Артином, Ван дер Варденом, Б. А. Венковым, Ю. В. Линником, Сельбергом и др. Предложенные ими решения очень трудны. Чтобы полностью понять и усвоить их, даже хорошо подготовленному читателю порой требуется много времени напряженного труда.

Задачи, рассматриваемые в данной книге, принадлежат элементарной теории чисел и, как правило, являются элементарными и в обычном смысле этого слова. Поэтому значительная часть книги доступна широкому кругу читателей. В книге изредка встречаются трудные задачи, из которых некоторые еще недавно рассматривались такими видными исследователями, как Серпинский, Эрдёш, Шинцель и др. Номера таких задач отмечены звездочкой.

Оригинал этой книги, появившийся в Польше в 1964 г., содержал 200 с лишним задач. В настоящее издание, кроме этих задач, вошло еще около сорока новых задач, присланных мне автором.

Эта книга не является задачником по теории чисел. Она не содержит тренировочных примеров и задач, необходимых для усвоения каких-то разделов учебной программы. Однако задачи и краткие решения, помещенные здесь, учат очень многому, так как, формируя математическое мышление, они создают известные предпосылки для самостоятельной работы в элементарной теории чисел и способствуют приобретению таких навыков, которые будут полезны в любой отрасли математики.


В настоящем издании сохранены ссылки автора на монографическую и журнальную литературу на польском и других иностранных языках. Знания, необходимые для успешной работы над отдельными задачами, читатель может почерпнуть из следующих книг по теории чисел: 1) И. М. Виноградов, Основы теории чисел, 7-е изд. М., 1965; 2) А. А. Бухштаб, Теория чисел, 2-е изд. М., 1966; 3) Ш. X. Михелович, Теория чисел, 2-е изд. М., 1967. Читателю также будут полезны две книги В. Серпинского в русском переводе: 1) О решении уравнений в целых числах. М., 1961; 2) Что мы знаем и чего не знаем о простых числах. М., 1963.

По моей инициативе здесь помещены в качестве приложения два извлечения в русском переводе из книги В. Серпинского «Элементарная теория чисел», изданной в Варшаве в 1964 г. на английском языке. В первом извлечении дается изложение весьма элементарного доказательства постулата Бертрана, принадлежащего П. Эрдёшу, а во втором — доказательство теоремы Шерка, принадлежащее Серпинскому.

В книге имеется несколько примечаний, написанных мною. Они отмечены номерами в квадратных скобках.

Я выражаю свое уважение и признательность редактору книги Юрию Алексеевичу Гастеву, ценные указания которого были учтены мною на последнем этапе работы над рукописью этой книги.

	\begin{flushright}
	\textit{И. Мельников}
\end{flushright}
\newpage

	\section[Задачи]{\center ЗАДАЧИ}
	\subsection[I. Делимость чисел]{\center I. Делимость чисел}

\begin{enumerate} 
\item Найти все натуральные числа $n$, для которых число $n^2+1$ делится на $n+1$.
\item Найти все целые числа $x \neq 3$, такие, что $x - 3 \mid x^3 - 3${\footnote{Символ $a \mid b$ читается так: «$a$ делит $b$» и означает, что число $b$ делится на числo $a$ без остатка. — \textit{Прим. перев}.}}.
\item Доказать, что если $7 \mid a^2 + b^2$, где $a$ и $b$ — целые числа, то $7 \mid a$ и $7 \mid b$.
\item Доказать, что существует бесконечно много натуральных чисел $n$, для которых число $4n^2+14$ делится одновременно на $5$ и на $13$.
\item Доказать, что для натуральных $n$ имеем $169 \mid 3^{3n+3} - 26n - 27$.
\item Доказать, что $19 \mid 2^{2^{6k+2}} + 3$ для $k=1, 2, ...$.
\item Доказать утверждение М. Крайчика о том, что $13 \mid 2^{70}+3^{70}$.
\item Доказать, что $F_n \mid 2^{F_n} - 2$, где $F_n=2^{2^n}+1$, $n=1, 2, ...$.
\item Доказать, что существует бесконечное число натуральных чисел $n$, для которых $n \mid 2^n+1$.
\item Доказать, что если $k$ — нечетное число, а $n$ — натуральное, то $2^{n+2} \mid k^{2^{n}}-1$.
\item Доказать, что $11\cdot31\cdot61 \mid 20^{15}-1$.
\item Доказать, что для натуральных $m$ и $a>1$ имеем{\footnote{Символ $(a, b)$ означает наибольший общий делитель чисел $a$ и $b$. — \textit{Прим. перев.}}}:
$$
\left( \frac{a^m - 1}{a-1}, a-1 \right) = (a-1, m).
$$
\item Доказать, что для каждого натурального числа $n$ число $3\cdot(1^5+2^5+...+n^5)$ делится на число $1^3+2^3+...+n^3$.
\item Найти все натуральные числа $n>1$, для которых число $1^n+2^n+...+(n-1)^n$ делится на $n$.
\item Исследовать, для каких натуральных $n$ которое из двух чисел $a_n=2^{2n+1}-2^{n+1}+1$ и $b_n=2^{2n+1}+2^{n+1}+1$  делится и которое не делится на $5$.
\item Доказать, что для каждого натурального числа $n$ существует такое натуральное число $x$, что каждый из членов бесконечной последовательности 
$$
x+1,\     x^{x}+1,\     x^{x^{x}}+1,\      x^{x^{x^{x}}}+1,\    ...
$$
делится на $n$.
\item Доказать, что существует бесконечно много нечетных чисел $n$, для которых ни при каком четном $x$ ни одно из чисел бесконечной последовательности 
$$
x+1,\     x^{x}+1,\     x^{x^{x}}+1,\      x^{x^{x^{x}}}+1,\    ...
$$
не делится на $n$.
\item Доказать, что для всех натуральных $n$ имеем $n^2 \mid (n+1)^n -1 $.
\item Доказать, что для всех натуральных $n$ имеем $(2^n-1)^2 \mid 2^{(2^n-1)n}-1$.
\item Доказать, что существует бесконечно много натуральных чисел $n$, таких, что $n \mid 2^n+1$, и найти все такие простые числа $n$.

\end{enumerate}

\begin{withdot}
	\addtocounter{namedlistcounter}{20}
	\item Найти все нечетные числа $n$, такие, что $n \mid 3^n+1$.
	\item Доказать, что для каждого натурального числа $a>1$ существует бесконечно много натуральных чисел $n$, таких, что $n \mid a^n+1$.
	\item Доказать, что существует бесконечно много натуральных чисел $n$, для которых $n \mid 2^n+2$.
\end{withdot}

\begin{enumerate}
	\setcounter{enumi}{23}
\item Найти все натуральные числа $a$, для которых число $a^{10}+1$ делится на $10$.
\end{enumerate}

\begin{withdot}
	\addtocounter{namedlistcounter}{24}
	\item Доказать, что не существует натурального числа $n>1$, для которого $n \mid 2^n-1$.
\end{withdot}

\begin{enumerate}
	\setcounter{enumi}{25}
	\item Найти все натуральные числа $n$, для которых $3 \mid n\cdot2^n+1$.
	\item Доказать, что для каждого простого нечетного числа $p$ существует бесконечно много натуральных чисел $n$, для которых $p \mid n\cdot2^n+1$.
	\item Доказать, что для каждого натурального числа $n$ существуют натуральные числа $x>n$ и $y$, такие, что $x^x \mid y^y$, но $x \nmid y${\footnote{Читается: «$x$ не делит $y$». — \textit{Прим. перев.}}}.
	\item Доказать, что существует бесконечное число натуральных чисел $n$, для которых число $n^2-3$ делится на точный квадрат, больший единицы, и найти наименьшее из таких натуральных чисел $n$.
\end{enumerate}

\begin{withdot}
	\addtocounter{namedlistcounter}{29}
	\item Доказать, что для нечетных $n$ имеем $n \mid 2^{n!}-1$.
\end{withdot}

\begin{enumerate}
		\setcounter{enumi}{30}
		\item  Доказать, что в бесконечной последовательности
		$$
		2^n-3 \ \ \ \ \ \ \	(n = 2,3,4, . . .)
		$$
		существует бесконечно много членов, делящихся на $5$, и бесконечно много делящихся на $13$, но ни один член этой последовательности не делится на $5 \cdot 13$.
\end{enumerate}

\begin{withdot}
	\addtocounter{namedlistcounter}{31}
	\item Найти два наименьших составных числа $n$, таких, что $n \mid 2^n-2$ и $n \mid 3^n-3$.
	\item Найти наименьшее натуральное число $n$, такое, что $n \mid 2^n-2$, но $n \nmid 3^n-3$.
\end{withdot}

\begin{enumerate}
	\setcounter{enumi}{33}
	\item  Найти наименьшее натуральное число $n$, такое, что $n \nmid 2^n-2$, но $n \mid 3^n-3$.
	\item Для каждого натурального числа $a$ найти составное число $n$, такое, что $n \mid a^n - a$.
	\item Доказать, что если для целых чисел $a$, $b$ и $c$ имеем $9 \mid a^3 + b^3 + c^3$, то по крайней мере одно из чисел $a$, $b$, $c$ делится на 3.
	\item Доказать, что если для целых чисел $a_k \ (k =1, 2, 3, 4, 5)$ имеем:
	$$
	9 \mid a_1^3 + a_2^3 + a_3^3 + a_4^3 + a_5^3 ,
	$$
	то $3 \mid a_1a_2a_3a_4a_5$.
	\item Доказать, что если $x$, $y$ и $z$ — натуральные числа, $(x,y) = 1$ и $x^2+y^2=z^4$, то $7 \mid xy$, и что условие $(x,y)=1$ является здесь необходимым.
	
\end{enumerate}

\begin{withdot}
	\addtocounter{namedlistcounter}{38}
	\item Доказать, что существует бесконечно много пар натуральных	чисел $x$, $y$, таких, что 
	$$
	x(x+1) \mid y(y+1),\ x \nmid y,\ x+1 \nmid y,\ x \nmid y+1,\ x+1 \nmid y+1,
	$$
	и найти пару наименьших таких чисел $x$, $y$.
	
\end{withdot}

\begin{enumerate}
	\setcounter{enumi}{39}
	\item Для каждого натурального числа $s \leqslant 25$, а также для числа $s = 100$ найти наименьшее натуральное число $n_s$, имеющее сумму цифр, равную $s$ (в десятичной системе счисления), и делящееся на $s$.
\end{enumerate}

\begin{withdot}
	\addtocounter{namedlistcounter}{40}
	\item Доказать, что для каждого натурального числа $s$ существует натуральное число $n$ с суммой цифр $s$ (в десятичной системе счисления), делящееся на $s$.
	\item Доказать:

	а)	что каждое натуральное число имеет натуральных делителей вида $4k+1$ не меньше, чем вида $4k+3$;

	б)	что существует бесконечно много натуральных чисел, имеющих натуральных делителей вида $4k+1$ столько же, сколько и вида $4k+3$;

	в)	что существует бесконечно много натуральных чисел, имеющих натуральных делителей вида $4k+1$ более, чем вида $4k+3$.
	
\end{withdot}

\begin{enumerate}
	\setcounter{enumi}{42}
	\item Доказать, что если $a$, $b$, и $c$ — произвольные целые числа, a $n$ — натуральное число $>3$, то существует целое число $k$, такое, что ни одно из чисел $k+a$, $k + b$ и $k + c$ не делится на $n$.
	
\end{enumerate}

\subsection[II. Взаимно простые числа]{\center II. Взаимно простые числа}

\begin{enumerate}
	\setcounter{enumi}{43}
	\item Доказать:

	а)	что $(n, 2^{2^n}+1)=1$, для $n=1, 2, ...$;

	б)	что существует бесконечно много натуральных чисел $n$, для которых $(n, 2^n-1)>1$, и найти наименьшее из них.
	
	\item Доказать, что при всяком целом $k$ числа $2k+1$ и $9k+4$ являются взаимно простыми, а для чисел $2k-1$ и $9k+4$ найти их наибольший общий делитель в зависимости от целого числа $k$.
	
	\item Доказать: а) что существует бесконечная возрастающая последовательность попарно взаимно простых треугольных чисел (т. е. чисел $t_n=\frac{n(n+1)}{2}$, где $n=1, 2, ...$);

	б) что существует бесконечная возрастающая последовательность попарно взаимно простых тетраэдральных чисел (т. е. чисел вида $T_n=\frac{1}{6}n(n+1)(n+2)$, где $n=1, 2, ...$).
	
	\item Доказать, что если $a$ и $b$ — различные целые числа, то существует бесконечно много таких натуральных чисел $n$, что числа $a + n$ и $b + n$ являются натуральными взаимно простыми.
	
\end{enumerate}


\begin{withdot}
	\addtocounter{namedlistcounter}{47}
	\item Доказать, что если $a$, $b$ и $c$ — различные целые числа, то существует бесконечно много натуральных чисел $n$, таких, что числа $a+n$, $b+n$ и $c+n$ являются попарно взаимно простыми.
	
\end{withdot}

\begin{enumerate}
	\setcounter{enumi}{48}
	\item Дать пример таких четырех различных натуральных чисел $a$, $b$, $c$ и $d$, для которых не существует ни одного натурального числа $n$, такого, чтобы числа $a+n$, $b+n$, $c+n$ и $d+n$ были бы попарно взаимно простыми.
	\item Доказать, что каждое натуральное число $>6$ является суммой двух взаимно простых натуральных чисел $>1$.
\end{enumerate}

\begin{withdot}
	\addtocounter{namedlistcounter}{50}
	\item Доказать, что каждое натуральное число $>17$ является суммой трех натуральных попарно взаимно простых чисел $>1$ и что число $17$ этим свойством не обладает.
	\item Доказать, что каждое четное число $2k$ для каждого натурального числа $m$ является разностью двух натуральных чисел, взаимно простых с $m$.
	\item Доказать, что из последовательности Фибоначчи (определяемой условиями $u_1=u_2=1$, $u_{n+2}=u_n+u_{n+1}$ для $n=1, 2, ...$) можно извлечь бесконечную возрастающую последовательность с попарно взаимно простыми членами.
	
\end{withdot}

\subsection[III. Арифметические прогрессии]{\center III. Арифметические прогрессии}

\begin{enumerate}
	\setcounter{enumi}{53}
	\item Доказать, что существуют арифметические прогрессии произвольной длины, составленные из различных попарно взаимно простых	натуральных чисел.
	\item Доказать, что для каждого натурального числа $k$ множество всех натуральных чисел $n$, у которых число натуральных делителей кратно $k$, содержит бесконечную арифметическую прогрессию.
	\item Доказать, что существует бесконечно много систем натуральных чисел $x$, $y$ и $z$, для которых числа $x(x+1)$, $y(y+1)$ и $z(z+1)$ составляют возрастающую арифметическую прогрессию.
	\item Найти все прямоугольные треугольники, стороны которых выражаются натуральными числами, образующими арифметическую прогрессию.
	\item Найти бесконечную возрастающую арифметическую прогрессию,	состоящую из натуральных чисел, имеющую наименьшую разность и не содержащую ни одного треугольного числа.
	\item Найти необходимое и достаточное условие того, чтобы арифметическая прогрессия $ak+b \ \ (k=0,1,2, ...)$, где $a$ и $b$ — натуральные числа, содержала бесконечно много членов, являющихся квадратами натуральных чисел.
\end{enumerate}

\begin{withdot}
	\addtocounter{namedlistcounter}{59}
	\item Доказать, что существуют арифметические прогрессии произвольной длины, составленные из различных чисел, являющихся степенями натуральных чисел с натуральными показателями $>1$.
\end{withdot}

\begin{enumerate}
	\setcounter{enumi}{60}
	\item Доказать, что не существует бесконечной арифметической прогрессии, составленной из различных натуральных чисел, каждый член которой является степенью натурального числа с натуральным показателем $>1$.
	\item Доказать, что не существует четырех последовательных натуральных чисел, каждое из которых было бы степенью натурального числа с натуральным показателем $>1$.
	\item Доказать элементарно, что в каждой возрастающей арифметической прогрессии, членами которой являются натуральные числа, существует отрезок произвольной длины, состоящий только из составных чисел.
\end{enumerate}

\begin{withdot}
	\addtocounter{namedlistcounter}{63}
	\item Доказать элементарно, что если $a$ и $b$ — натуральные взаимно простые числа $m$, то для каждого натурального числа т в арифметической прогрессии $ak+b \ \ (k=0,1,2,...)$ существует бесконечно много членов, взаимно простых с $m$.
\end{withdot}

\begin{enumerate}
	\setcounter{enumi}{64}
	\item Доказать, что для каждого натурального числа $s$ в каждой возрастающей арифметической прогрессии, состоящей из натуральных чисел, существуют числа, у которых первые $s$ цифр могут быть произвольными (в десятичной системе счисления).
	\item Найти все возрастающие арифметические прогрессии, состоящие из трех членов последовательности Фибоначчи (см. задачу 53), и доказать, что не существует возрастающих арифметических последовательностей, состоящих из четырех членов последовательности Фибоначчи.
\end{enumerate}

\begin{withdot}
	\addtocounter{namedlistcounter}{66}
	\item Найти возрастающую арифметическую прогрессию с наименьшей разностью, состоящую из натуральных чисел и не содержащую ни одного числа последовательности Фибоначчи.
	\item Найти прогрессию $ak+b \ \ (k=0,1,2,...)$, где $a$ и $b$ — взаимно простые натуральные числа, не содержащую ни одного числа последовательности Фибоначчи.
\end{withdot}

\begin{enumerate}
	\setcounter{enumi}{68}
	\item Доказать, что в каждой арифметической прогрессии $ak+b \ \ (k=0,1,2,...)$, где $a$ и $b$ — взаимно простые натуральные числа, существует бесконечно много членов, попарно взаимно простых.
\end{enumerate}

\begin{withdot}
	\addtocounter{namedlistcounter}{69}
	\item Доказать, что в каждой арифметической прогрессии $ak+b \ \ (k=0,1,2,...)$, где $a$ и $b$ — натуральные числа, существует бесконечно много членов, имеющих одинаковые простые делители.
\end{withdot}

\begin{enumerate}
	\setcounter{enumi}{70}
	\item Из теоремы Дирихле, согласно которой в каждой арифметической прогрессии $ak+b \ \ (k=0,1,2,...)$, где $a$ и $b$ — натуральные взаимно простые числа, существует бесконечно много простых чисел [1], вывести следствие: в каждой такой прогрессии для каждого натурального числа $s$ существует бесконечно много членов, являющихся произведениями $s$ различных простых чисел.
	\item Найти все арифметические прогрессии с разностью $10$, состоящие из более чем двух простых чисел.
	\item Найти все арифметические прогрессии с разностью $100$, состоящие из более чем двух простых чисел.
\end{enumerate}

\begin{withdot}
	\addtocounter{namedlistcounter}{73}
	\item Найти десятичленную возрастающую арифметическую прогрессию, состоящую из простых чисел, последний член которой есть наименьшее возможное при этих условиях число.
\end{withdot}

\begin{enumerate}
	\setcounter{enumi}{74}
	\item Дать пример бесконечной возрастающей арифметической прогрессии, состоящей из натуральных чисел, ни один член которой не является ни суммой, ни разностью двух простых чисел.
\end{enumerate}

\subsection[IV. Простые и составные числа]{\center IV. Простые и составные числа}

\begin{enumerate}
	\setcounter{enumi}{75}
	\item Доказать, что для каждого четного числа $n>6$ существуют простые числа $p$ и $q$, меньшие $n-1$, такие, что $(n-p, n-q)=1$.
	\item Найти все простые числа, являющиеся одновременно суммами и разностями двух простых чисел.
	\item Найти три наименьших натуральных числа $n$, таких, что между $n$ и $n+10$ нет ни одного простого числа, а также три наименьших натуральных числа $m$, таких, что между $10m$ и $10(m+1)$ нет ни одного простого числа.
	\item Доказать, что каждое простое число вида $4k+1$ является длиной гипотенузы прямоугольного треугольника, стороны которого выражаются натуральными числами.
	\item Найти четыре решения уравнения $p^2+1 = q^2 + r^2$ в простых числах $p$, $q$, $r$.
	
\end{enumerate}












































\newpage

\section[Решения задач]{\center РЕШЕНИЯ ЗАДАЧ}
\subsection[I. Делимость чисел]{\center I. Делимость чисел}

\begin{enumerate}
	\item Существует только одно такое натуральное число $n=1$. Действительно, так как $n^2+1=n(n+1)-(n-1)$, то из предположения $n+1 \mid n^2+1$ следует, что $n+1 \mid n-1$, а последнее для натуральных $n$ возможно лишь тогда, когда $n-1=0$, т. е. когда $n=1$.
	\item Пусть $x-3=t$ есть целое число $\neq0$, такое, что $t \mid (t+3)^3-3$. Это условие равносильно утверждению, что $t \mid 3^3-3$ или $t \mid 24$. Таким образом, необходимо и достаточно, чтобы $t$ было целочисленным делителем числа $24$, т. е. одним из чисел $\pm1$, $\pm2$, $\pm3$, $\pm4$, $\pm6$, $\pm8$, $\pm12$, $\pm24$. Отсюда для $x=t+3$ получаем следующие значения: $-21$, $-9$, $-5$, $-3$, $-1$, $0$, $1$, $2$, $4$, $5$, $6$, $7$, $9$, $11$, $15$ и $27$.
	\item
	
\end{enumerate}


	\newpage
	\tableofcontents
	
	\thispagestyle{empty} % 
	
	\newpage
	
	\setcounter{secnumdepth}{0}  
	
	\phantomsection
	
		\section*{Описание}
	
	{\bf Название:} 250 задач по элементарной теории чисе
	
{\bf Автор:} Вацлав Франциск Серпинский

{\bf Переводчик (с польского):} И. Г. Мельников
	
{\bf Издательство:} Москва: Просвещение, 1968
	
		{\bf Редактор:} Ю. А. Гастев
	
		{\bf Художественный редактор:} В. С. Эрденко
	
		{\bf Технический редактор:} Н. Ф. Макарова
	
		{\bf Корректоры:} К. А. Иванова
	
		{\bf Аннотация:} Сборник задач по элементарной теории чисел (от совсем простых до довольно трудных), с решениями и комментариями. Может быть использована в работе школьных и студенческих математических кружков.
		\thispagestyle{empty} % выключаем отображение номера для этой страницы

	
\end{document}


