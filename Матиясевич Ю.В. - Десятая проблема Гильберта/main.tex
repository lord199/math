
%\documentclass[oneside,final,14pt]{extreport}
\documentclass[12pt, a4paper, openany]{book}



\usepackage[left=1cm,right=1cm,top=2cm,bottom=2cm,bindingoffset=0cm]{geometry}
%\usepackage[koi8-r]{inputenc}
%\usepackage[russianb]{babel}
\usepackage{vmargin}

\setpapersize{A4}
\usepackage[T2A]{fontenc}
\usepackage[utf8x]{inputenc}
\usepackage[english, russian]{babel}
\setmarginsrb{2cm}{2cm}{2cm}{2cm}{0pt}{5mm}{0pt}{0mm}
\usepackage{indentfirst}
\usepackage{nicefrac} % For comparison
%\usepackage{xfrac}    % Works better with other fonts
%\usepackage[unicode, pdftex]{hyperref}
\usepackage{lettrine}
\usepackage[usenames]{color}
\usepackage{colortbl}
\usepackage{mathtext}
\usepackage{epigraph}
\usepackage{amsmath, amsfonts, amssymb, mathrsfs}
%\usepackage{mathptmx}
%\usepackage{txfonts}
\usepackage{pxfonts}
\usepackage[pagestyles]{titlesec}
\usepackage{ebgaramond}
\usepackage{awesomebox}
\usepackage{enumitem}
\usepackage{makeidx}
\makeindex
    \usepackage{etoolbox}
\makeatletter
\newlength\epitextskip
\pretocmd{\@epitext}{\em}{}{}
\apptocmd{\@epitext}{\em}{}{}
\patchcmd{\epigraph}{\@epitext{#1}\\}{\@epitext{#1}\\[\epitextskip]}{}{}
\makeatother

\DeclareSymbolFont{Xlargesymbols}{OMX}{cmex}{m}{n}
\DeclareMathSymbol{\Xsum}{\mathop}{Xlargesymbols}{80}

\setlength\epigraphrule{0pt}
\setlength\epitextskip{2ex}
\setlength\epigraphwidth{.8\textwidth}

\usepackage{xfrac}    % Works better with other fonts
\usepackage[colorlinks=true,linkcolor=black,urlcolor=black,bookmarksopen=true]{hyperref}

\usepackage{fancyhdr} % пакет для установки колонтитулов
\pagestyle{fancy} % смена стиля оформления страниц
\fancyhf{} % очистка текущих значений
\fancyhead[C]{\thepage} % установка верхнего колонтитула
\renewcommand{\headrulewidth}{0pt} % убрать разделительную линию


% Настройка вертикальных и горизонтальных отступов
\titlespacing{\chapter}{0pt}{5pt}{5pt}
\titlespacing{\section}{\parindent}{4mm}{4mm}
\titlespacing{\subsection}{\parindent}{3mm}{3mm}


%
%\renewcommand{\tabcolsep}{1cm}   %% increase table column spacing

% Настройка задачи со зведочкой
\newcounter{namedlistcounter}  % number the items
\newenvironment{withdot}
{\begin{list}
		{\arabic{namedlistcounter}*.} % labeling 
		{\usecounter{namedlistcounter}   % set counter
			\setlength{\leftmargin}{3em}} % set spacing 
	}
	{\end{list}}


\newcommand{\anonsection}[1]{ \section*{#1} \addcontentsline{toc}{section}{\numberline {}#1}} 

\makeatletter %%%%% <---- Starting chapter without a pagebreak
\renewcommand\chapter{\par%
	\thispagestyle{plain}% \global\@topnum\z@
	\@afterindentfalse \secdef\@chapter\@schapter}
\makeatother %%%%% <---- Starting chapter without a pagebreak
\titleformat{\chapter}[display]
{\normalfont\bfseries}{}{0pt}{\Large}

\newpagestyle{mystyle}{
	\sethead[\thepage][][]{}{}{\thepage}	
}

\renewcommand{\rmdefault}{cmr}


\pagestyle{mystyle}
\sloppy

\begin{document}

	
	\begin{titlepage}
		
		\begin{center}
			%\vfill
			
			%\vfill
			\topskip 0pt
		%	\vspace*{\fill}
			\begin{flushleft}
		{\large	МАТЕМАТИЧЕСКАЯ\\ЛОГИКА\\И ОСНОВАНИЯ\\МАТЕМАТИКИ}			

			\ \\
\ \\
			\ \\
\ \\
			{\large Ю.В.МАТИЯСЕВИЧ\\}
			\ \\
			\ \\
			{\Huge\bf ДЕСЯТАЯ ПРОБЛЕМА ГИЛЬБЕРТА}


			\vspace*{\fill}    
			
			\vfill
			
Москва

Издательская фирма

«Физико-математическая литература»

ВО «Наука»

1\ 9\ 9\ 3

			\end{flushleft}
		\end{center}
		
	\end{titlepage}
	
	\thispagestyle{empty} % выключаем отображение номера для этой страницы
	
	\newpage
	
\setcounter{secnumdepth}{0}  
	
	\section[ПРЕДИСЛОВИЕ]{\center ПРЕДИСЛОВИЕ}

На Втором Международном конгрессе математиков в Париже Давид Гильберт [1900] сделал свой знаменитый доклад «Математические проблемы», содержащий 23 проблемы или, точнее, 23 группы родственных проблем, которые 19-й век оставлял в наследие 20-му. Проблема под номером десять была посвящена диофантовым уравнениям.
	\begin{center}
		{\small 10. ЗАДАЧА О РАЗРЕШИМОСТИ ДИОФАНТОВА УРАВНЕНИЯ.}
\end{center}


\hangindent=1.5cm \hangafter=0 {\footnotesize Пусть задано диофантово уравнение с произвольными неизвестными и целыми рациональными числовыми коэффициентами. \textit{Указать способ, при помощи которого возможно после конечного числа операций установить, разрешимо ли это уравнение в целых рациональных числах.}}

\

Под «способом», который предлагает найти Д. Гильберт, в настоящее время подразумевают «алгоритм». В начале века, когда проблемы формулировались, ещё не было математически строго общего понятия алгоритма. Отсутствие такого понятия не могло само по себе служить препятствием к положительному решению 10-й проблемы Гильберта, поскольку про конкретные алгоритмы всегда было ясно, что они действительно дают требуемый общий способ решения соответствующих проблем.

В 30-е годы в работах К. Гёделя, А. Чёрча, А. М. Тьюринга и других логиков было выработано строгое общее понятие алгоритма, которое дало принципиальную возможность устанавливать алгоритмическую неразрешимость, т. е. доказывать невозможность алгоритма с требуемыми свойствами. Тогда же были найдены первые примеры алгоритмически неразрешимых проблем, сначала в самой математической логике, а затем и в других разделах математики.

Таким образом, теория алгоритмов создала необходимые предпосылки для попыток доказать неразрешимость 10-й проблемы Гильберта. Первые работы в этом направлении были опубликованы в начале 50-х годов, а в 1970 году исследования завершились «отрицательным решением» 10-й проблемы Гильберта.

В случае 10-й проблемы Гильберта, как и в случае других проблем, долго ожидавших своего решения, не меньшее, а пожалуй, большее значение имеет математический аппарат, развитый для решения проблемы и находящий затем другие приложения, порой неожиданные. Основной технический результат, полученный при доказательстве неразрешимости 10-й проблемы  Гильберта — это теорема о совпадении класса \textit{диофантовых множеств} и класса \textit{перечислимых множеств}. В качестве одного из следствий этой теоремы, формулировка которого не содержит специальных терминов, приведем следующее: \textit{можно явно указать  полином от многих переменных с целыми коэффициентами такой,  что множество всех его положительных значений, принимаемых  при целочисленных значениях переменных, есть в точности  множество всех простых чисел}. 

Настоящая книга посвящена алгоритмической неразрешимости 10-й проблемы Гильберта и родственным вопросам; многочисленные частичные результаты, полученные в направлении положительного решения 10-й проблемы Гильберта, здесь почти  не рассматриваются. 

Отрицательное решение 10-й проблемы Гильберта излагали (с разной степенью детализации) многие авторы, в частности: Азра [1971], Белл и Маховер [1977]. Гермес [1972, 1978], Девис [1973а, 1974]. Захаров [1970, 1986], Капланский [1977], Манин [1973, 1977], Маргенштерн [1981]. Матиясевич [1972а], Мияйлович, Маркович и Дошен [1986], Руохонен [1972, 1980], Саломаа [1985], Смориньский [1987], Сусман [1971], Такахаши [1974], Фенстад [1971], Хавел [1973], Хиросе [1973]. 

Одной из отличительных особенностей настоящей книги является то. что она. помимо собственно отрицательного решения 10-й проблемы Гильберта, содержит ряд приложений разработанной для этого решения техники; приложения эти в настоящее  время разбросаны, в основном, по журнальным публикациям. За два десятилетия, прошедшие со времени решения проблемы, были получены многообразные упрощения и модификации первоначального доказательства. Настоящая книга также содержит ряд новых, ранее не публиковавшихся доказательств. 

Естественно, что для понимания отрицательного решения  10-й проблемы Гильберта требуются знания как по теории чисел, так и по математической логике. Стремясь сделать книгу доступной для более широкой аудитории, в особенности, для начинающих математиков, автор старался ограничиться минимальными требованиями к математической подготовке читателя. В частности, у него не предполагается специальных знаний по теории алгоритмов, все необходимые понятия вводятся в книге, которая, тем самым, может служить для первоначального знакомства с этим увлекательным предметом (конечно, эта книга не может служить для систематического изучения даже основ теории алгоритмов). Немногочисленные требуемые сведения по  теории чисел, выходящие за рамки общематематической подготовки, приведены в \textit{приложениях} в конце книги. 

Книга снабжена многочисленными \textit{упражнениями} различной сложности — от совершенно элементарных до составляющих предмет небольшого исследования. Цель упражнений — ознакомить читателя с многообразными результатами, не приводя их доказательств. Вопрос о том, что следует отнести к основному содержанию книги, излагаемому с полным доказательством, а что — к упражнениям, естественно, решался субъективно. В упражнения попадали, в частности, результаты, которые требовали специальных знаний или имели громоздкие доказательства, результаты, далекие, по-видимому, от окончательных или представляющие ограниченный интерес. Упражнения снабжены указаниями к решению и/или отсылкой к литературному источнику. 

Помимо упражнений в книгу включены немногочисленные открытые вопросы и нерешённые проблемы. Деление опять-таки субъективное. Открытый вопрос, возможно, не закрыт до сих пор лишь из-за того, что никто серьёзно над ним не задумывался, и ответ на открытый вопрос, быть может, окажется малополезным. С другой стороны, нерешённые проблемы, приведённые в книге, привлекали многих серьёзных исследователей, и, возможно, решение этих проблем потребует десятилетий. 

Каждая глава завершается комментариями, в которых излагается история получения соответствующих результатов. Это представляется необходимым, поскольку логический порядок изложения материала, использованный в книге, часто не соответствует хронологическому порядку получения соответствующих результатов. 

\textit{Список литературы} содержит все основные публикации, нацеленные на получение отрицательного решения 10-й проблемы Гильберта, и большинство публикаций, использующих разработанную для этого технику. Автор будет признателен за указания на относящиеся к этой тематике работы, не вошедшие в этот список. 

Нумерация формул в каждом параграфе своя. При ссылке на выделенную формулу из другого параграфа к номеру формулы добавляется номер параграфа, например, формула (5.3) — это формула (3) из пятого параграфа той же главы. Аналогично, формула (2.4.6) — это формула (6) из § 2.4, т. е. четвёртого параграфа главы 2. 

Книгу не обязательно читать последовательно. Её можно условно разбить на две части. Первая часть, в которой даётся решение 10-й проблемы Гильберта, состоит из глав 1-5. Приведённая ниже диаграмма показывает зависимость друг от друга параграфов первой части: чтение каждого параграфа предполагает знакомство с теми параграфами, которые на диаграмме расположены не ниже и не правее него. 
\vspace{1em}

			\begin{tabular}{ p{4em} p{4em} p{4em} p{4em} p{4em} p{4em} p{4em}  }
 &  &  &  &  & § 5.1 &  \\
 &  &  &  &  & § 5.2 &  \\
§ 1.1 &  &  &  &  &  &  \\
§ 1.2 &  &  &  &  &  &  \\
§ 1.3 &  &  &  &  &  &  \\
§ 1.4 &  &  &  &  &  &  \\
§ 1.5 &  &  &  & § 3.1  & § 5.3 & § 5.4 \\
 & § 1.6  & § 2.1 &  &  &  &  \\
  &  & § 2.2 &  &  &  &  \\
  &  & § 2.3 &  &  &  &  \\
  &  & § 2.4 &  &  &  &  \\  
  &  & § 2.5 &  &  &  &  \\
§ 3.1 & § 3.2 & § 3.3 & § 3.4 &  &  &  \\
  &  &  &  § 3.5 & § 3.6 & § 5.5 & \\	
§ 4.1 &  &  &  &  &  &  \\
§ 4.2 &  & § 4.3 & § 4.4 &  &  &  \\
 &  &  & § 4.5 &  &  &  \\
 &  &  & § 4.6 &  &  & § 5.6 \\
 &  &  &  &  &  & § 5.7 \\
\end{tabular}
\vspace{1em}

Аналогично, следующая диаграмма показывает зависимость параграфов второй части, посвященной приложениям, от параграфов первой части. 

\vspace{1em}
			\begin{tabular}{ p{3.5em} p{3.5em} p{3.5em} p{3.5em} p{3.5em} p{3.5em} p{3.5em}  p{3.5em}}
§ 2.1 & § 2.4 & § 3.4 & § 5.4 & § 5.5 & § 5.7 & § 7.1 & § 7.2\\
§ 7.3 & § 6.4 & § 6.2 & & § 6.1 & § 9.1 & § 9.2 & § 9.3\\
§ 7.4 & & § 6.3 &  &  & § 10.1 & § 10.2 & \\

\end{tabular}
\vspace{1em}

Между собой параграфы второй части связаны слабо. В § 6.1-6.3 приведены три разных способа для достижения одной и той же цели; достаточно знать любой из них для чтения § 6.4-6.6 и § 9.1. Аналогично, в § 4.5 и § 6.5 приведены две разные конструкции универсальных уравнений, и знакомства с любой из них достаточно для чтения § 6.6 и § 8.1. Чтение § 8.2 предполагает знакомство с § 7.2, которое в свою очередь предполагает знание § 6.2-6.3; в § 9.4 используются результаты § 9.2, а в § 10.1 — результаты § 6.6. 

\newpage
\section[ГЛАВА 1. ОСНОВНЫЕ ПОНЯТИЯ]{\center Г\ л\ а\ в\ а \ \ 1 \\ \textbf{ОСНОВНЫЕ ПОНЯТИЯ}}

{\small В этой главе будет введено основное понятие, изучаемое в данной книге, — понятие диофантова множества, и будут установлены его простейшие свойства. }

\subsection[§ 1.1. Разрешимость диофантовых уравнений как массовая проблема]{\center § 1.1. Разрешимость диофантовых уравнений как массовая проблема}

Напомним, что диофантовыми уравнениями называют уравнения вида

\begin{equation}
D(x_1, \dots, x_m)=0,
\end{equation}

\noindent где $D$ — полином с целыми коэффициентами. Наряду с (1)) диофантово уравнение может быть записано в более общем виде 

\begin{equation}
D_L(x_1, \dots, x_m)=D_R(x_1, \dots, x_m),
\end{equation}

\noindentгде $D_L$ и $D_R$ также являются полиномами с целыми коэффициентами. Говоря о «произвольном диофантовом уравнении», мы будем иметь в виду уравнение типа (1), поскольку уравнение типа (2) легко приводится к виду (1) перенесением всех членов в левую часть. С другой стороны, выписывая конкретные уравнения, мы часто будем использовать запись вида (2), если она легче для восприятия. Другое преимущество более общей записи вида (2), которым мы будем пользоваться, состоит в том, что, записывая уравнение в виде (2), мы можем потребовать, чтобы $D_L$ и $D_R$ были полиномами с неотрицательными коэффициентами. 

Поскольку диофантовы уравнения, как правило, имеют много неизвестных, следует различать \textit{степень уравнения} (1) \textit{относительно данной неизвестной} $x_i$ и \textit{(полную) степень уравнения} (1), под которой мы будем подразумевать максимальную  суммарную (по всем неизвестным) степень одночленов, составляющих полином $D$. 

Существенной характеристикой диофантовых уравнений является не только их вид (1), но и множество допустимых значений неизвестных. Гильберт в 10-й проблеме говорит о решениях в \textit{целых рациональных числах}. В этой книге мы будем говорить просто о \textit{целых числах}, поскольку \textit{целые алгебраические числа} в ней почти не будут рассматриваться. (Вопросы о разрешимости диофантовых уравнений в других типах неизвестных рассматриваются в § 1.3, 7.3, 7.4.) 

Десятая проблема Гильберта является примером \textit{массовой проблемы}. Массовая проблема — это проблема, состоящая из счётного множества \textit{индивидуальных проблем}, на каждую из которых надо дать конкретный ответ «ДА» или «НЕТ». Эти индивидуальные проблемы мы будем называть \textit{подпроблемами} соответствующей массовой проблемы. Каждая индивидуальная проблема специфицируется конечным объёмом информации (в случае 10-й проблемы Гильберта такой информацией является полином $D$ из (1)). Суть массовой проблемы состоит в том, что требуется найти единый метод, пригодный для получения ответа на любую из её индивидуальных подпроблем. Со времени Диофанта специалисты по теории чисел нашли решения огромного количества диофантовых уравнений и установили отсутствие решений у массы других уравнений, однако при этом для разных классов уравнений или даже отдельных уравнений приходилось изобретать свой особый метод. Д. Гильберт в 10-й проблеме предлагал найти  \textit{универсальный метод} для распознавания разрешимости диофантовых уравнений. 

Решение массовой проблемы может быть либо прямым — посредством указания процедуры нахождения ответа для каждой индивидуальной подпроблемы, либо косвенным — путем свед\textbf{е}ния данной массовой проблемы к другой массовой проблеме, решение которой уж\textbf{е} известно. Мы не будем давать формального определения сведения, поскольку общая теория сводимости нам не потребуется, а в конкретных случаях сведения одной массовой проблемы к другой из контекста будет ясно, что имеется в виду. 

Установление неразрешимости данной массовой проблемы тоже может быть либо прямым, либо косвенным. При косвенном доказательстве мы также сводим одну проблему к другой, но это сведение производится в другую сторону — чтобы установить неразрешимость некоторой массовой проблемы, надо к ней свести другую массовую проблему, неразрешимость которой уже установлена. На протяжении нескольких первых глав книги мы будем сводить к 10-й проблеме Гильберта все более и более сложные проблемы. Эта цепочка сведений должна в конце концов оборваться на проблеме, для которой мы даём прямое доказательство неразрешимости. Чтобы дать такое доказательство, мы должны суметь каким-то образом обозреть все мыслимые способы решения проблемы. Принципиальная возможность сделать это появилась после выработки математически строгого общего понятия алгоритма. Соответствующие определения будут даны в главе 5, где и будет установлена алгоритмическая неразрешимость 10-й проблемы Гильберта. 

\subsection[§ 1.2. Системы диофантовых уравнений ]{\center § 1.2. Системы диофантовых уравнений }
\setcounter{equation}{0}
В 10-й проблеме Гильберт спрашивал про способ для установления существования или отсутствия решений лишь у отдельных диофантовых уравнений, хотя сам Диофант рассматривал и системы уравнений. Легко, однако, понять, что положительное решение 10-й проблемы Гильберта давало бы нам также способ узнавать наличие или отсутствие решений и у произвольных систем диофантовых уравнений. Действительно, система из $k$ диофантовых уравнений 

\begin{equation}
	\begin{split}
D_1(x_1, \dots, x_m)=0 \\
\dots\dots\dots\dots\dots\dots \\
D_k(x_1, \dots, x_m)=0 \\
	\end{split}
\end{equation}
 
\noindent имеет решение в целых $x_1, \dots, x_m$  тогда и только тогда, когда имеет решение диофантово уравнение 

\begin{equation}
D^2_1(x_1, \dots,x_m)+\dots+D^2_k(x_1, \dots,x_m)=0;
\end{equation}

\noindentболее того, множества решений у (1) и (2) совпадают. Таким образом, для систем диофантовых уравнений количество уравнений не является такой существенной характеристикой, как в случае систем линейных алгебраических или дифференциальных уравнений. 

В дальнейшем мы будем пользоваться и обратной возможностью — преобразованием диофантова уравнения 

\begin{equation}
	D(x_1, \dots,x_m)=0;
\end{equation}

\noindentв некоторую систему диофантовых уравнений 

\begin{equation}
	\begin{split}
		D_1(x_1, \dots, x_m,y_1,\dots,y_l)=0 \\
		\dots\dots\dots\dots\dots\dots\dots\dots\dots \\
		D_k(x_1, \dots, x_m,y_1,\dots,y_l)=0
	\end{split}
\end{equation}

\noindentимеющую, быть может, дополнительные неизвестные $y_1,\dots,y_l$. Переход от (3) и (4) не обязательно является обратным преобразованием к переходу от (1) к (2) — если систему (4) свернуть описанным выше способом в одно диофантово уравнение, то  этим уравнением окажется, вообще говоря, отнюдь не исходное уравнение (3). Единственная связь между (3) и (4), которая  будет нас интересовать, такова: уравнение (3) должно иметь  решение в том и только том случае, когда имеется решение у  системы (4). При этом не требуется ни чтобы каждое решение  уравнения (3) было продолжимо (посредством выбора значений  $y_1,\dots,y_l$) до какого-то решения системы (4), ни чтобы каждое  решение системы (4) содержало решение уравнения (3). 

Цель перехода от уравнения (3) к эквивалентной по разрешимости системе (4) может состоять в том, чтобы получить систему, в которой каждое отдельное уравнение было бы очень простым. Например, легко понять, что любое диофантово уравнение можно преобразовать в эквивалентную в описанном выше смысле систему, состоящую из уравнений двух типов 

\begin{equation}
\alpha = \beta + \gamma
\end{equation}
\noindent и 
\begin{equation}
	\alpha = \beta \gamma
\end{equation}

\noindent где $\alpha$, $\beta$ и $\gamma$ - конкретные натуральные числа или какие-то из неизвестных $x_1,\dots,x_m$, $y_1,\dots,y_l$. Проиллюстрируем такое преобразование на примере уравнения 

\begin{equation}
	4x^3y-2x^2z^3-3y^2x+5z=0.
\end{equation}

\noindent Сначала мы избавимся от вычитаний и получим уравнение 

\begin{equation}
	4x^3y+5z=2x^2z^3+3y^2x.
\end{equation}

\noindent Затем введём 14 новых переменных и получим эквивалентную  систему 

\begin{equation}
	\begin{split}
p_1=4x,\quad  p_2=p_1x,\quad  p_3=p_2x,\quad  p_4=p_3y; \\
q_1=5z; \\
r_1=2x,\quad  r_2=r_1x,\quad  r_3=r_2z,\quad  r_4=r_3z,\quad r_5=r_4z; \\
s_1=3y,\quad s_2=s_1y; \\
t_1=p_4+q,\quad u_1=r_5+s_2,\quad t_1=lu_1.\\
	\end{split}
\end{equation}

В качестве примера применения этой несложной техники  преобразования уравнений посмотрим, что получится, если сначала некоторое диофантово уравнение преобразовать в эквивалентную по разрешимости систему (1), состоящую из уравнений  типа (5) и (6), а затем свернуть эту систему в одно уравнение (2). Ясно, что исходное уравнение будет иметь или не иметь решение одновременно с новым уравнением (2); смысл такого двойного преобразования состоит в том, что новое  уравнение (2) будет иметь степень 4 независимо от степени  исходного уравнения. Таким образом, \textit{для положительного решения 10-й проблемы Гильберта было бы достаточно найти способ  узнавать наличие или отсутствие решений у уравнений 4-й степени}. 

\subsection[§ 1.3. Решения в натуральных числах]{\center § 1.3. Решения в натуральных числах}
\setcounter{equation}{0}

В 10-й проблеме Гильберт спрашивал про решения диофантовых уравнений в целых числах. Иногда разрешимость уравнения в целых числах очевидна; например, ясно, что уравнение 

\begin{equation}
	(x+1)^3+(y+1)^3=(z+1)^3
\end{equation}

\noindent имеет бесконечно много решений вида $x = z$, $y = -1$. В то же время тот факт, что уравнение (1) не имеет решений с неотрицательными $x$, $y$ и $z$, весьма нетривиален. Таким образом, для конкретного диофантова уравнения \textit{проблема распознавания наличия целочисленных решений и проблема распознавания наличия неотрицательных целочисленных решений} — это, вообще говоря, две разные массовые проблемы. 

С другой стороны, пусть 

\begin{equation}
	D(x_1,\dots,x_m)=0
\end{equation}

\noindent — произвольное диофантово уравнение, и мы интересуемся наличием у него неотрицательных решений. Рассмотрим систему уравнений 

\begin{equation}
	\begin{split}
	D(x_1,\dots,x_m)=0,\\
	x_1=y^2_{1.1}+y^2_{1.2}+y^2_{1.3}+y^2_{1.4},\\
	\dots \dots \dots \dots \\
	x_m=y^2_{m.1}+y^2_{m.2}+y^2_{m.3}+y^2_{m.4}.\\
	\end{split}
\end{equation}

\noindent Понятно, что любое решение этой системы в произвольных целых числах содержит решение уравнения (2) в неотрицательных целых числах. Верно и обратное — для любого решения уравнения (1) в неотрицательных целых числах $x_1,\dots,x_m$ найдутся целочисленные значения $y_{1.1},\dots,y_{m.4}$, дающие решение системы (3), поскольку каждое неотрицательное целое число представимо в виде суммы квадратов четырёх целых чисел (см. Приложение 1). Как мы знаем из § 1.2, система (3) может быть свёрнута в одно уравнение 

\begin{equation}
	E(x_1,\dots,x_m,y_{1.1},\dots,y_{m.4})=0,
\end{equation}

\noindent разрешимое в целых числах тогда и только тогда, когда исходное уравнение (2) разрешимо в неотрицательных целых числах. 

Таким образом, мы показали, что \textit{массовая проблема распознавания наличия решений в неотрицательных целых числах сводится к массовой проблеме распознавания наличия решений в целых числах}. Тем самым мы установили, что для доказательства неразрешимости 10-й проблемы Гильберта в её оригинальной постановке достаточно доказать неразрешимость её аналога, касающегося наличия или отсутствия решений в неотрицательных целых числах. По техническим причинам несколько удобнее работать с неотрицательными числами, и в дальнейшем везде, где явно не будет оговорено противное, строчные курсивные латинские буквы будут обозначать неотрицательные целые числа. По традиции, идущей от математической логики, мы будем называть такие числа \textit{натуральными}, считая тем самым 0 натуральным числом. 

Наши дальнейшие усилия будут направлены на доказательство неразрешимости аналога 10-й проблемы Гильберта для натуральных решений. Мы достигнем этой цели в главе 5, но a priori могло бы оказаться, что этот аналог разрешим, хотя проблема в исходной постановке неразрешима. Проверим, что это не так, т. е. что, ограничивая область изменения неизвестных натуральными числами, мы, в принципе, ничего не теряем. 

Пусть 

\begin{equation}
	D(\textit{x}_1,\dots,\textit{x}_m)=0
\end{equation}

\noindent — произвольное диофантово уравнение, и мы интересуемся его решениями в целых числах $x_1,\dots,x_m$. Рассмотрим уравнение

\begin{equation}
	D(x_1-y_1,\dots,x_m-y_m)=0.
\end{equation}

Ясно, что любое решение уравнения (6) (в натуральных числах $x_1,\dots,x_m$, $y_1,\dots,y_m$, по нашему соглашению) порождает решение 

\begin{equation}
	\begin{split}
\textit{x}_1=x_1-y_1 \\
\dots\dots\dots \\
\textit{x}_m=x_m-y_m \\
	\end{split}
\end{equation}

\noindent уравнения (5) в целых числах $\textit{x}_1,\dots,\textit{x}_m$. С другой стороны, для любого решения $\textit{x}_1,\dots,\textit{x}_m$ уравнения (5) найдутся натуральные числа $x_1,\dots,x_m$, $y_1,\dots,y_m$, удовлетворяющие (7) и тем самым образующие решение уравнения (6). 

Таким образом, мы осуществили обратное свед\textbf{е}ние — показали, что \textit{проблема распознавания наличия целочисленных решений сводится к проблеме распознавания наличия натуральных решений}. В результате оказывается, что две эти проблемы эквивалентны как \textit{массовые проблемы}, хотя, как обсуждалось в начале этого параграфа, для конкретного уравнения ответ может зависеть от области допустимых значений неизвестных. 

\subsection[§ 1.4. Диофантовы множества ]{\center § 1.4. Диофантовы множества }
\setcounter{equation}{0}

Наряду с системами диофантовых уравнений в теории чисел рассматриваются также \textit{семейства диофантовых уравнений}. Под семейством диофантовых уравнений мы понимаем равенство вида 


\begin{equation}
	D(a_1,\dots,a_n,x_1,\dots,x_m)=0
\end{equation}














\newpage
\section[ПРИЛОЖЕНИЯ]{\center ПРИЛОЖЕНИЯ}

\subsection[1. Теорема о четырёх квадратах]{\center 1. Теорема о четырёх квадратах}


\newpage
\section[УКАЗАНИЯ К УПРАЖНЕНИЯМ]{\center УКАЗАНИЯ К УПРАЖНЕНИЯМ}


\newpage
	\section[СПИСОК ЛИТЕРАТУРЫ]{\center СПИСОК ЛИТЕРАТУРЫ}
\renewcommand{\arraystretch}{1.3} %% increase table row spacing	
	Адлеман и Мандерс\quad  (Adleman L., Manders K.)


		\begin{tabular}{ p{3em}  p{15cm} }
			\lbrack1976\rbrack & Diophantine complexity // 17th Annual Symp. on Found, of Computer Sci. (Houston, Texas, 1976). - Long Beach, Calif.: IEEE Comput. Soc - P. 81-88.  
		\end{tabular}
		
\vspace{1em}

	Адлер\quad (Adler A.)
	
			\begin{tabular}{ p{3em}  p{15cm} }
\lbrack 1969а\rbrack & Some recursively unsolvable problems in analysis // Proc.	Amer. Math. Soc. - V. 22. N 2. - P. 523-526.\\
\lbrack1969б\rbrack & Extentions of nonstandard models of number theory //  Z. math. Logik Grundl. Math. - Bd 15, N 4. - S. 289-290. \\
\lbrack	1969в\rbrack  & Existential formulas in arithmetic //  Dissert. Abstrs. - V. 29, N 8. - P. 2962-2963.\\
\lbrack1971\rbrack & A reduction of homogeneous diophantine problem //  J. London Math. Soc. - V. 3. N 3. - P. 446-448.\\

	\end{tabular}

\vspace{1em}	
		Азра\quad  (Azra J. P.)
	
	\begin{tabular}{ p{3em}  p{15cm} }
		\lbrack 1971\rbrack & Relations diophantiennes et la solution negative du $10^e$ probleme de Hilbert // Lect. Notes Math. - V. 244. - P. 11-28. 
		
		
	\end{tabular}

	%	 
%
%

	\newpage
	\tableofcontents
	
	\thispagestyle{empty} % 
	
	\newpage
	
	\setcounter{secnumdepth}{0}  
	
	\phantomsection
	
		\section*{Описание}
	
	{\bf Название:} Десятая проблема Гильберта
	
{\bf Автор:} Матиясевич Ю.В.
	
{\bf Издательство:} М: Физматлит. 1993. - 224 с. - ISBN 5-02-014326-X
	
		{\bf Рецензент:}  доктор физико-математических наук С.И. Адян
	
		{\bf Аннотация:} Дается полное доказательство алгоритмической неразрешимости 10-й проблемы Гильберта, касающейся диофантовых уравнений, вместе с необходимыми сведениями из теории алгоритмов и теории чисел, а также приложения развитой для этого техники к другим массовым проблемам теории чисел, алгебры, анализа, теоретического программирования.
		
		Для математиков, в том числе аспирантов и студентов старших курсов.
		
		Библиогр. 247 назв.
		\thispagestyle{empty} % выключаем отображение номера для этой страницы

	
\end{document}

\printindex
