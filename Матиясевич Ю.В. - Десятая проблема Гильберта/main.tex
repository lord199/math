
%\documentclass[oneside,final,14pt]{extreport}
\documentclass[12pt, a4paper, openany]{book}



\usepackage[left=1cm,right=1cm,top=2cm,bottom=2cm,bindingoffset=0cm]{geometry}
%\usepackage[koi8-r]{inputenc}
%\usepackage[russianb]{babel}
\usepackage{vmargin}

\setpapersize{A4}
\usepackage[T2A]{fontenc}
\usepackage[utf8x]{inputenc}
\usepackage[english, russian]{babel}
\setmarginsrb{2cm}{2cm}{2cm}{2cm}{0pt}{5mm}{0pt}{0mm}
\usepackage{indentfirst}
\usepackage{nicefrac} % For comparison
%\usepackage{xfrac}    % Works better with other fonts
%\usepackage[unicode, pdftex]{hyperref}
\usepackage{lettrine}
\usepackage[usenames]{color}
\usepackage{colortbl}
\usepackage{mathtext}
\usepackage{epigraph}
\usepackage{amsmath, amsfonts, amssymb, mathrsfs}
%\usepackage{mathptmx}
%\usepackage{txfonts}
\usepackage{pxfonts}
\usepackage[pagestyles]{titlesec}
\usepackage{ebgaramond}
\usepackage{awesomebox}
\usepackage{enumitem}
\usepackage{makeidx}
\makeindex
    \usepackage{etoolbox}
\makeatletter
\newlength\epitextskip
\pretocmd{\@epitext}{\em}{}{}
\apptocmd{\@epitext}{\em}{}{}
\patchcmd{\epigraph}{\@epitext{#1}\\}{\@epitext{#1}\\[\epitextskip]}{}{}
\makeatother

\DeclareSymbolFont{Xlargesymbols}{OMX}{cmex}{m}{n}
\DeclareMathSymbol{\Xsum}{\mathop}{Xlargesymbols}{80}

\setlength\epigraphrule{0pt}
\setlength\epitextskip{2ex}
\setlength\epigraphwidth{.8\textwidth}

\usepackage{xfrac}    % Works better with other fonts
\usepackage[colorlinks=true,linkcolor=black,urlcolor=black,bookmarksopen=true]{hyperref}

\usepackage{fancyhdr} % пакет для установки колонтитулов
\pagestyle{fancy} % смена стиля оформления страниц
\fancyhf{} % очистка текущих значений
\fancyhead[C]{\thepage} % установка верхнего колонтитула
\renewcommand{\headrulewidth}{0pt} % убрать разделительную линию


% Настройка вертикальных и горизонтальных отступов
\titlespacing{\chapter}{0pt}{5pt}{5pt}
\titlespacing{\section}{\parindent}{4mm}{4mm}
\titlespacing{\subsection}{\parindent}{3mm}{3mm}


% Настройка задачи со зведочкой
\newcounter{namedlistcounter}  % number the items
\newenvironment{withdot}
{\begin{list}
		{\arabic{namedlistcounter}*.} % labeling 
		{\usecounter{namedlistcounter}   % set counter
			\setlength{\leftmargin}{3em}} % set spacing 
	}
	{\end{list}}


\newcommand{\anonsection}[1]{ \section*{#1} \addcontentsline{toc}{section}{\numberline {}#1}} 

\makeatletter %%%%% <---- Starting chapter without a pagebreak
\renewcommand\chapter{\par%
	\thispagestyle{plain}% \global\@topnum\z@
	\@afterindentfalse \secdef\@chapter\@schapter}
\makeatother %%%%% <---- Starting chapter without a pagebreak
\titleformat{\chapter}[display]
{\normalfont\bfseries}{}{0pt}{\Large}

\newpagestyle{mystyle}{
	\sethead[\thepage][][]{}{}{\thepage}	
}

\renewcommand{\rmdefault}{cmr}


\pagestyle{mystyle}
\sloppy

\begin{document}

	
	\begin{titlepage}
		
		\begin{center}
			%\vfill
			
			%\vfill
			\topskip 0pt
		%	\vspace*{\fill}
			\begin{flushleft}
		{\large	МАТЕМАТИЧЕСКАЯ\\ЛОГИКА\\И ОСНОВАНИЯ\\МАТЕМАТИКИ}			

			\ \\
\ \\
			\ \\
\ \\
			{\large Ю.В.МАТИЯСЕВИЧ\\}
			\ \\
			\ \\
			{\Huge\bf ДЕСЯТАЯ ПРОБЛЕМА ГИЛЬБЕРТА}


			\vspace*{\fill}    
			
			\vfill
			
Москва

Издательская фирма

«Физико-математическая литература»

ВО «Наука»

1\ 9\ 9\ 3

			\end{flushleft}
		\end{center}
		
	\end{titlepage}
	
	\thispagestyle{empty} % выключаем отображение номера для этой страницы
	
	\newpage
	
\setcounter{secnumdepth}{0}  
	
	\section[Предисловие]{\center ПРЕДИСЛОВИЕ}

На Втором Международном конгрессе математиков в Париже Давид Гильберт [1900] сделал свой знаменитый доклад «Математические проблемы», содержащий 23 проблемы или, точнее, 23 группы родственных проблем, которые 19-й век оставлял в наследие 20-му. Проблема под номером десять была посвящена диофантовым уравнениям.
	\begin{center}
		{\small 10. ЗАДАЧА О РАЗРЕШИМОСТИ ДИОФАНТОВА УРАВНЕНИЯ.}
\end{center}


\hangindent=1.5cm \hangafter=0 {\footnotesize Пусть задано диофантово уравнение с произвольными неизвестными и целыми рациональными числовыми коэффициентами. \textit{Указать способ, при помощи которого возможно после конечного числа операций установить, разрешимо ли это уравнение в целых рациональных числах.}}

\

Под «способом», который предлагает найти Д. Гильберт, в настоящее время подразумевают «алгоритм». В начале века, когда проблемы формулировались, ещё не было математически строго общего понятия алгоритма. Отсутствие такого понятия не могло само по себе служить препятствием к положительному решению 10-й проблемы Гильберта, поскольку про конкретные алгоритмы всегда было ясно, что они действительно дают требуемый общий способ решения соответствующих проблем.

В 30-е годы в работах К. Гёделя, А. Чёрча, А. М. Тьюринга и других логиков было выработано строгое общее понятие алгоритма, которое дало принципиальную возможность устанавливать алгоритмическую неразрешимость, т. е. доказывать невозможность алгоритма с требуемыми свойствами. Тогда же были найдены первые примеры алгоритмически неразрешимых проблем, сначала в самой математической логике, а затем и в других разделах математики.

Таким образом, теория алгоритмов создала необходимые предпосылки для попыток доказать неразрешимость 10-й проблемы Гильберта. Первые работы в этом направлении были опубликованы в начале 50-х годов, а в 1970 году исследования завершились «отрицательным решением» 10-й проблемы Гильберта.

В случае 10-й проблемы Гильберта, как и в случае других проблем, долго ожидавших своего решения, не меньшее, а пожалуй, большее значение имеет математический аппарат, развитый для решения проблемы и находящий затем другие приложения, порой неожиданные. Основной технический результат, полученный при доказательстве неразрешимости 10-й проблемы  Гильберта — это теорема о совпадении класса \textit{диофантовых множеств} и класса \textit{перечислимых множеств}. В качестве одного из следствий этой теоремы, формулировка которого не содержит специальных терминов, приведем следующее: \textit{можно явно указать  полином от многих переменных с целыми коэффициентами такой,  что множество всех его положительных значений, принимаемых  при целочисленных значениях переменных, есть в точности  множество всех простых чисел}. 

Настоящая книга посвящена алгоритмической неразрешимости 10-й проблемы Гильберта и родственным вопросам; многочисленные частичные результаты, полученные в направлении положительного решения 10-й проблемы Гильберта, здесь почти  не рассматриваются. 

Отрицательное решение 10-й проблемы Гильберта излагали (с разной степенью детализации) многие авторы, в частности: Азра [1971], Белл и Маховер [1977]. Гермес [1972, 1978], Девис [1973а, 1974]. Захаров [1970, 1986], Капланский [1977], Манин [1973, 1977], Маргенштерн [1981]. Матиясевич [1972а], Мияйлович, Маркович и Дошен [1986], Руохонен [1972, 1980], Саломаа [1985], Смориньский [1987], Сусман [1971], Такахаши [1974], Фенстад [1971], Хавел [1973], Хиросе [1973]. 

Одной из отличительных особенностей настоящей книги является то. что она. помимо собственно отрицательного решения 10-й проблемы Гильберта, содержит ряд приложений разработанной для этого решения техники; приложения эти в настоящее  время разбросаны, в основном, по журнальным публикациям. За два десятилетия, прошедшие со времени решения проблемы, были получены многообразные упрощения и модификации первоначального доказательства. Настоящая книга также содержит ряд новых, ранее не публиковавшихся доказательств. 
	\newpage
	\tableofcontents
	
	\thispagestyle{empty} % 
	
	\newpage
	
	\setcounter{secnumdepth}{0}  
	
	\phantomsection
	
		\section*{Описание}
	
	{\bf Название:} Десятая проблема Гильберта
	
{\bf Автор:} Матиясевич Ю.В.
	
{\bf Издательство:} М: Физматлит. 1993. - 224 с. - ISBN 5-02-014326-X
	
		{\bf Рецензент:}  доктор физико-математических наук С.И. Адян
	
		{\bf Аннотация:} Дается полное доказательство алгоритмической неразрешимости 10-й проблемы Гильберта, касающейся диофантовых уравнений, вместе с необходимыми сведениями из теории алгоритмов и теории чисел, а также приложения развитой для этого техники к другим массовым проблемам теории чисел, алгебры, анализа, теоретического программирования.
		
		Для математиков, в том числе аспирантов и студентов старших курсов.
		
		Библиогр. 247 назв.
		\thispagestyle{empty} % выключаем отображение номера для этой страницы

	
\end{document}

\printindex
