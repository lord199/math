
%\documentclass[oneside,final,14pt]{extreport}
\documentclass[12pt, a4paper, openany]{book}



\usepackage[left=1cm,right=1cm,top=2cm,bottom=2cm,bindingoffset=0cm]{geometry}
%\usepackage[koi8-r]{inputenc}
%\usepackage[russianb]{babel}
\usepackage{vmargin}

\setpapersize{A4}
\usepackage[T2A]{fontenc}
\usepackage[utf8x]{inputenc}
\usepackage[english, russian]{babel}
\setmarginsrb{2cm}{2cm}{2cm}{2cm}{0pt}{5mm}{0pt}{0mm}
\usepackage{indentfirst}
\usepackage{nicefrac} % For comparison
%\usepackage{xfrac}    % Works better with other fonts
%\usepackage[unicode, pdftex]{hyperref}
\usepackage{lettrine}
\usepackage[usenames]{color}
\usepackage{colortbl}
\usepackage{mathtext}
\usepackage{epigraph}
\usepackage{amsmath, amsfonts, amssymb, mathrsfs}
%\usepackage{mathptmx}
%\usepackage{txfonts}
\usepackage{pxfonts}
\usepackage[pagestyles]{titlesec}
\usepackage{ebgaramond}
\usepackage{awesomebox}
\usepackage{enumitem}
\usepackage{makeidx}
\makeindex
    \usepackage{etoolbox}
\makeatletter
\newlength\epitextskip
\pretocmd{\@epitext}{\em}{}{}
\apptocmd{\@epitext}{\em}{}{}
\patchcmd{\epigraph}{\@epitext{#1}\\}{\@epitext{#1}\\[\epitextskip]}{}{}
\makeatother
%running fraction with slash - requires math mode.
\newcommand*\rfrac[2]{{}^{#1}\!/_{#2}}

\DeclareSymbolFont{Xlargesymbols}{OMX}{cmex}{m}{n}
\DeclareMathSymbol{\Xsum}{\mathop}{Xlargesymbols}{80}

\setlength\epigraphrule{0pt}
\setlength\epitextskip{2ex}
\setlength\epigraphwidth{.8\textwidth}

\usepackage{xfrac}    % Works better with other fonts
\usepackage[colorlinks=true,linkcolor=black,urlcolor=black,bookmarksopen=true]{hyperref}

\usepackage{fancyhdr} % пакет для установки колонтитулов
\pagestyle{fancy} % смена стиля оформления страниц
\fancyhf{} % очистка текущих значений
\fancyhead[C]{\thepage} % установка верхнего колонтитула
\renewcommand{\headrulewidth}{0pt} % убрать разделительную линию


% Настройка вертикальных и горизонтальных отступов
\titlespacing{\chapter}{0pt}{5pt}{5pt}
\titlespacing{\section}{\parindent}{4mm}{4mm}
\titlespacing{\subsection}{\parindent}{3mm}{3mm}


%
%\renewcommand{\tabcolsep}{1cm}   %% increase table column spacing

% Настройка задачи со зведочкой
\newcounter{namedlistcounter}  % number the items
\newenvironment{withdot}
{\begin{list}
		{\arabic{namedlistcounter}*.} % labeling 
		{\usecounter{namedlistcounter}   % set counter
			\setlength{\leftmargin}{3em}} % set spacing 
	}
	{\end{list}}


\newcommand{\anonsection}[1]{ \section*{#1} \addcontentsline{toc}{section}{\numberline {}#1}} 

\makeatletter %%%%% <---- Starting chapter without a pagebreak
\renewcommand\chapter{\par%
	\thispagestyle{plain}% \global\@topnum\z@
	\@afterindentfalse \secdef\@chapter\@schapter}
\makeatother %%%%% <---- Starting chapter without a pagebreak
\titleformat{\chapter}[display]
{\normalfont\bfseries}{}{0pt}{\Large}

\newpagestyle{mystyle}{
	\sethead[\thepage][][]{}{}{\thepage}	
}

\renewcommand{\rmdefault}{cmr}


\pagestyle{mystyle}
\sloppy

\begin{document}

	
	\begin{titlepage}
\begin{center}

			%\vfill
			
			%\vfill
			\topskip 0pt
		%	\vspace*{\fill}

		{\large	А. Н. КОЛМОГОРОВ

			\ \\
\ \\
			\ \\
\ \\
			\vfill
			\ \\
			\ \\
			{\Huge\bf О ПРОФЕССИИ МАТЕМАТИКА}
			\ \\
				\ \\
			\textit{(Издание третье, дополненное)}


			\vspace*{\fill}    
			
			\vfill
			
И З Д А Т Е Л Ь С Т В О

МОСКОВСКОГО УНИВЕРСИТЕТА

1960
}
\end{center}
	\end{titlepage}
	
	\thispagestyle{empty} % выключаем отображение номера для этой страницы

	\newpage
	\begin{center}
		
			\ \\
\ \\
\ \\
			\ \\
\ \\
\ \\
	Печатается по постановлению
	
	Редакционно-издательского совета
	
	Московского университета
\end{center}
	
	\newpage
	
\setcounter{secnumdepth}{0}  
	\ \\
	\ \\
	\section[1. За многочисленное и талантливое пополнение кадров советских математиков]{\center 1. ЗА МНОГОЧИСЛЕННОЕ И ТАЛАНТЛИВОЕ ПОПОЛНЕНИЕ КАДРОВ СОВЕТСКИХ МАТЕМАТИКОВ}

Значение математических методов в таких науках, как механика, физика или астрономия, хорошо известно. Также
всем известно и то, что математика необходима в практической работе инженеров и техников. Элементарные знания по геометрии или умение пользоваться буквенными формулами необходимы почти каждому мастеру или квалифицированному рабочему. Но менее ясным для многих является вопрос о том, что значит иметь специальность математика и заниматься самой математикой в качестве основной профессии.

Очень многие представляют себе дело так, что в учебниках и математических справочниках собрано уже вполне достаточно формул и правил для решения всевозможных, встречающихся на практике математических задач. Даже очень образованные люди часто спрашивают с недоумением: разве в математике можно сделать что-либо новое?

Поэтому и математика иногда представляют себе как скучного человека, выучившего большое число формул.и теорем, и считают, что его задача состоит в том, чтобы заученные готовые знания передать другим.

Во всем этом верно только то, что математические сведения, сообщаемые в средней школе и на первых ступенях изучения математики в высшей школе, добыты человечеством давно. Но даже и эти простейшие математические сведения могут применяться умело и с пользой только в том случае, если они усвоены творчески, так, что учащийся видит сам, как можно было бы прийти к ним самостоятельно. От преподавателя математики и в высшей и средней школе требуется не только твердое знание преподаваемой им науки. Хорошо преподавать математику может только человек, который сам ею увлечен и воспринимает ее как живую, развивающуюся науку. Вероятно, многие учащиеся средней школы знают, насколько увлекательной, а благодаря этому легкой и доступнои становится математика у таких преподавателей.

Еще в большей степени самостоятельность и способность по-новому подойти к математической формулировке задачи необходимы тому, кто применяет математику в решении технических проблем. Это относится к работе каждого инженера. Но так как требующиеся при этом математические знания и способности имеются не у всех, то болыцинство наших научно-исследовательских технических институтов и даже некоторые крупные заводы стали усиленно привлекать специалистов-математиков для работы вместе с инженерами над техническими проблемами.

Математики, способные руководить большими вычислительными работами, особенно дефицитны. В настоящее время имеется много задач, в которых для получения числового результата требуются вычисления, превосходящие возможности одного человека. Расчет упругих напряжений в плотинах, фильтрации воды под плотинами, сопротивлений, испытываемых самолетами при полете, или траекторий снарядов — вот типичные примеры таких задач.

Уже давно при научных институтах, проектных организациях и заводах, нуждающихся в решении подобных задач,
стали возникать вычислительные бюро со многими десятками вычислителей, оборудованные арифмометрами и вычислительными автоматами, требующими для выполнения арифметических действий над многозначными числами лишь набора их при помощи клавиш и нажатия соответствующей кнопки ($+$, $-$, $\times$, $\div$). Однако современные наука и техника сталкиваются с такими задачами, которые при этом уровне организации вычислительных работ требуют многих месяцев, а иногда и лет работы десятков вычислителей. Такое положение вызвало бурное развитие современной «машинной математики», о которой рассказывается в этой брошюре.

Конструирование и обслуживание современных вычислительных машин превратились в большие инженерные специальности, для которых специалисты готовятся на соответствующих отделениях технических вузов. Для работы же вычислителя в вычислительном бюро старого типа или для введения задачи в современную электронную вычислительную машину достаточно среднего общего образования и полугодичного производственного обучения. Для того чтобы довести решение математических задач до передачи для получения численных результатов вычислительному бюро или вычислительной машине необходимо большое количество людей с глубокими математическими знаниями.

Теория «вычислительных методов» математики развилась сейчас в большую науку и потребность в специалистах, владеющих этими методами, с развитием «машинной математики» возрастает. Перед нами возникают своеобразные задачи «программирования», т. е. приведения процесса вычислений к виду, допускающему полную автоматизацию решения на машинах задач определенного типа.

Ошибочным является представление о математике как о науке законченной, раз навсегда построенной в своих теоретических основах. В действительности математика обогащается совершенно новыми теориями и перестраивается в ответ на новые запросы механики (нелинейные колебания, механика сверхзвуковых скоростей), физики (математические методы квантовой физики) и других смежных наук. Кроме того, и в недрах самой математики после накопления большого числа разрозненных специальных задач, решенных частными приемами, создаются новые общие теории, освешающие эти задачи с иных точек зрения и позволяющие решать их однообразными методами. Например, методы возникающего на наших глазах «функционального анализа» относятся к математическому анализу (который был создан еще в XVII—XVIII вв. и преподается во всех высших технических заведениях) примерно так, как относится алгебра к арифметике. Так называемые «операторные методы» функционального анализа уже нашли широкое применение в современной физике и технике.

Советскому Союзу сейчас требуется большое количество самостоятельных исследователей по теоретическим вопросам математики. При сравнении изданных обзоров успехов советской математики за 1917—1947 гг. обнаруживается, что в первом пятнадцатилетии было около двухсот математиков, внесших в математическую науку что-либо существенно новое, во втором же пятнадцатилетии — 600—800.

Количество математиков с университетской подготовкой, требующихся для работы над задачами, выдвигаемыми естествознанием и техникой, значительно больше, особенно если учесть, что, кроме теоретической разработки вопроса, здесь, как правило, необходимо проведение больших расчетных работ. Постоянно возрастает ежегодная потребность научных и научно-технических институтов и «вычислительных центров» в молодых сотрудниках-математиках, выпущенных университетами.

Если учесть еще потребность нашей страны в преподавателях математики в педагогических и учительских институтах, то станет понятным, почему Советскому государству требуется так много математиков самой высокой квалификации, подготовляемых на механико-математических и физико-математических факультетах университетов.

За последние годы в нашей стране проведены важные мероприятия, направленные на повышение квалификации преподавателей математики высших учебных заведений, на привлечение в университеты большого числа молодежи, имеющей склонность к математике.

Интересно в связи с этим вспомнить, что в первые годы после Великой Октябрьской социалистической революции молодежь стремилась почти исключительно в высшие технические учебные заведения. Многим молодым людям представлялось тогда, что только таким путем они примут непосрелдственное участие в социалистическом строительстве. В первые революционные годы такие настроения имели некоторое разумное основание. Но потом, когда развитие науки стало насущнейшей с хозяйственной точки зрения потребностью нашей страны, необходимы были усилия, чтобы преодолеть недоверие части молодежи к перспективам, ожидающим ее при поступлении в университеты. Эти настроения теперь изжиты. Но в применении к математике, которая издали, даже среди других наук, представляется слишком сухой и отвлеченной, с ними приходится бороться еще и сейчас.

С 1952 г. прием на математические специальности университетов СССР значительно увеличен по сравнению с предыдущими годами. Очень важно, чтобы при этом расширенном приеме на математические специальности попала не только хорошо подготовленная, но и любящая математику молодежь{\footnote{В Московском, Ленинградском, Киевском, Саратовском и Томском государственных университетах имеются механико-математические (или математико-механические) факультеты с основными специальностями: математика, механика. В остальных университетах имеются физико-математические факультеты со специальностями. математика, физика, механика и астрономия.}}. Для этого необходимо, чтобы всюду на местах была создана возможность этим любителям математики определить свои склонности и оценить свои силы и возможности.

Чтобы сделать выбор вполне сознательно, полезно принять участие в работе математического кружка и в местной математической олимпиаде. Быть может, еще более полезно почитать соответствующую литературу{\footnote{См. список литературы в приложении.}} и попробовать свои силы в решении более трудных задач.


	\section[2. Несколько замечаний о характере работы математика-исследователя]{\center 2. НЕСКОЛЬКО ЗАМЕЧАНИЙ О ХАРАКТЕРЕ РАБОТЫ МАТЕМАТИКА-ИССЛЕДОВАТЕЛЯ}


Как и всякая наука, математика требует прежде всего твердого знания того, что по исследуемому вопросу уже сделано. Но не следует думать, что в математике труднее, чем в других науках, добраться до возможности делать что-либо новое. Опыт говорит скорее о другом: способные математики, как правило, начинают самостоятельные научные исследования очень рано. Если математические открытия, сделанные в 16- или 17-летнем возрасте, являются все же исключениями, собираемыми с особенной тщательностью в популярных книжках по истории математики, то начало серьезной научной работы в 19—20 лет на средних курсах университетов достаточно типично для биографий многих наших ученых{\footnote{Академик С. Л. Соболев в 1933 г. в возрасте 25 лет был уже избран в члены-корреспонденты АН СССР. В 1953 г. членом-корреспондентом АН СССР избран 25-летний математик комсомолец С. Н. Мергелян.}}.

Конечно, широта постановки задач приходит обычно несколько позднее, но при решении отчетливо поставленных трудных конкретных задач совсем молодые люди часто с успехом соревнуются со сложившимися известными учеными. Ежегодно около десятка научных работ, выполненных студентами математических специальностей Московского университета, публикуется в таком издании, как Доклады Академии наук СССР.

В основе большинства математических открытий лежит какая-либо простая идея: наглядное геометрическое построение, новое элементарное неравенство и т. п. Нужно только применить надлежащим образом эту простую идею к решению задачи, которая с первого взгляда кажется недоступной. Много примеров этого можно найти и в популярной литературе, указанной в конце нашей брошюры. Поэтому вовсе не существует непроходимой стены между самыми новыми и трудными оригинальными математическими исследованиями и решением задач, доступных способному и достаточно упорному начинающему математику. Интересно с этой точки зрения прочесть некоторые главы из «Математической автобиографии» знаменитого советского алгебраиста Н. Г. Чеботарева{\footnote{Опубликована в журнале «Успехи математических наук», т. III, вып. 3, 1948.}}, где автор излагает историю своих научных поисков, начиная с первых опытов гимназиста до крупнейших открытий в алгебре.

Другое замечание относится к работе математиков над вопросами естествознания (механики, физики и техники). Сейчас, когда сотрудничество между математиками и представителями смежных специальностей развивается особенно широко, можно определенно сказать, что наиболее успешным оно оказывается при условии, если математик не ограничивается ролью исполнителя сделанного ему «заказа», а старается проникнуть в существо естественнонаучных и технических проблем. По существу здесь речь идет о том, что специалисты по математической и теоретической физике, теоретической механике или теоретической геофизике могут подготавливаться двумя путями: начинать свое образование с изучения физики, механики или геофизики, или же сначала изучать математику на математических отделениях университетов и потом основательно входить в ту или иную область применения математики.

Существует даже такая точка зрения, что второй путь дает лучшие результаты, т. е. что изучить на солидной математической основе аэромеханику, газовую динамику, сейсмологию или динамическую метеорологию легче, чем специалисту в какой-либо из этих областей восполнить недостаток математической подготовки. Такое мнение можно считать слишком крайним и заметить, например, что хорошее владение экспериментальной техникой встречается у математиков, перешедших на работу в какой-либо смежной области, лишь как редкое исключение. Но нельзя не признать, что из математиков по образованию произошел ряд крупнейших наших специалистов в смежных науках.

Трудно отделить математику от механики и сейсмологии в работах академиков М. А. Лаврентьева и С. Л. Соболева. В первую очередь как механики известны академики М. В. Келдыш, Л. И. Седов и чл.-кор. АН СССР Л. Н. Сретенский; как геофизики — члены-корреспонденты АН СССР А. Н. Тихонов и А. М. Обухов; как специалист по теоретической физике — акад. Н. Н. Боголюбов. Между тем все они окончили университеты в качестве математиков.

Можно было бы указать много связанных с именами математиков конкретных достижений в естествознании и технике, которые оказались весьма существенными с непосредственно практической стороны.


	\section[3. О математических способностях]{\center 3. О МАТЕМАТИЧЕСКИХ СПОСОБНОСТЯХ}


Необходимость специальных способностей для изучения и понимания математики часто преувеличивают. Впечатление исключительной трудности математики иногда создается ее плохим, чрезмерно формальным изложением на уроке. Обычныс средние человеческие способности вполне достаточны, чтобы при хорошем руководстве или по хорошим книгам не только усвоить математику, преподающуюся в средней школе, но и разобраться, например, в началах дифференциального и интегрального исчислений. Тем не менее, когда дело идет о выборе математики в качестве основной специальности, вполне естественно желание проверить математические способности, или, как говорят иногда, математическую «одаренность». Ведь несомненно, что разные люди воспринимают математические рассуждения, решают математические задачи или — на более высокой ступени — приходят к новым математическим открытиям с различной скоростью, легкостью и успехом. И, конечио, следует стремиться к тому, чтобы из миллионов нашей молодежи специалистами-математиками становились именно те, кто в этой области будет работать наиболее успешно.

Поэтому содействие выдвижению математически одаренной молодежи является одной из важных задач школьных математических кружков, математических олимпиад и других мероприятий по пропаганде математических знаний и распространению интереса к самостоятельным занятиям математикой: Не следует спешить с чрезмерно оанним созданием для отдельных молодых людей репутации математических «талантов». Но вовремя подтолкнуть советом или премированием на олимпиаде способных математиков в сторону выбора математики в качестве своей дальнейшей работы необходимо.

В чем же заключаются эти способности? Следует прежде всего подчеркнуть, что успех в математике меныше всего основан на механическом запоминании большого числа фактов, отдельных формул и т. п. Хорошая память в математике, как и во всяком другом деле, является полезной, но никакой особенной, выдающейся памятью большинство крупных ученых-математиков не обладало.

В частности, фокусники, запоминающие длинные ряды многозначных чисел и складывающие или перемножающие их в уме, совсем не могут служить примером людей с хорошими математическими способностями в серьезном смысле слова.

Умение производить алгебраические вычисления, в смысле умелого преобразования сложных буквенных выражений, нахождения удачных путей для решения уравнений, не подхолящих под стандартные правила, и т. п., уже ближе соприкасается с теми способностями, которые часто требуются от математика в серьезной научной работе.

Принято даже думать, что исключительно большое развитие таких вычислительных, или иногда говорят «алгоритмических», способностей является характерным для одного из нескольких основных типов математической одаренности.

В школьной алгебре с трудностями, требующими для своего преодоления такого рода способностей, школьники прежде всего сталкиваются при разложении алгебраических выражений на множители. Среди задач, данных в приложении 3 к этой брошюре, задачи 1 и 2 дают представление о том, что иногда разложение очень простых выражений на множители требует большого остроумия.

Далее, основной областью применения этого рода способностей становится решение уравнений. Однако везде, где это возможно, математики стремятся сделать изучаемые ими проблемы геометрически наглядными. В средней школе достаточно ясно видно, насколько полезны графики для изучения свойств функций. Поэтому читатель не удивится утверждению, что геометрическое воображение, или, как говорят, «геометрическая интуиция», играет большую роль при исследовательской работе почти во всех разделах математики, даже самых отвлеченных.

В школе обычно с большим трудом дается наглядное представление пространственных фигур. Надо, например, быть уже очень хорошим математиком (по сравнению с обычным школьным уровнем), чтобы, закрыв глаза, без чертежа ясно представить себе, какой вид имеет пересечение поверхности куба с плоскостью, проходящей через центр куба и перпендикулярной одной из его диагоналей.

В задаче 4 (приложение 3) вся трудность заключается в том, чтобы наглядно понять, что за фигура получается при пересечении тетраэдров. При решении задач 5—7 тоже очень существенна геометрическая интуиция, хотя здесь уже больше остается и на долю твердого знания тех теорем, которые придется применить при доказательстве, и на долю умения логически рассуждать.

Искусство последовательного, правильно расчлененного логического рассуждения является также существенной стороной математических способностей.

В школе для развития этого искусства служит систематический курс теометрии с ее определениями, теоремами и доказательствами. Но часто наибольшую трудность для школьников в отношении понимания точного смысла сложной логической конструкции представляет принцип математической индукции, изучаемой в конце курса алгебры. Многие не в состоянии ясно увидеть реальное содержание этого принципа за нагромождением слов «если» и «то».

Понимание и умение правильно применять принцип математической индукции является хорошим критерием логическои зрелости, которая совершенно необходима математику.

Умение последовательно, логически рассуждать в незнакомой обстановке приобретается с трудом. На математических школьных олимпиадах самые неожиданные трудности возникают именно при решении задач, в которых не предполагается никаких предварительных знаний из школьного курса, но требуется правильно уловить смысл вопроса и рассуждать последовательно. Уже такой шуточный вопрос затрудняет многих десятиклассников: в хвойном лесу 800 000 елеи и ни на одной из них не более 500 000 игл; доказать, что по крайней мере у двух елей число игл точно одинаково (сравните в приложении 3 задачу 8; в задачах 10—12 тоже главная трудность не в сложности, а в необычности тех способов рассуждения, которые требуется применить).

Различные стороны математических способностей встречаются в разных комбинациях. Уже исключительное развитие одной из них иногда позволяет приходить к неожиданным и замечательным открытиям, хотя чрезмерная односторонность, конечно, опасна. Само собой разумеется, что никакие способности не помогут без увлечения своим делом, без систематической повседневной работы.

Математические способности проявляются обычно довольно рано и требуют непрерывного упражнения. Полный отрыв от математики в течение нескольких лет после средней школы часто оказывается трудно поправимым. Работа чертежника, лаборанта, обращение с машиностроительными деталями, сборка радиоаппаратуры и т. п., по-видимому, содержат в себе много элементов, родственных с работой математика, например, в смысле развития пространственного воображения и функционального мышления. Соприкосновение на работе с современной тсхникой может пробудить более сознательный интерес к приложениям математики. Но мы очень советуем молодым людям, намеревающимся поступить на математическое отделение университета, проработав после школы несколько лет на производстве, заранее заниматься математикой и не только путем подготовки к вступительным экзаменам (для чего при всех университетах существуют специальные подготовительные курсы), но и путем участия в математических кружках и олимпиадах и самостоятельного чтения. Иначе никакие льготы для «производственников» при поступленни в вузы не помогут им во время работы в университете не отстать от своих товарищей, пришедших со свежими знаниями и увлечениями прямо из школы.

	\section[4. Математические кружки, олимпиады, самостоятельное чтение, подготовка к вступительным экзаменам в университеты]{\center 4. МАТЕМАТИЧЕСКИЕ КРУЖКИ, ОЛИМПИАДЫ, САМОСТОЯТЕЛЬНОЕ ЧТЕНИЕ, ПОДГОТОВКА К ВСТУПИТЕЛЬНЫМ ЭКЗАМЕНАМ В УНИВЕРСИТЕТЫ}



Преподавание в школе во время обязательных классных занятий рассчитано в основном на твердое усвоение математики всеми учащимися. Попробовать свои силы в решении более трудных задач, ближе познакомиться с тем, как наука справляется с решением более сложных математических проблем, и с тем, как математика применяется в естествознании и в технике, можно в математическом кружке. Такие кружки ведут преподаватели математики во многих школах, Силами университетов и педагогических институтов во многих городах организованы межшкольные математические кружки и систематическое чтение лекций для школьников по отдельным вопросам математики или ее истории.

Естественно, что все эти начинания, как и математические олимпиады, широко открыты и для работающей молодежи, интересующейся математикой.

Математические олимпиады, на которых предлагаются трудные задачи и «победителям» выдаются премии и похвалъные отзывы, удаются там, где хорошо поставлена работа в кружках. Олимпиады должны проводиться для завершения работы, ведущейся в течение года, а не как изолированное праздничное мероприятие.

Задачи, предлагаемые в кружках и на олимпиадах, иногда носят искусственный и даже шуточный характер. В этом нет беды, если задачи подобраны так, что для их решения требуется серьезная работа мысли, похожая на ту, которая требуется от взрослого, самостоятельно работающего математика.

В докладах, читаемых в кружках их участниками, и в лекциях, читаемых учителями и преподавателями высшей школы, широко освещаются основные пути развития математической науки, значение математики для естествознания и техники. Конечно, очснь хорошо, если удается в задачах, предлагаемых в кружках, дать принципиально важный или убедительный своей полезностью материал, но было бы напрасно требовать, чтобы таким условиям была полчинена вся та болышая «тренировочная» работа молодого математика, которая достигается решением задач.

Независимо от участия в кружках можно заняться самостоятельным решением более трудных задач. Имеется несколько интереспых сборников задач для любителей математики. Некоторые из них написапы так, что читатель, решая последовательно связанные друг с другом задачи, может живо представить себе пути развития довольно сложных математических теорий. Кроме таких задачников повышенного уровня, в приложении 4 к этой брошюре указано много вполне доступных книжек по отдельным вопросам математики. Некоторые из книжек, минуя, по возможности, технические трудности, вводят читателя в круг вопросов, служащих о и в настоящее время предметом еще не законченного научного исследования.

Занятия в кружках, слушание лекций и чтение дополнительной литературы не должны, конечно, отвлекать учащихся школ или подготовительных курсов от более элементарной обязательной учебной работы. Следует помнить, что для того, чтобы быть принятым в университет, прежде всего требуется твердое знание школьного курса н умение на основе этих знании четко и уверенно решать более обычные, так сказать, стандартные задачи.

В приложении 2 приводятся примеры типичных задач, предлагавшихся на экзаменах при поступлении на механико-математический факультет Московского государственного университета. Если сравнить эти задачи с предлагавшимися на олимпиадах, то можно заметить их существенное отличие. Для решения экзаменационных задач не требуется какой-либо особой изобретательности. В большинстве случаев задачи решаются последовательным применением изучаемых в школе правил и приемов. Если же их решение и требует некоторой самостоятельности мысли, то. дело ограничивается необходимостью систематически исследовать поставленный вопрос в самом естественном направлении. Например, приступая к решению задачи 5 {приложение 2), следует ясно представить себе, как должен быть распсложен в шестиугольнике искомый квадрат для того, чтобы его нельзя было увеличить, не выходя за пределы шестиугольника. Ясно, что для этого он должен упираться в периметр шестиугольника по меньшей мере двумя вершинами. Такие квадраты (у которых по меньшей мере две вершины лежат на периметре шестиугольника) надо подвергнуть более детальному исследованию. К сожалению, некоторые экзаменующиеся не могли преодолеть уже этого первого этапа и даже предлагали в качестве решения задачи чертежи, в которых квадрат свободно висел внутри шестиугольника, не прикасаясь к его периметру.
	
	Иногда экзаменующимся с целью проверить на решении одной задачи их знание целого ряда формул, правил и теорем школьного курса экзаменаторы предлагают задачи со сложными формулировками условий, придавая им весьма. искусственный и запутанный вид. Независимо от вопроса о правильности такой тенденции не следует чрезмерно бояться задач этого рода. По своей идее они обычно бывают даже значительно элементарнее задач с более короткими и красивыми формулировками. Все дело при решении таких комбинированных задач со сложно и запутанно формулируемыми условиями сводится обычно к тому, чтобы правильно прочесть условия задачи и не запутаться в длинном ряде выкладок и рассуждений, каждое звено которых вполне элементарно, хотя и требует применения ряда формул и теорем из школьного курса.
	
Очень важно правильно распределить свои силы между твердым усвоением школьного курса, серьезным продумыванием наиболее существенных и трудных с идейной стороны узловых вопросов этого курса, тренировкой в решении задач конкурсного типа и (при наличии для этого своболного времени) развитием своих более самостоятельных интересов путем дополнительного чтения, участия в кружках и олимпиадах.
	
Мне хочется в заключение заметить. что по наблюдениям многих преподавателей Московского университета сборники конкурсных задач и материалы школьных кружков и олимпиад поселили в некоторой части нашей молодежи чрезмерный страх перед поступлением в университет (и, в частности, в Московский). Для каждого поступающего естественно желание достигнуть того, чтобы уверенно решать любую задачу конкурсного типа, но не следует думать, что в университеты принимают только решивших все предложенные на экзаменах задачи,
	
В приложении 2 к этой брошюре приведены билеты предлагавшиеся на письменных экзамепах по математике поступающим на механико-математический факультет МГУ{\footnote{В настоящее время на мехмате производится два экзамена по математике — письменный и устный. До 1955 г. система была несколько сложнее: один устный экзамен и два письменных — по геометрии и по алгебре. В первых двух изданиях этой брошюры приведены образцы экз. билетов, соответствовавших этой более сложной системе, которая теперь отменена}}. Билеты содержат по четыре задачи. Грубо говоря, оценки 5, 4, 3 соответствовали четырем, трем и двум решенным задачам. Решавшие менее двух задач, как правило, получали «двойку» и к дальнейшим экзаменам не допускались.
	
На устных экзаменах задача экзаменатора в советском вузе, вопреки распространенному воззрению школьников, состоит не в том, чтобы поскорее «срезать» пезадачливого поступающего, а в том, чтобы тщательно взвесить, учитывая все обстоятельства экзаменационной обстановки, перспективы его дальнейшей работы по избранной им специальности.
	


	\newpage
	\tableofcontents
	
	\thispagestyle{empty} % 

	\newpage

\setcounter{secnumdepth}{0}  

\phantomsection	
	\begin{center}
				\ \\
	\ \\
	\ \\
	\ \\
	\textbf{Колмогоров Андрей Николаевич}
	\ \\
	О ПРОФЕССИИ МАТЕМАТИКА
		\ \\
			\ \\
	Редактор \textit{С. Ф. Кондрашкова}
	
Техн. редактор \textit{Г. И. Георгиева}
	\ \\
		\ \\
\parbox{7,2cm}{%
	\sloppy\setlength\parfillskip{0pt}
	Сдано в производство 15. VI 1959 г.
	
	Подписано к печати 14. III 60 г.
	
	Л-90183 \ \ Формат бум. 60$\times$92$\rfrac{1}{16}$
	
	
	Печ. л. 2,0 \ \ Бум. д. 1,0 \ \  Уч-изд. д. 1,88
}

Заказ 783

Тираж 40 000 (2 $\cdot$ 15 000) Цена 55 к.

Издательство Московского

универсистета

Москва, Ленинские горы

Административный корпус

	\ \\
		\ \\

1 типография Издательства МГУ

Москва, Моховая, 9





	\end{center}
			\thispagestyle{empty} % выключаем отображение номера для этой страницы


	
\end{document}

\printindex
