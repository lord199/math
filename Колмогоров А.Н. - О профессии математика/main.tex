
%\documentclass[oneside,final,14pt]{extreport}
\documentclass[12pt, a4paper, openany]{book}



\usepackage[left=1cm,right=1cm,top=2cm,bottom=2cm,bindingoffset=0cm]{geometry}
%\usepackage[koi8-r]{inputenc}
%\usepackage[russianb]{babel}
\usepackage{vmargin}

\setpapersize{A4}
\usepackage[T2A]{fontenc}
\usepackage[utf8x]{inputenc}
\usepackage[english, russian]{babel}
\setmarginsrb{2cm}{2cm}{2cm}{2cm}{0pt}{5mm}{0pt}{0mm}
\usepackage{indentfirst}
\usepackage{nicefrac} % For comparison
%\usepackage{xfrac}    % Works better with other fonts
%\usepackage[unicode, pdftex]{hyperref}
\usepackage{lettrine}
\usepackage[usenames]{color}
\usepackage{colortbl}
\usepackage{mathtext}
\usepackage{epigraph}
\usepackage{amsmath, amsfonts, amssymb, mathrsfs}
%\usepackage{mathptmx}
%\usepackage{txfonts}
\usepackage{pxfonts}
\usepackage[pagestyles]{titlesec}
\usepackage{ebgaramond}
\usepackage{awesomebox}
\usepackage{enumitem}
\usepackage{makeidx}
\makeindex
    \usepackage{etoolbox}
\makeatletter
\newlength\epitextskip
\pretocmd{\@epitext}{\em}{}{}
\apptocmd{\@epitext}{\em}{}{}
\patchcmd{\epigraph}{\@epitext{#1}\\}{\@epitext{#1}\\[\epitextskip]}{}{}
\makeatother
%running fraction with slash - requires math mode.
\newcommand*\rfrac[2]{{}^{#1}\!/_{#2}}

\DeclareSymbolFont{Xlargesymbols}{OMX}{cmex}{m}{n}
\DeclareMathSymbol{\Xsum}{\mathop}{Xlargesymbols}{80}

\setlength\epigraphrule{0pt}
\setlength\epitextskip{2ex}
\setlength\epigraphwidth{.8\textwidth}

\usepackage{xfrac}    % Works better with other fonts
\usepackage[colorlinks=true,linkcolor=black,urlcolor=black,bookmarksopen=true]{hyperref}

\usepackage{fancyhdr} % пакет для установки колонтитулов
\pagestyle{fancy} % смена стиля оформления страниц
\fancyhf{} % очистка текущих значений
\fancyhead[C]{\thepage} % установка верхнего колонтитула
\renewcommand{\headrulewidth}{0pt} % убрать разделительную линию


% Настройка вертикальных и горизонтальных отступов
\titlespacing{\chapter}{0pt}{5pt}{5pt}
\titlespacing{\section}{\parindent}{4mm}{4mm}
\titlespacing{\subsection}{\parindent}{3mm}{3mm}


%
%\renewcommand{\tabcolsep}{1cm}   %% increase table column spacing

% Настройка задачи со зведочкой
\newcounter{namedlistcounter}  % number the items
\newenvironment{withdot}
{\begin{list}
		{\arabic{namedlistcounter}*.} % labeling 
		{\usecounter{namedlistcounter}   % set counter
			\setlength{\leftmargin}{3em}} % set spacing 
	}
	{\end{list}}


\newcommand{\anonsection}[1]{ \section*{#1} \addcontentsline{toc}{section}{\numberline {}#1}} 

\makeatletter %%%%% <---- Starting chapter without a pagebreak
\renewcommand\chapter{\par%
	\thispagestyle{plain}% \global\@topnum\z@
	\@afterindentfalse \secdef\@chapter\@schapter}
\makeatother %%%%% <---- Starting chapter without a pagebreak
\titleformat{\chapter}[display]
{\normalfont\bfseries}{}{0pt}{\Large}

\newpagestyle{mystyle}{
	\sethead[\thepage][][]{}{}{\thepage}	
}

\renewcommand{\rmdefault}{cmr}


\pagestyle{mystyle}
\sloppy

\begin{document}

	
	\begin{titlepage}
\begin{center}

			%\vfill
			
			%\vfill
			\topskip 0pt
		%	\vspace*{\fill}

		{\large	А. Н. КОЛМОГОРОВ

			\ \\
\ \\
			\ \\
\ \\
			\vfill
			\ \\
			\ \\
			{\Huge\bf О ПРОФЕССИИ МАТЕМАТИКА}
			\ \\
				\ \\
			\textit{(Издание третье, дополненное)}


			\vspace*{\fill}    
			
			\vfill
			
И З Д А Т Е Л Ь С Т В О

МОСКОВСКОГО УНИВЕРСИТЕТА

1960
}
\end{center}
	\end{titlepage}
	
	\thispagestyle{empty} % выключаем отображение номера для этой страницы

	\newpage
	\begin{center}
		
			\ \\
\ \\
\ \\
			\ \\
\ \\
\ \\
	Печатается по постановлению
	
	Редакционно-издательского совета
	
	Московсого университета
\end{center}
	
	\newpage
	
\setcounter{secnumdepth}{0}  
	\ \\
	\ \\
	\section[1. За многочисленное и талантливое пополнение кадров советских математиков]{\center 1. ЗА МНОГОЧИСЛЕННОЕ И ТАЛАНТЛИВОЕ ПОПОЛНЕНИЕ КАДРОВ СОВЕТСКИХ МАТЕМАТИКОВ}

Значение математических методов в таких науках, как механика, физика или астрономия, хорошо известно. Также
всем известно и то, что математика необходима в практической работе инженеров и техников. Элементарные знания по геометрии или умение пользоваться буквенными формулами необходимы почти каждому мастеру или квалифицированному рабочему. Но менее ясным для многих является вопрос о том, что значит иметь специальность математика и заниматься самой математикой в качестве основной профессии.

Очень многие представляют себе дело так, что в учебниках и математических справочниках собрано уже вполне достаточно формул и правил для решения всевозможных, встречающихся на практике математических задач. Даже очень образованные люди часто спрашивают с недоумением: разве в математике можно сделать что-либо новое?

Поэтому и математика иногда представляют себе как скучного человека, выучившего большое число формул.и теорем, и считают, что его задача состоит в том, чтобы заученные готовые знания передать другим.

Во всем этом верно только то, что математические сведения, сообщаемые в средней школе и на первых ступенях
изучения математики в высшей школе, добыты человенеством давно. Но даже и эти простейшие математические сведения могут применяться умело и с пользой только в том случае, если они усвоены творчески, так, что учащийся видит сам, как можно было бы прийти к ним самостоятельно. От преподавателя математики и в высшей и средней школе требуется не только твердое знание преподаваемой им науки. Хорошо преподавать математику может только человек, который сам ею увлечен и воспринимает ее как живую, развивающуюся
науку. Вероятно, многие учащиеся средней школы знают, насколько увлекательной, а благодаря этому легкой и доступнои становится математика у таких преподавателей.

Еще в большей степени самостоятельность и способность по-новому подойти к математической формулировке задачи
необходимы тому, кто применяет математику в решении технических проблем. Это относится к работе каждого инженера. Но так как требующиеся при этом математические знания и способности имеются не у всех, то болыцинство наших научно-исследовательских технических институтов и даже некоторые крупные заводы стали усиленно привлекать специалистов-математиков для работы вместе с инженерами над техническими проблемами.

Математики, способные руководить большими вычислительными работами, особенно дефицитны. В настоящее время
имеется много задач, в которых для получения числового результата требуются вычисления, превосходящие возможности одного человека. Расчет упругих напряжений в плотинах, фильтрации воды под плотинами, сопротивлений, испытываемых самолетами при полете, или траекторий снарядов — вот типичные примеры таких задач.

Уже давно при научных институтах, проектных организациях и заводах, нуждающихся в решении подобных задач,
стали возникать вычислительные бюро со многими десятками вычислителей, оборудованные арифмометрами и вычислительными автоматами, требующими для выполнения арифметических действий над многозначными числами лишь набора их при помощи клавиш и нажатия соответствующей кнопки ($+$, $-$, $\times$, $\div$). Однако современные наука и техника сталкиваются с такими задачами, которые при этом уровне организации вычислительных работ требуют многих месяцев, а иногда и лет работы десятков вычислителей. Такое положение вызвало бурное развитие современной «машинной математики», о которой рассказывается в этой брошюре.

Конструирование и обслуживание современных вычислительных машин превратились в большие инженерные специальности, для которых специалисты готовятся на соответствующих отделениях технических вузов. Для работы же вычислителя в вычислительном бюро старого типа или для введения задачи в современную электронную вычислительную машину достаточно среднего общего образования и полугодичного производственного обучения. Для того чтобы довести решение математических задач до передачи для получения численных результатов вычислительному бюро или вычислительной машине необходимо большое количество людей с глубокими математическими знаниями.

Теория «вычислительных методов» математики развилась сейчас в большую науку и потребность в специалистах, владеющих этими методами, с развитием «машинной математики» возрастает. Перед нами возникают своеобразные задачи «программирования», т. е. приведения процесса вычислений к виду, допускающему полную автоматизацию решения на машинах задач определенного типа.

Ошибочным является представление о математике как о науке законченной, раз навсегда построенной в своих теоретических основах. В действительности математика обогащается совершенно новыми теориями и перестраивается в ответ на новые запросы механики (нелинейные колебания, механика сверхзвуковых скоростей), физики (математические методы квантовой физики) и других смежных наук. Кроме того, и в недрах самой математики после накопления большого числа разрозненных специальных задач, решенных частными приемами, создаются новые общие теории, освешающие эти задачи с иных точек зрения и позволяющие решать их однообразными методами. Например, методы возникающего на наших глазах «функционального анализа» относятся к математическому анализу (который был создан еще в XVII—XVIII вв. и преподается во всех высших технических заведениях) примерно так, как относится алгебра к арифметике. Так называемые «операторные методы» функционального анализа уже нашли широкое применение в современной физике и технике.

Советскому Союзу сейчас требуется большое количество самостоятельных исследователей по теоретическим вопросам математики. При сравнении изданных обзоров успехов советской математики за 1917—1947 гг. обнаруживается, что в первом пятнадцатилетии было около двухсот математиков, внесших в математическую науку что-либо существенно новое, во втором же пятнадцатилетии — 600—800.

Количество математиков с университетской подготовкой, требующихся для работы над задачами, выдвигаемыми естествознанием и техникой, значительно больше, особенно если учесть, что, кроме теоретической разработки вопроса, здесь, как правило, необходимо проведение больших расчетных работ. Постоянно возрастает ежегодная потребность научных и научно-технических институтов и «вычислительных центров» в молодых сотрудниках-математиках, выпущенных университетами.

Если учесть еще потребность нашей страны в преподавателях математики в педагогических и учительских институтах, то станет понятным, почему Советскому государству требуется так много математиков самой высокой квалификации, подготовляемых на механико-математических и физико-математических факультетах университетов.

За последние годы в нашей стране проведены важные мероприятия, направленные на повышение квалификации преподавателей математики высших учебных заведений, на привлечение в университеты большого числа молодежи, имеющей склонность к математике.

Интересно в связи с этим вспомнить, что в первые годы после Великой Октябрьской социалистической революции молодежь стремилась почти исключительно в высшие технические учебные заведения. Многим молодым людям представлялось тогда, что только таким путем они примут непосрелдственное участие в социалистическом строительстве. В первые революционные годы такие настроения имели некоторое разумное основание. Но потом, когда развитие науки стало
насущнейшей с хозяйственной точки зрения потребностью нашей страны, необходимы были усилия, чтобы преодолеть недоверие части молодежи к перспективам, ожидающим ее при поступлении в университеты. Эти настроения теперь изжиты. Но в применении к математике, которая издали, даже среди других наук, представляется слишком сухой и отвлеченной, с ними приходится бороться еще и сейчас.

С 1952 г. прием на математические специальности университетов СССР значительно увеличен по сравнению с предыдущими годами. Очень важно, чтобы при этом расширенном приеме на математические специальности попала не только хорошо подготовленная, но и любящая математику молодежь{\footnote{В Московском, Ленинградском, Киевском, Саратовском и Томском государственных университетах имеются механико-математические (или математико-механические) факультеты с основными специальностями: математика, механика. В остальных университетах имеются физико-математические факультеты со специальностями. математика, физика, механика и астрономия.}}. Для этого необходимо, чтобы всюду на местах была создана возможность этим любителям математики определить свои склонности и оценить свои силы и возможности.

Чтобы сделать выбор вполне сознательно, полезно принять участие в работе математического кружка и в местной математической олимпиаде. Быть может, еще более полезно почитать соответствующую литературу{\footnote{См. список литературы в приложении.}} и попробовать свои силы в решении более трудных задач.


	\section[2. Несколько замечаний о характере работы математика-исследователя]{\center 2. НЕСКОЛЬКО ЗАМЕЧАНИЙ О ХАРАКТЕРЕ РАБОТЫ МАТЕМАТИКА-ИССЛЕДОВАТЕЛЯ}


Как и всякая наука, математика требует прежде всего твердого знания того, что по исследуемому вопросу уже сделано. Но не следует думать, что в математике труднее, чем в других науках, добраться до возможности делать что-либо новое. Опыт говорит скорее о другом: способные математики, как правило, начинают самостоятельные научные исследования очень рано. Если математические открытия, сделанные в 16- или 17-летнем возрасте, являются все же исключениями, собираемыми с особенной тщательностью в популярных книжках по истории математики, то начало серьезной научной работы в 19—20 лет на средних курсах университетов достаточно типично для биографий многих наших ученых{\footnote{Академик С. Л. Соболев в 1933 г. в возрасте 25 лет был уже избран в члены-корреспонденты АН СССР. В 1953 г. членом-корреспондентом АН СССР избран 25-летний математик комсомолец С. Н. Мергелян.}}.

	\newpage
	\tableofcontents
	
	\thispagestyle{empty} % 

	\newpage

\setcounter{secnumdepth}{0}  

\phantomsection	
	\begin{center}
				\ \\
	\ \\
	\ \\
	\ \\
	\textbf{Колмогоров Андрей Николаевич}
	\ \\
	О ПРОФЕССИИ МАТЕМАТИКА
		\ \\
			\ \\
	Редактор \textit{С. Ф. Кондрашкова}
	
Техн. редактор \textit{Г. И. Георгиева}
	\ \\
		\ \\
Сдано в производство 15. VI 1959 г.

Подписано к печати 14. III 60 г.

Л-90183	Формат бум. 60 $\times$ 92$\rfrac{1}{16}$


Печ. л. 2,0 Бум. д. 1,0 Уч-изд. д. 1,88

Заказ 783

Тираж 40 000 (2*15 000) Цена 55 к.

Издательство Московского

универсистета

Москва, Ленинские горы

Административный корпус

	\ \\
		\ \\

1 типография Издательства МГУ

Москва, Моховая, 9

	\end{center}
			\thispagestyle{empty} % выключаем отображение номера для этой страницы


	
\end{document}

\printindex
