
%\documentclass[oneside,final,14pt]{extreport}
\documentclass[12pt, a4paper, openany]{book}



\usepackage[left=1cm,right=1cm,top=2cm,bottom=2cm,bindingoffset=0cm]{geometry}
%\usepackage[koi8-r]{inputenc}
%\usepackage[russianb]{babel}
\usepackage{vmargin}

\setpapersize{A4}
\usepackage[T2A]{fontenc}
\usepackage[utf8x]{inputenc}
\usepackage[english, russian]{babel}
\setmarginsrb{2cm}{2cm}{2cm}{2cm}{0pt}{5mm}{0pt}{0mm}
\usepackage{indentfirst}
\usepackage{nicefrac} % For comparison
%\usepackage{xfrac}    % Works better with other fonts
%\usepackage[unicode, pdftex]{hyperref}
\usepackage{lettrine}
\usepackage[usenames]{color}
\usepackage{colortbl}
\usepackage{mathtext}
\usepackage{epigraph}
\usepackage{amsmath, amsfonts, amssymb, mathrsfs}
%\usepackage{mathptmx}
%\usepackage{txfonts}
\usepackage{pxfonts}
\usepackage[pagestyles]{titlesec}
\usepackage{ebgaramond}
\usepackage{awesomebox}
\usepackage{enumitem}
\usepackage{makeidx}
\makeindex
    \usepackage{etoolbox}
\makeatletter
\newlength\epitextskip
\pretocmd{\@epitext}{\em}{}{}
\apptocmd{\@epitext}{\em}{}{}
\patchcmd{\epigraph}{\@epitext{#1}\\}{\@epitext{#1}\\[\epitextskip]}{}{}
\makeatother
%running fraction with slash - requires math mode.
\newcommand*\rfrac[2]{{}^{#1}\!/_{#2}}

\DeclareSymbolFont{Xlargesymbols}{OMX}{cmex}{m}{n}
\DeclareMathSymbol{\Xsum}{\mathop}{Xlargesymbols}{80}

\setlength\epigraphrule{0pt}
\setlength\epitextskip{2ex}
\setlength\epigraphwidth{.8\textwidth}

\usepackage{xfrac}    % Works better with other fonts
\usepackage[colorlinks=true,linkcolor=black,urlcolor=black,bookmarksopen=true]{hyperref}

\usepackage{fancyhdr} % пакет для установки колонтитулов
\pagestyle{fancy} % смена стиля оформления страниц
\fancyhf{} % очистка текущих значений
\fancyhead[C]{\thepage} % установка верхнего колонтитула
\renewcommand{\headrulewidth}{0pt} % убрать разделительную линию


% Настройка вертикальных и горизонтальных отступов
\titlespacing{\chapter}{0pt}{5pt}{5pt}
\titlespacing{\section}{\parindent}{4mm}{4mm}
\titlespacing{\subsection}{\parindent}{3mm}{3mm}


%
%\renewcommand{\tabcolsep}{1cm}   %% increase table column spacing

% Настройка задачи со зведочкой
\newcounter{namedlistcounter}  % number the items
\newenvironment{withdot}
{\begin{list}
		{\arabic{namedlistcounter}*.} % labeling 
		{\usecounter{namedlistcounter}   % set counter
			\setlength{\leftmargin}{3em}} % set spacing 
	}
	{\end{list}}


\newcommand{\anonsection}[1]{ \section*{#1} \addcontentsline{toc}{section}{\numberline {}#1}} 

\makeatletter %%%%% <---- Starting chapter without a pagebreak
\renewcommand\chapter{\par%
	\thispagestyle{plain}% \global\@topnum\z@
	\@afterindentfalse \secdef\@chapter\@schapter}
\makeatother %%%%% <---- Starting chapter without a pagebreak
\titleformat{\chapter}[display]
{\normalfont\bfseries}{}{0pt}{\Large}

\newpagestyle{mystyle}{
	\sethead[\thepage][][]{}{}{\thepage}	
}

\renewcommand{\rmdefault}{cmr}


\pagestyle{mystyle}
\sloppy

\begin{document}

	
	\begin{titlepage}
\begin{center}

			%\vfill
			
			%\vfill
			\topskip 0pt
		%	\vspace*{\fill}

		{\large	А. Н. КОЛМОГОРОВ

			\ \\
\ \\
			\ \\
\ \\
			\vfill
			\ \\
			\ \\
			{\Huge\bf О ПРОФЕССИИ МАТЕМАТИКА}
			\ \\
				\ \\
			\textit{(Издание третье, дополнненное)}


			\vspace*{\fill}    
			
			\vfill
			
И З Д А Т Е Л Ь С Т В О

МОСКОВСКОГО УНИВЕРСИТЕТА

1960
}
\end{center}
	\end{titlepage}
	
	\thispagestyle{empty} % выключаем отображение номера для этой страницы

	\newpage
	\begin{center}
		
			\ \\
\ \\
\ \\
			\ \\
\ \\
\ \\
	Печатается по постановлению
	
	Редакционно-издательского совета
	
	Московсого университета
\end{center}
	
	\newpage
	
\setcounter{secnumdepth}{0}  
	\ \\
	\ \\
	\section[1. За многочисленное и талантливое пополнение кадров советских математиков]{\center 1. ЗА МНОГОЧИСЛЕННОЕ И ТАЛАНТЛИВОЕ ПОПОЛНЕНИЕ КАДРОВ СОВЕТСКИХ МАТЕМАТИКОВ}

Значение математических методов в таких науках, как механика, физика или астрономия, хорошо известно. Также
всем известно и то, что математика необходима в практической работе инженеров и техников. Элементарные знания по геометрии или умение пользоваться буквенными формулами необходимы почти каждому мастеру или квалифицированному рабочему. Но менее ясным для многих является вопрос о том, что значит иметь специальность математика и заниматься самой математикой в качестве основной профессии.

Очень многие представляют себе дело так, что в учебниках и математических справочниках собрано уже вполне достаточно формул и правил для решения всевозможных, встречающихся на практике математических задач. Даже очень образованные люди часто спрашивают с недоумением: разве в математике можно сделать что-либо новое?

Поэтому и математика иногда представляют себе как скучного человека, выучившего большое число формул.и теорем, и считают, что его задача состоит в том, чтобы заученные готовые знания передать другим.

Во всем этом верно только то, что математические сведения, сообщаемые в средней школе и на первых ступенях
изучения математики в высшей школе, добыты человенеством давно. Но даже и эти простейшие математические - сведения могут применяться умело и с пользой только в том случае, если они усвоены творчески, так, что учащийся видит сам, как можно было бы прийти к ним самостоятельно. От преподавателя математики и в высшей и средней школе требуется не только твердое знание преподаваемой им науки. Хорошо преподавать математику может только человек, который сам

	\newpage
	\tableofcontents
	
	\thispagestyle{empty} % 

	\newpage

\setcounter{secnumdepth}{0}  

\phantomsection	
	\begin{center}
				\ \\
	\ \\
	\ \\
	\ \\
	\textbf{Колмогоров Андрей Николаевич}
	\ \\
	О ПРОФЕССИИ МАТЕМАТИКА
		\ \\
			\ \\
	Редактор \textit{С. Ф. Кондрашкова}
	
Техн. редактор \textit{Г. И. Георгиева}
	\ \\
		\ \\
Сдано в производство 15. VI 1959 г.

Подписано к печати 14. III 60 г.

Л-90183	Формат бум. 60 $\times$ 92$\rfrac{1}{16}$


Печ. л. 2,0 Бум. д. 1,0 Уч-изд. д. 1,88

Заказ 783

Тираж 40 000 (2*15 000) Цена 55 к.

Издательство Московского

универсистета

Москва, Ленинские горы

Административный корпус

	\ \\
		\ \\

1 типография Издательства МГУ

Москва, Моховая, 9

	\end{center}
			\thispagestyle{empty} % выключаем отображение номера для этой страницы
	\newpage
	
	\setcounter{secnumdepth}{0}  
	
	\phantomsection
	
		\section*{Описание}
	
	{\bf Название:} О ПРОФЕССИИ МАТЕМАТИКА
	
{\bf Автор:} Колмогоров Андрей Николаевич
	
{\bf Издательство:} Издательство Московского универсистета, Москва, Ленинские горы, Административный корпус
	
		{\bf Редактор:}  \textit{С. Ф. Кондрашкова}
		
		{\bf Техн. редактор:}  \textit{Г. И. Георгиева}
	
		{\bf Аннотация:} Дается полное доказательство алгоритмической неразрешимости 10-й проблемы Гильберта, касающейся диофантовых уравнений, вместе с необходимыми сведениями из теории алгоритмов и теории чисел, а также приложения развитой для этого техники к другим массовым проблемам теории чисел, алгебры, анализа, теоретического программирования.
		
		Для математиков, в том числе аспирантов и студентов старших курсов.
		
		Библиогр. 247 назв.
		\thispagestyle{empty} % выключаем отображение номера для этой страницы

	
\end{document}

\printindex
