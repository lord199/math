
%\documentclass[oneside,final,14pt]{extreport}
\documentclass[12pt, a4paper, openany]{book}



\usepackage[left=1cm,right=1cm,top=2cm,bottom=2cm,bindingoffset=0cm]{geometry}
%\usepackage[koi8-r]{inputenc}
%\usepackage[russianb]{babel}
\usepackage{vmargin}

\setpapersize{A4}
\usepackage[T2A]{fontenc}
\usepackage[utf8x]{inputenc}
\usepackage[english, russian]{babel}
\setmarginsrb{2cm}{2cm}{2cm}{2cm}{0pt}{5mm}{0pt}{0mm}
\usepackage{indentfirst}
\usepackage{nicefrac} % For comparison
%\usepackage{xfrac}    % Works better with other fonts
%\usepackage[unicode, pdftex]{hyperref}
\usepackage{lettrine}
\usepackage[usenames]{color}
\usepackage{colortbl}
\usepackage{mathtext}
\usepackage{epigraph}
\usepackage{amsmath, amsfonts, amssymb, mathrsfs}
%\usepackage{mathptmx}
%\usepackage{txfonts}
\usepackage{pxfonts}
\usepackage[pagestyles]{titlesec}
\usepackage{ebgaramond}
\usepackage{awesomebox}
\usepackage{enumitem}
\usepackage{makeidx}
\usepackage[letterspace=150]{microtype}
\makeindex
    \usepackage{etoolbox}
\makeatletter
\newlength\epitextskip
\pretocmd{\@epitext}{\em}{}{}
\apptocmd{\@epitext}{\em}{}{}
\patchcmd{\epigraph}{\@epitext{#1}\\}{\@epitext{#1}\\[\epitextskip]}{}{}
\makeatother
%running fraction with slash - requires math mode.
\newcommand*\rfrac[2]{{}^{#1}\!/_{#2}}

\DeclareSymbolFont{Xlargesymbols}{OMX}{cmex}{m}{n}
\DeclareMathSymbol{\Xsum}{\mathop}{Xlargesymbols}{80}

\setlength\epigraphrule{0pt}
\setlength\epitextskip{2ex}
\setlength\epigraphwidth{.8\textwidth}

\usepackage{xfrac}    % Works better with other fonts
\usepackage[colorlinks=true,linkcolor=black,urlcolor=black,bookmarksopen=true]{hyperref}

\usepackage{fancyhdr} % пакет для установки колонтитулов
\pagestyle{fancy} % смена стиля оформления страниц
\fancyhf{} % очистка текущих значений
\fancyhead[C]{\thepage} % установка верхнего колонтитула
\renewcommand{\headrulewidth}{0pt} % убрать разделительную линию


% Настройка вертикальных и горизонтальных отступов
\titlespacing{\chapter}{0pt}{5pt}{5pt}
\titlespacing{\section}{\parindent}{4mm}{4mm}
\titlespacing{\subsection}{\parindent}{3mm}{3mm}


%
%\renewcommand{\tabcolsep}{1cm}   %% increase table column spacing

% Настройка задачи со зведочкой
\newcounter{namedlistcounter}  % number the items
\newenvironment{withdot}
{\begin{list}
		{\arabic{namedlistcounter}*.} % labeling 
		{\usecounter{namedlistcounter}   % set counter
			\setlength{\leftmargin}{3em}} % set spacing 
	}
	{\end{list}}


\newcommand{\anonsection}[1]{ \section*{#1} \addcontentsline{toc}{section}{\numberline {}#1}} 

\makeatletter %%%%% <---- Starting chapter without a pagebreak
\renewcommand\chapter{\par%
	\thispagestyle{plain}% \global\@topnum\z@
	\@afterindentfalse \secdef\@chapter\@schapter}
\makeatother %%%%% <---- Starting chapter without a pagebreak
\titleformat{\chapter}[display]
{\normalfont\bfseries}{}{0pt}{\Large}

\newpagestyle{mystyle}{
	\sethead[\thepage][][]{}{}{\thepage}	
}

\renewcommand{\rmdefault}{cmr}


\pagestyle{mystyle}
\sloppy

\begin{document}

	
	\begin{titlepage}
\begin{center}

			%\vfill
			
			%\vfill
			\topskip 0pt
		%	\vspace*{\fill}

		{\large	\textls{А. Н. КОЛМОГОРОВ}

			\ \\
\ \\
			\ \\
\ \\
			\vfill
			\ \\
			\ \\
			{\Huge\bf О ПРОФЕССИИ МАТЕМАТИКА}
			\ \\
				\ \\
			\textit{(Издание третье, дополненное)}


			\vspace*{\fill}    
			
			\vfill
			
\textls{ИЗДАТЕЛЬСТВО}

МОСКОВСКОГО УНИВЕРСИТЕТА

1960
}
\end{center}
	\end{titlepage}
	
	\thispagestyle{empty} % выключаем отображение номера для этой страницы

	\newpage
	\begin{center}
		
			\ \\
\ \\
\ \\
			\ \\
\ \\
\ \\
	Печатается по постановлению
	
	Редакционно-издательского совета
	
	Московского университета
\end{center}
	
	\newpage
	
\setcounter{secnumdepth}{0}  
	\ \\
	\ \\
	\section[1. За многочисленное и талантливое пополнение кадров советских математиков]{\center 1. ЗА МНОГОЧИСЛЕННОЕ И ТАЛАНТЛИВОЕ ПОПОЛНЕНИЕ КАДРОВ СОВЕТСКИХ МАТЕМАТИКОВ}

Значение математических методов в таких науках, как механика, физика или астрономия, хорошо известно. Также
всем известно и то, что математика необходима в практической работе инженеров и техников. Элементарные знания по геометрии или умение пользоваться буквенными формулами необходимы почти каждому мастеру или квалифицированному рабочему. Но менее ясным для многих является вопрос о том, что значит иметь специальность математика и заниматься самой математикой в качестве основной профессии.

Очень многие представляют себе дело так, что в учебниках и математических справочниках собрано уже вполне достаточно формул и правил для решения всевозможных, встречающихся на практике математических задач. Даже очень образованные люди часто спрашивают с недоумением: разве в математике можно сделать что-либо новое?

Поэтому и математика иногда представляют себе как скучного человека, выучившего большое число формул.и теорем, и считают, что его задача состоит в том, чтобы заученные готовые знания передать другим.

Во всем этом верно только то, что математические сведения, сообщаемые в средней школе и на первых ступенях изучения математики в высшей школе, добыты человечеством давно. Но даже и эти простейшие математические сведения могут применяться умело и с пользой только в том случае, если они усвоены творчески, так, что учащийся видит сам, как можно было бы прийти к ним самостоятельно. От преподавателя математики и в высшей и средней школе требуется не только твердое знание преподаваемой им науки. Хорошо преподавать математику может только человек, который сам ею увлечен и воспринимает ее как живую, развивающуюся науку. Вероятно, многие учащиеся средней школы знают, насколько увлекательной, а благодаря этому легкой и доступной становится математика у таких преподавателей.

Еще в большей степени самостоятельность и способность по-новому подойти к математической формулировке задачи необходимы тому, кто применяет математику в решении технических проблем. Это относится к работе каждого инженера. Но так как требующиеся при этом математические знания и способности имеются не у всех, то большинство наших научно-исследовательских технических институтов и даже некоторые крупные заводы стали усиленно привлекать специалистов-математиков для работы вместе с инженерами над техническими проблемами.

Математики, способные руководить большими вычислительными работами, особенно дефицитны. В настоящее время имеется много задач, в которых для получения числового результата требуются вычисления, превосходящие возможности одного человека. Расчет упругих напряжений в плотинах, фильтрации воды под плотинами, сопротивлений, испытываемых самолетами при полете, или траекторий снарядов — вот типичные примеры таких задач.

Уже давно при научных институтах, проектных организациях и заводах, нуждающихся в решении подобных задач,
стали возникать вычислительные бюро со многими десятками вычислителей, оборудованные арифмометрами и вычислительными автоматами, требующими для выполнения арифметических действий над многозначными числами лишь набора их при помощи клавиш и нажатия соответствующей кнопки ($+$, $-$, $\times$, $\div$). Однако современные наука и техника сталкиваются с такими задачами, которые при этом уровне организации вычислительных работ требуют многих месяцев, а иногда и лет работы десятков вычислителей. Такое положение вызвало бурное развитие современной «машинной математики», о которой рассказывается в этой брошюре.

Конструирование и обслуживание современных вычислительных машин превратились в большие инженерные специальности, для которых специалисты готовятся на соответствующих отделениях технических вузов. Для работы же вычислителя в вычислительном бюро старого типа или для введения задачи в современную электронную вычислительную машину достаточно среднего общего образования и полугодичного производственного обучения. Для того чтобы довести решение математических задач до передачи для получения численных результатов вычислительному бюро или вычислительной машине необходимо большое количество людей с глубокими математическими знаниями.

Теория «вычислительных методов» математики развилась сейчас в большую науку и потребность в специалистах, владеющих этими методами, с развитием «машинной математики» возрастает. Перед нами возникают своеобразные задачи «программирования», т. е. приведения процесса вычислений к виду, допускающему полную автоматизацию решения на машинах задач определенного типа.

Ошибочным является представление о математике как о науке законченной, раз навсегда построенной в своих теоретических основах. В действительности математика обогащается совершенно новыми теориями и перестраивается в ответ на новые запросы механики (нелинейные колебания, механика сверхзвуковых скоростей), физики (математические методы квантовой физики) и других смежных наук. Кроме того, и в недрах самой математики после накопления большого числа разрозненных специальных задач, решенных частными приемами, создаются новые общие теории, освещающие эти задачи с иных точек зрения и позволяющие решать их однообразными методами. Например, методы возникающего на наших глазах «функционального анализа» относятся к математическому анализу (который был создан еще в XVII—XVIII вв. и преподается во всех высших технических заведениях) примерно так, как относится алгебра к арифметике. Так называемые «операторные методы» функционального анализа уже нашли широкое применение в современной физике и технике.

Советскому Союзу сейчас требуется большое количество самостоятельных исследователей по теоретическим вопросам математики. При сравнении изданных обзоров успехов советской математики за 1917—1947 гг. обнаруживается, что в первом пятнадцатилетии было около двухсот математиков, внесших в математическую науку что-либо существенно новое, во втором же пятнадцатилетии — 600—800.

Количество математиков с университетской подготовкой, требующихся для работы над задачами, выдвигаемыми естествознанием и техникой, значительно больше, особенно если учесть, что, кроме теоретической разработки вопроса, здесь, как правило, необходимо проведение больших расчетных работ. Постоянно возрастает ежегодная потребность научных и научно-технических институтов и «вычислительных центров» в молодых сотрудниках-математиках, выпущенных университетами.

Если учесть еще потребность нашей страны в преподавателях математики в педагогических и учительских институтах, то станет понятным, почему Советскому государству требуется так много математиков самой высокой квалификации, подготовляемых на механико-математических и физико-математических факультетах университетов.

За последние годы в нашей стране проведены важные мероприятия, направленные на повышение квалификации преподавателей математики высших учебных заведений, на привлечение в университеты большого числа молодежи, имеющей склонность к математике.

Интересно в связи с этим вспомнить, что в первые годы после Великой Октябрьской социалистической революции молодежь стремилась почти исключительно в высшие технические учебные заведения. Многим молодым людям представлялось тогда, что только таким путем они примут непосредственное участие в социалистическом строительстве. В первые революционные годы такие настроения имели некоторое разумное основание. Но потом, когда развитие науки стало насущнейшей с хозяйственной точки зрения потребностью нашей страны, необходимы были усилия, чтобы преодолеть недоверие части молодежи к перспективам, ожидающим ее при поступлении в университеты. Эти настроения теперь изжиты. Но в применении к математике, которая издали, даже среди других наук, представляется слишком сухой и отвлеченной, с ними приходится бороться еще и сейчас.

С 1952 г. прием на математические специальности университетов СССР значительно увеличен по сравнению с предыдущими годами. Очень важно, чтобы при этом расширенном приеме на математические специальности попала не только хорошо подготовленная, но и любящая математику молодежь{\footnote{В Московском, Ленинградском, Киевском, Саратовском и Томском государственных университетах имеются механико-математические (или математико-механические) факультеты с основными специальностями: математика, механика. В остальных университетах имеются физико-математические факультеты со специальностями. математика, физика, механика и астрономия.}}. Для этого необходимо, чтобы всюду на местах была создана возможность этим любителям математики определить свои склонности и оценить свои силы и возможности.

Чтобы сделать выбор вполне сознательно, полезно принять участие в работе математического кружка и в местной математической олимпиаде. Быть может, еще более полезно почитать соответствующую литературу{\footnote{См. список литературы в приложении.}} и попробовать свои силы в решении более трудных задач.


	\section[2. Несколько замечаний о характере работы математика-исследователя]{\center 2. НЕСКОЛЬКО ЗАМЕЧАНИЙ О ХАРАКТЕРЕ РАБОТЫ МАТЕМАТИКА-ИССЛЕДОВАТЕЛЯ}


Как и всякая наука, математика требует прежде всего твердого знания того, что по исследуемому вопросу уже сделано. Но не следует думать, что в математике труднее, чем в других науках, добраться до возможности делать что-либо новое. Опыт говорит скорее о другом: способные математики, как правило, начинают самостоятельные научные исследования очень рано. Если математические открытия, сделанные в 16- или 17-летнем возрасте, являются все же исключениями, собираемыми с особенной тщательностью в популярных книжках по истории математики, то начало серьезной научной работы в 19—20 лет на средних курсах университетов достаточно типично для биографий многих наших ученых{\footnote{Академик С. Л. Соболев в 1933 г. в возрасте 25 лет был уже избран в члены-корреспонденты АН СССР. В 1953 г. членом-корреспондентом АН СССР избран 25-летний математик комсомолец С. Н. Мергелян.}}.

Конечно, широта постановки задач приходит обычно несколько позднее, но при решении отчетливо поставленных трудных конкретных задач совсем молодые люди часто с успехом соревнуются со сложившимися известными учеными. Ежегодно около десятка научных работ, выполненных студентами математических специальностей Московского университета, публикуется в таком издании, как Доклады Академии наук СССР.

В основе большинства математических открытий лежит какая-либо простая идея: наглядное геометрическое построение, новое элементарное неравенство и т. п. Нужно только применить надлежащим образом эту простую идею к решению задачи, которая с первого взгляда кажется недоступной. Много примеров этого можно найти и в популярной литературе, указанной в конце нашей брошюры. Поэтому вовсе не существует непроходимой стены между самыми новыми и трудными оригинальными математическими исследованиями и решением задач, доступных способному и достаточно упорному начинающему математику. Интересно с этой точки зрения прочесть некоторые главы из «Математической автобиографии» знаменитого советского алгебраиста Н. Г. Чеботарева{\footnote{Опубликована в журнале «Успехи математических наук», т. III, вып. 3, 1948.}}, где автор излагает историю своих научных поисков, начиная с первых опытов гимназиста до крупнейших открытий в алгебре.

Другое замечание относится к работе математиков над вопросами естествознания (механики, физики и техники). Сейчас, когда сотрудничество между математиками и представителями смежных специальностей развивается особенно широко, можно определенно сказать, что наиболее успешным оно оказывается при условии, если математик не ограничивается ролью исполнителя сделанного ему «заказа», а старается проникнуть в существо естественнонаучных и технических проблем. По существу здесь речь идет о том, что специалисты по математической и теоретической физике, теоретической механике или теоретической геофизике могут подготавливаться двумя путями: начинать свое образование с изучения физики, механики или геофизики, или же сначала изучать математику на математических отделениях университетов и потом основательно входить в ту или иную область применения математики.

Существует даже такая точка зрения, что второй путь дает лучшие результаты, т. е. что изучить на солидной математической основе аэромеханику, газовую динамику, сейсмологию или динамическую метеорологию легче, чем специалисту в какой-либо из этих областей восполнить недостаток математической подготовки. Такое мнение можно считать слишком крайним и заметить, например, что хорошее владение экспериментальной техникой встречается у математиков, перешедших на работу в какой-либо смежной области, лишь как редкое исключение. Но нельзя не признать, что из математиков по образованию произошел ряд крупнейших наших специалистов в смежных науках.

Трудно отделить математику от механики и сейсмологии в работах академиков М. А. Лаврентьева и С. Л. Соболева. В первую очередь как механики известны академики М. В. Келдыш, Л. И. Седов и чл.-кор. АН СССР Л. Н. Сретенский; как геофизики — члены-корреспонденты АН СССР А. Н. Тихонов и А. М. Обухов; как специалист по теоретической физике — акад. Н. Н. Боголюбов. Между тем все они окончили университеты в качестве математиков.

Можно было бы указать много связанных с именами математиков конкретных достижений в естествознании и технике, которые оказались весьма существенными с непосредственно практической стороны.


	\section[3. О математических способностях]{\center 3. О МАТЕМАТИЧЕСКИХ СПОСОБНОСТЯХ}


Необходимость специальных способностей для изучения и понимания математики часто преувеличивают. Впечатление исключительной трудности математики иногда создается ее плохим, чрезмерно формальным изложением на уроке. Обычные средние человеческие способности вполне достаточны, чтобы при хорошем руководстве или по хорошим книгам не только усвоить математику, преподающуюся в средней школе, но и разобраться, например, в началах дифференциального и интегрального исчислений. Тем не менее, когда дело идет о выборе математики в качестве основной специальности, вполне естественно желание проверить математические способности, или, как говорят иногда, математическую «одаренность». Ведь несомненно, что разные люди воспринимают математические рассуждения, решают математические задачи или — на более высокой ступени — приходят к новым математическим открытиям с различной скоростью, легкостью и успехом. И, конечно, следует стремиться к тому, чтобы из миллионов нашей молодежи специалистами-математиками становились именно те, кто в этой области будет работать наиболее успешно.

Поэтому содействие выдвижению математически одаренной молодежи является одной из важных задач школьных математических кружков, математических олимпиад и других мероприятий по пропаганде математических знаний и распространению интереса к самостоятельным занятиям математикой: Не следует спешить с чрезмерно ранним созданием для отдельных молодых людей репутации математических «талантов». Но вовремя подтолкнуть советом или премированием на олимпиаде способных математиков в сторону выбора математики в качестве своей дальнейшей работы необходимо.

В чем же заключаются эти способности? Следует прежде всего подчеркнуть, что успех в математике меньше всего основан на механическом запоминании большого числа фактов, отдельных формул и т. п. Хорошая память в математике, как и во всяком другом деле, является полезной, но никакой особенной, выдающейся памятью большинство крупных ученых-математиков не обладало.

В частности, фокусники, запоминающие длинные ряды многозначных чисел и складывающие или перемножающие их в уме, совсем не могут служить примером людей с хорошими математическими способностями в серьезном смысле слова.

Умение производить алгебраические вычисления, в смысле умелого преобразования сложных буквенных выражений, нахождения удачных путей для решения уравнений, не подходящих под стандартные правила, и т. п., уже ближе соприкасается с теми способностями, которые часто требуются от математика в серьезной научной работе.

Принято даже думать, что исключительно большое развитие таких вычислительных, или иногда говорят «алгоритмических», способностей является характерным для одного из нескольких основных типов математической одаренности.

В школьной алгебре с трудностями, требующими для своего преодоления такого рода способностей, школьники прежде всего сталкиваются при разложении алгебраических выражений на множители. Среди задач, данных в приложении 3 к этой брошюре, задачи 1 и 2 дают представление о том, что иногда разложение очень простых выражений на множители требует большого остроумия.

Далее, основной областью применения этого рода способностей становится решение уравнений. Однако везде, где это возможно, математики стремятся сделать изучаемые ими проблемы геометрически наглядными. В средней школе достаточно ясно видно, насколько полезны графики для изучения свойств функций. Поэтому читатель не удивится утверждению, что геометрическое воображение, или, как говорят, «геометрическая интуиция», играет большую роль при исследовательской работе почти во всех разделах математики, даже самых отвлеченных.

В школе обычно с большим трудом дается наглядное представление пространственных фигур. Надо, например, быть уже очень хорошим математиком (по сравнению с обычным школьным уровнем), чтобы, закрыв глаза, без чертежа ясно представить себе, какой вид имеет пересечение поверхности куба с плоскостью, проходящей через центр куба и перпендикулярной одной из его диагоналей.

В задаче 4 (приложение 3) вся трудность заключается в том, чтобы наглядно понять, что за фигура получается при пересечении тетраэдров. При решении задач 5—7 тоже очень существенна геометрическая интуиция, хотя здесь уже больше остается и на долю твердого знания тех теорем, которые придется применить при доказательстве, и на долю умения логически рассуждать.

Искусство последовательного, правильно расчлененного логического рассуждения является также существенной стороной математических способностей.

В школе для развития этого искусства служит систематический курс геометрии с ее определениями, теоремами и доказательствами. Но часто наибольшую трудность для школьников в отношении понимания точного смысла сложной логической конструкции представляет принцип математической индукции, изучаемой в конце курса алгебры. Многие не в состоянии ясно увидеть реальное содержание этого принципа за нагромождением слов «если» и «то».

Понимание и умение правильно применять принцип математической индукции является хорошим критерием логической зрелости, которая совершенно необходима математику.

Умение последовательно, логически рассуждать в незнакомой обстановке приобретается с трудом. На математических школьных олимпиадах самые неожиданные трудности возникают именно при решении задач, в которых не предполагается никаких предварительных знаний из школьного курса, но требуется правильно уловить смысл вопроса и рассуждать последовательно. Уже такой шуточный вопрос затрудняет многих десятиклассников: в хвойном лесу 800 000 елеи и ни на одной из них не более 500 000 игл; доказать, что по крайней мере у двух елей число игл точно одинаково (сравните в приложении 3 задачу 8; в задачах 10—12 тоже главная трудность не в сложности, а в необычности тех способов рассуждения, которые требуется применить).

Различные стороны математических способностей встречаются в разных комбинациях. Уже исключительное развитие одной из них иногда позволяет приходить к неожиданным и замечательным открытиям, хотя чрезмерная односторонность, конечно, опасна. Само собой разумеется, что никакие способности не помогут без увлечения своим делом, без систематической повседневной работы.

Математические способности проявляются обычно довольно рано и требуют непрерывного упражнения. Полный отрыв от математики в течение нескольких лет после средней школы часто оказывается трудно поправимым. Работа чертежника, лаборанта, обращение с машиностроительными деталями, сборка радиоаппаратуры и т. п., по-видимому, содержат в себе много элементов, родственных с работой математика, например, в смысле развития пространственного воображения и функционального мышления. Соприкосновение на работе с современной техникой может пробудить более сознательный интерес к приложениям математики. Но мы очень советуем молодым людям, намеревающимся поступить на математическое отделение университета, проработав после школы несколько лет на производстве, заранее заниматься математикой и не только путем подготовки к вступительным экзаменам (для чего при всех университетах существуют специальные подготовительные курсы), но и путем участия в математических кружках и олимпиадах и самостоятельного чтения. Иначе никакие льготы для «производственников» при поступлении в вузы не помогут им во время работы в университете не отстать от своих товарищей, пришедших со свежими знаниями и увлечениями прямо из школы.

	\section[4. Математические кружки, олимпиады, самостоятельное чтение, подготовка к вступительным экзаменам в университеты]{\center 4. МАТЕМАТИЧЕСКИЕ КРУЖКИ, ОЛИМПИАДЫ, САМОСТОЯТЕЛЬНОЕ ЧТЕНИЕ, ПОДГОТОВКА К ВСТУПИТЕЛЬНЫМ ЭКЗАМЕНАМ В УНИВЕРСИТЕТЫ}



Преподавание в школе во время обязательных классных занятий рассчитано в основном на твердое усвоение математики всеми учащимися. Попробовать свои силы в решении более трудных задач, ближе познакомиться с тем, как наука справляется с решением более сложных математических проблем, и с тем, как математика применяется в естествознании и в технике, можно в математическом кружке. Такие кружки ведут преподаватели математики во многих школах, Силами университетов и педагогических институтов во многих городах организованы межшкольные математические кружки и систематическое чтение лекций для школьников по отдельным вопросам математики или ее истории.

Естественно, что все эти начинания, как и математические олимпиады, широко открыты и для работающей молодежи, интересующейся математикой.

Математические олимпиады, на которых предлагаются трудные задачи и «победителям» выдаются премии и похвальные отзывы, удаются там, где хорошо поставлена работа в кружках. Олимпиады должны проводиться для завершения работы, ведущейся в течение года, а не как изолированное праздничное мероприятие.

Задачи, предлагаемые в кружках и на олимпиадах, иногда носят искусственный и даже шуточный характер. В этом нет беды, если задачи подобраны так, что для их решения требуется серьезная работа мысли, похожая на ту, которая требуется от взрослого, самостоятельно работающего математика.

В докладах, читаемых в кружках их участниками, и в лекциях, читаемых учителями и преподавателями высшей школы, широко освещаются основные пути развития математической науки, значение математики для естествознания и техники. Конечно, очень хорошо, если удается в задачах, предлагаемых в кружках, дать принципиально важный или убедительный своей полезностью материал, но было бы напрасно требовать, чтобы таким условиям была подчинена вся та большая «тренировочная» работа молодого математика, которая достигается решением задач.

Независимо от участия в кружках можно заняться самостоятельным решением более трудных задач. Имеется несколько интересных сборников задач для любителей математики. Некоторые из них написаны так, что читатель, решая последовательно связанные друг с другом задачи, может живо представить себе пути развития довольно сложных математических теорий. Кроме таких задачников повышенного уровня, в приложении 4 к этой брошюре указано много вполне доступных книжек по отдельным вопросам математики. Некоторые из книжек, минуя, по возможности, технические трудности, вводят читателя в круг вопросов, служащих о и в настоящее время предметом еще не законченного научного исследования.

Занятия в кружках, слушание лекций и чтение дополнительной литературы не должны, конечно, отвлекать учащихся школ или подготовительных курсов от более элементарной обязательной учебной работы. Следует помнить, что для того, чтобы быть принятым в университет, прежде всего требуется твердое знание школьного курса и умение на основе этих знании четко и уверенно решать более обычные, так сказать, стандартные задачи.

В приложении 2 приводятся примеры типичных задач, предлагавшихся на экзаменах при поступлении на механико-математический факультет Московского государственного университета. Если сравнить эти задачи с предлагавшимися на олимпиадах, то можно заметить их существенное отличие. Для решения экзаменационных задач не требуется какой-либо особой изобретательности. В большинстве случаев задачи решаются последовательным применением изучаемых в школе правил и приемов. Если же их решение и требует некоторой самостоятельности мысли, то. дело ограничивается необходимостью систематически исследовать поставленный вопрос в самом естественном направлении. Например, приступая к решению задачи 5 (приложение 2), следует ясно представить себе, как должен быть расположен в шестиугольнике искомый квадрат для того, чтобы его нельзя было увеличить, не выходя за пределы шестиугольника. Ясно, что для этого он должен упираться в периметр шестиугольника по меньшей мере двумя вершинами. Такие квадраты (у которых по меньшей мере две вершины лежат на периметре шестиугольника) надо подвергнуть более детальному исследованию. К сожалению, некоторые экзаменующиеся не могли преодолеть уже этого первого этапа и даже предлагали в качестве решения задачи чертежи, в которых квадрат свободно висел внутри шестиугольника, не прикасаясь к его периметру.
	
	Иногда экзаменующимся с целью проверить на решении одной задачи их знание целого ряда формул, правил и теорем школьного курса экзаменаторы предлагают задачи со сложными формулировками условий, придавая им весьма. искусственный и запутанный вид. Независимо от вопроса о правильности такой тенденции не следует чрезмерно бояться задач этого рода. По своей идее они обычно бывают даже значительно элементарнее задач с более короткими и красивыми формулировками. Все дело при решении таких комбинированных задач со сложно и запутанно формулируемыми условиями сводится обычно к тому, чтобы правильно прочесть условия задачи и не запутаться в длинном ряде выкладок и рассуждений, каждое звено которых вполне элементарно, хотя и требует применения ряда формул и теорем из школьного курса.
	
Очень важно правильно распределить свои силы между твердым усвоением школьного курса, серьезным продумыванием наиболее существенных и трудных с идейной стороны узловых вопросов этого курса, тренировкой в решении задач конкурсного типа и (при наличии для этого свободного времени) развитием своих более самостоятельных интересов путем дополнительного чтения, участия в кружках и олимпиадах.
	
Мне хочется в заключение заметить. что по наблюдениям многих преподавателей Московского университета сборники конкурсных задач и материалы школьных кружков и олимпиад поселили в некоторой части нашей молодежи чрезмерный страх перед поступлением в университет (и, в частности, в Московский). Для каждого поступающего естественно желание достигнуть того, чтобы уверенно решать любую задачу конкурсного типа, но не следует думать, что в университеты принимают только решивших все предложенные на экзаменах задачи,
	
В приложении 2 к этой брошюре приведены билеты предлагавшиеся на письменных экзаменах по математике поступающим на механико-математический факультет МГУ{\footnote{В настоящее время на мехмате производится два экзамена по математике — письменный и устный. До 1955 г. система была несколько сложнее: один устный экзамен и два письменных — по геометрии и по алгебре. В первых двух изданиях этой брошюры приведены образцы экз. билетов, соответствовавших этой более сложной системе, которая теперь отменена.}}. Билеты содержат по четыре задачи. Грубо говоря, оценки 5, 4, 3 соответствовали четырем, трем и двум решенным задачам. Решавшие менее двух задач, как правило, получали «двойку» и к дальнейшим экзаменам не допускались.
	
На устных экзаменах задача экзаменатора в советском вузе, вопреки распространенному воззрению школьников, состоит не в том, чтобы поскорее «срезать» незадачливого поступающего, а в том, чтобы тщательно взвесить, учитывая все обстоятельства экзаменационной обстановки, перспективы его дальнейшей работы по избранной им специальности. Нормы приема на первый курс наших вузов сейчас столь велики, что даже в Московском университете приемные и экзаменационные комиссии более всего озабочены тем, чтобы не потерять ни одного поступающего, достаточно подготовленного и способного серьезно работать на данном факультете. Между тем часто случается, что более боязливые молодые люди, подготовленные не хуже других, предпочитают подавать заявления не туда, куда им хочется попасть, а туда, где, по их сведениям, конкурс поменьше.

	\section[5. Элементарная и высшая математика]{\center 5. ЭЛЕМЕНТАРНАЯ И ВЫСШАЯ МАТЕМАТИКА}



\hangindent=6cm \hangafter=0  Поворотным пунктом в математике была декартова \textit{переменная величина}. Благодаря этому в математику вошли \textit{движение} и \textit{диалектика} и благодаря этому же стало \textit{немедленно необходимым дифференциальное и интегральное исчисление}.

\hangindent=6cm \hangafter=0  Лишь дифференциальное исчисление дает естествознанию возможность изображать математически не только \textit{состояния}, но и \textit{процессы} движение.

\hangindent=6cm \hangafter=0  Ф. Энгельс. Диалектика природы, 1950, стр. 206 и 218.

\

Отмечаемый в этих словах Энгельса поворот в математике произошел в ХVII в. одновременно с созданием основ математического естествознания. Значение этого поворота настолько велико, что до настоящего времени образовавшиеся в результате этого поворота разделы математики объединяют под названием «высшей математики», в отличие от сложившейся ранее «элементарной математики».

Некоторые основные понятия высшей математики вошли в настоящее время в программы средней школы, где основательно изучаются функциональные зависимости между переменными величинами и сообщаются некоторые сведения из теории пределов. Однако дифференциальное и интегральное исчисления, на которые опирается большинство наиболее серьезных и важных применений математики к естествознанию и технике, остаются за рамками программы средней школы.

Редко выбирают начала дифференциального и интегрального исчисления и в качестве предмета занятий в школьных математических кружках, так как в них обычно стремятся предлагать такой материал, который после сравнительно коротких вводных объяснений позволяет сразу взяться за самостоятельное решение задач: изучение же начал дифференциального и интегрального исчислений требует довольно длительной систематической работы. С другой стороны, сила и общность метода дифференциального и интегрального исчислений таковы, что, не ознакомившись с ними, нельзя как следует понять все значение математики для естествознания и техники и даже полностью оценить всю красоту и увлекательность самой математической науки. Например, в рамках элементарной математики нахождение и доказательство формул для объемов сколько-либо сложных фигур или площадей поверхностей представляется чрезвычайно трудным. Уже объем пирамиды, как известно, доставляет школьникам много мучений, Вывод формул объема конуса

$$V= \frac{1}{3}\pi R^2H,$$

\noindent объема пара

$$V= \frac{4}{3}\pi R^3 $$

\noindent или поверхности шара

$$ S=4\pi R^2$$

\noindent не менее сложен. Особенно неприятно то, что вывод каждой из этих формул требует своеобразных приемов и не дает представления о том, как справиться с задачами на нахождение площадей или объемов, не разобранных в учебнике геометрии. Но стоит познакомиться с началами интегрального исчисления, как обнаруживается, что, например, объемы всех тел вращения находятся при помощи интегрирования однообразным, простым и вполне естественным способом. При владении интегральным исчислением в принципе не представляют затруднений и любые другие задачи на определение площадей или объемов: все они делаются именно задачами, доступными решению определенным методом, в то время как в пределах элементарной математики каждая из приведенных выше формул являлась теоремой со своим собственным приемом доказательства.

Элементарные приемы решения задач на «максимум или минимум» являются сложной и весьма хитроумной наукой. Все это нагромождение своеобразных и тонких приемов оказывается в большинстве случаев совершенно излишним, если пользоваться дифференциальным исчислением. В этого рода применениях высшей математики разобраться еще проще, так как здесь требуется только ознакомиться с понятием производной, излагаемым в самом начале дифференциального исчисления, научиться вычислять производные простейших функций и ознакомиться с правилами применения производных к нахождению максимумов и минимумов.
\noindent Понятие производной{\footnote{См., например, статью «Дифференциальное исчисление». БСЭ. изд. 9, т. 14.}}
$$f^1(x)=\frac{dy}{dx}$$
\noindent от функции
$$ y = f(x)$$
\noindent по своему наглядному содержанию очень просто; если считать независимое переменное $x$ временем, то производная $f^1$ окажется просто скоростью изменения зависимого переменного $y$. Этим и объясняется основная роль понятия производной при изучении процессов изменения величин во времени,

Если обратиться к механике, то уже такая простая задача, как вывод найденного Галилеем закона падения тел, находит вполне удовлетворительное решение только при использовании средств высшей математики. Вывод хорошо известной формулы

$$ s = \frac{1}{2} gt^2 $$

\noindent для пути, пройденного тела за время $t$ при свободном падении в пустоте ($g$ — ускорение силы тяжести), который дается в элементарных учебниках физики, страдает некоторой сложностью и искусственностью. Но достаточно усвоить, что скорость есть производная от пройденного пути по времени

$$v=\frac{ds}{dt} $$

\noindent а ускорение — производная по времени от скорости

$$v=\frac{dv}{dt},$$

\noindent и ознакомиться с простейшими правилами интегрирования, как все сведется к очень простому вычислению:

$$ v = \int_{0}^{t} gdt = gt,$$

$$ s = \int_{0}^{t} v dt = \int_{0}^{t} gt dt = \frac{1}{2} gt^2.$$

В большом числе более сложных задач механики и физики основные законы течения изучаемых явлений тоже могут быть очень просто выражены при помощи уравнений, связывающих изучаемые величины с их производными по времени. Уравнения, связывающие искомые функции с их производными, называются дифференциальными.
	
При помощи дифференциальных уравнений сравнительно просто записываются законы движения небесных тел под действием всемирного тяготения, закономерности работы самых различных радиотехнических схем, закономерности распределения напряжений в различных механических конструкциях и т. д. Разработка методов решения таких уравнений и является одной из основных задач, которые естествознание и техника ставят математике.
		
Основательно изучить дифференциальное и интегральное исчисления до высшей школы довольно трудно. Еще труднее сколько-нибудь заметно продвинуться в теории решения дифференциальных уравнений. Однако тем, кого математика может увлечь и заинтересовать именно с этой стороны, можно все же попробовать ознакомиться с простейшими понятиями дифференциального и интегрального исчислений параллельно с окончанием средней школы. Во всяком случае таков был путь к математике многих наших ученых, причем для некоторых из них именно знакомство с высшей математикой и было решающим аргументом для того, чтобы окончательно остановиться на математике в качестве своей специальности. Можно с этой целью обратиться к указанной в приложении 4 к этой брошюре литературе (6, 7, 9) или сразу читать какой-либо из сравнительно доступных учебников для вузов.
		
Тому, кто не решится на такой труд, приведенные здесь краткие замечания помогут понять, насколько шире и интереснее, чем можно заранее себе представить, перспективы, которые откроются ему при дальнейшем изучении математики.


	\section[6. Современная машинная математика и кибернетика]{\center 6. СОВРЕМЕННАЯ МАШИННАЯ МАТЕМАТИКА И КИБЕРНЕТИКА}		
		
		
Современная математическая теория дает средства, в принципе достаточные, для решения самых разнообразных задач. Уже на первом курсе университета студенты знакомятся с методами нахождения с любой заданной точностью корней алгебраических уравнений какой угодно высокой степени. При изучении теории дифференциальных уравнений обнаруживается, что существуют общие методы нахождения их решений, хотя и приближенных, но тоже обладающих любой наперед заданной точностью.

Однако при практическом решении таких задач с целью получить определенный числовой результат обнаруживается, что обладать принципиальной схемой решения еще не достаточно. Например, при расчете траектории артиллерийского снаряда эта траектория разбивается на много десятков коротких отрезков, которые рассчитываются последовательно. Для расчета каждого следующего участка приходится проделать несколько десятков арифметических действий. Расчет одной траектории даже у вычислителя, пользующегося вспомогательными таблицами и арифмометром, занимает много часов и даже несколько дней.

Кораблестроительные расчеты или расчеты, связанные с постройкой плотин больших электростанций, занимают месяцы и даже годы работы специальных вычислительных бюро. Такое положение, естественно, привело к необходимости усовершенствовать машинную вычислительную технику. Прежде всего, наряду с обычными арифмометрами получили широкое распространение «малые вычислительные машины», выполняющие автоматически четыре арифметических действия над многозначными числами. Перемножение двух восьмизначных чисел занимает на такой машине 40 секунд.

При использовании этих машин вычислитель принужден еще записывать результаты каждого действия, потом вновь вводить их в машину. За последние двадцать лет широко развернулась работа по созданию «больших вычислительных машин», которые без вмешательства человека выполняют длинные ряды арифметических действий.

Программа работы такой машины задается пробитием дырочек на бумажной ленте. Машина сама выполняет в указанном порядке арифметические действия, фиксирует промежуточные результаты, использует их в дальнейших вычислениях и, наконец, выдает окончательный результат пробитым на ленте или карточках или даже отпечатанным.

Сначала в подобных сложных вычислительных машинах использовались механические элементы типа колесиков обычного арифмометра и электромагнитные реле, замыкающие и размыкающие ток, приводящий в движение элементы машины. Полный переворот в вычислительной технике произошел около Десяти лет назад, когда было показано, что возможно обойтись совсем без механического перемещения элементов машины, заменив их электронными лампами (диодами, триодами и т. д.) и их комбинациями (триггерами и т. п.) Благодаря этому стало возможным производить в одну секунду, например, по нескольку тысяч умножений многозначных чисел. Еще несколько позднее электронные лампы стали заменяться полупроводниковыми элементами, имеющими значительно меньшие размеры, для «запоминания» большого числа промежуточных данных (до нескольких сотен тысяч) были введены магнитные барабаны и т. д. (см., например, книгу А. И. Китова, указанную в приложении 4). Стало возможным делать вычисления, требующие, например, 20 миллионов операций для предсказания по данным метеорологических станций погоды на следующий день, вычислять траекторию снаряда за время, меньшее времени его полета, и т. д.

Большие вычислительные машины иногда специально строятся для какой-либо одной цели (например, для предсказания погоды), но чаще имеют универсальный характер, т. е. предназначаются для решения самых разнообразных задач. В этом случае они размещаются в «вычислительных центрах», обслуживающих различные научные и технические учреждения, не имеющие собственных больших вычислительных машин. Часто вычислительные машины подключаются к приборам, управляющим автоматически тем или иным процессом. Если управление быстро протекающим процессом
требует сложных вычислений, основанных на данных, получаемых в ходе этого процесса, то без скоростных вычислительных машин подобная задача была бы вообще неосуществима. Сфера применения таких управляющих машин быстро растет (см. указанную в приложении 4 книгу И. А. Полетаева).

Управляющие машины во многом походят на управляющие механизмы, возникшие естественным образом в ходе эволюции живых существ (нервная система, механизм сохранения и передачи по наследству признаков каждого вида животных и растений). Общие закономерности устройства управляющих систем изучаются недавно возникшими науками: теорией информации и кибернетикой, которые в значительной своей части являются математическими и предъявляют к чистой математике много новых запросов.
	\newpage
	
\addtocontents{toc}{\bigskip \textls{Приложения}:}

		\ \\
	\ \\
	\subsection[1. Математическое отделение Московского государственного университета имени М. В. Ломоносова]{\center \textls{ПРИЛОЖЕНИЕ 1}}		
	\subsection*{\center МАТЕМАТИЧЕСКОЕ ОТДЕЛЕНИЕ МОСКОВСКОГО ГОСУДАРСТВЕННОГО УНИВЕРСИТЕТА ИМЕНИ М. В. ЛОМОНОСОВА}			
	
Среди математических отделений советских университетов математическое отделение механико-математического факультета Московского университета является самым большим по нормам приема студентов и по составу профессоров. Все иногородние студенты, принятые в Московский университет, получают место в университетском общежитии. Ввиду общесоюзного значения Московского университета мы приводим здесь некоторые подробности о его математическом отделении.

Математическое отделение готовит научных работников в области математики и ее применений, а также преподавателей математики в высших учебных заведениях, техникумах и средних школах. Кроме того, в последние годы перед математическим отделением поставлена задача подготовки специалистов по эксплуатации современных сложных вычислительных машин.
	
Московский университет в подготовке основных научных квалифицированных кадров математиков занимает в СССР ведущее место. Его питомцами являются академики
\linebreak
П. С. Александров, М. В. Келдыш, А. Н. Колмогоров, М. А. Лаврентьев, И. Г. Петровский, члены-корреспонденты АН СССР и И. М. Гельфанд, Л. А. Люстерник, А. И. Люстерник, А. И. Мальцев, Д. Е. Менынов, А. М. Обухов, Л. С. Понтрягин, А. Н. Тихонов, \linebreak А. Я. Хинчин и многие другие известные математики. Можно сказать, что почти половина научной работы в области математики осуществляется в СССР математиками, получившими образование в Московском университете. В соответствии с этим значительная часть наиболее способных студентов, оканчивающих математическое отделение, направляется для продолжения научной работы по математике в аспирантуру университета или зачисляется аспирантами и младшими научными сотрудниками Математического института АН СССР и других подобных учреждений.
	
Еще больше в настоящее время потребность в математиках для работы в научных и научно-технических институтах, разрабатывающих вопросы смежных наук (физики, геофизики и т. п.) и различных областей современной техники. Работа математиков в таких институтах не ограничивается организацией вычислений (в вычислительных бюро и на машинных вычислительных станциях), или решением поставленных перед ними механиками, физиками или техниками математических проблем. Многие ученые получившие первоначальное математическое образование, становятся впоследствии первоклассными исследователями в области той или иной конкретной науки, требующей большой математической культуры (например, М. В. Келдыш, М. А. Лаврентьев и Л. П. Сретенский — специалисты в области механики, А. Н. Тихонов и А. М. Обухов — в области геофизики). Математическое отделение является одним из существенных каналов подготовки специалистов во всех областях науки и техники, которые для своего развития требуют наиболее современного математического аппарата.

В последние годы основная масса оканчивающих математическое отделение распределяется на работу именно в различные научные и научно-технические институты, имеющие потребность в специалистах-математиках, и в создаваемые сейчас по всей стране вычислительные центры. Если обратиться к выпускам прошлых лет, то другая значительная часть окончивших математическое отделение факультета работает преподавателями высшей школы (профессорами, доцентами и ассистентами). Некоторая часть оканчивающих и сейчас направляется в вузы ассистентами.

Меньшая часть оканчивающих отделение направляется для работы в средней школе. Следует, однако, отметить, что отделение тесно связано с работой в средней школе, имея в составе своих профессоров вице-президента Академии педагогических наук А. И. Маркушевича и действительных членов той же академии П. С. Александрова и А. Я. Хинчина. На факультете проводится большая работа со школьниками (кружки, олимпиады).

При переходе на третий курс студенты-математики выбирают одну из двух специальностей: «математика» или «вычислительная математика», Студенты специальности «математика» начиная с четвертого курса специализируются по предметам одной из следующих кафедр:

1. Анализ.

2. Высшая алгебра.

3. Высшая геометрия и топология.

4. Дифференциальная геометрия.

5. Теория вероятностей.

6. Теория чисел.

7. Дифференциальные уравнения.

8. Теория функций и функциональный анализ.

9. Математическая логика и история математических наук.


При кафедре высшей геометрии и топологии имеется кабинет номографии.

Студенты специальности «вычислительная математика» работают в лабораториях вычислительного центра МГУ, располагающих большой современной вычислительной машиной. Студенты этой специальности и специализации «теория вероятностей и математическая статистика» проходят производственную практику в различных научных и научно-технических учреждениях.

В учебных планах для математиков предусматривается на старших курсах, кроме чисто математических предметов, изучение теоретической физики и других естественнонаучных и технических дисциплин по выбору.

Математическое отделение предоставляет своим студентам самые широкие возможности для специализации во всех областях математики и ее применений. Кроме обязательных, ежегодно читается несколько Десятков специальных курсов, в которых излагаются последние достижения науки. Десятки научных семинаров и кружков объединяют математиков всех поколений, занятых решением стоящих перед математикой актуальных проблем. В эту исследовательскую работу самым широким образом втягиваются студенты старших, средних, а иногда и младших курсов.

	\subsection[2. Задачи, предлагавшиеся на письменных вступительных экзаменах на механико-математическом факультете Московского государственного университета]{\center \textls{ПРИЛОЖЕНИЕ 2}}					
\subsection*{\center ЗАДАЧИ, ПРЕДЛАГАВШИЕСЯ НА ПИСЬМЕННЫХ ВСТУПИТЕЛЬНЫХ ЭКЗАМЕНАХ НА МЕХАНИКО-МАТЕМАТИЧЕСКОМ ФАКУЛЬТЕТЕ МОСКОВСКОГО ГОСУДАРСТВЕННОГО УНИВЕРСИТЕТ\footnote{Все задачи, кроме последней, предлагались в 1958 г.}	}		

\begin{description}
	\item[I.] а) В бассейн проведены 4 трубы. Когда открыта 1-я, 2-я, 3-я трубы, бассейн наполняется за 12 минут; когда открыты 2-я, 3-я и 4-я трубы, — за 15 минут; когда открыты только 1-я и 4-я трубы, — за 20 минут, За какое время наполнится бассейн, если открыть все 4 трубы?
	
	б) Доказать, что при любом целом $n>0$ число $4^n+15n-1$ делится на $3$ и на $9$.
	
	в) Внутри угла $A$ дана точка $M$. Провести через $M$ прямую $l$ так, чтобы она отсекала от заданного угла треугольник наименьшей площади. Указать способ построения прямой $l$.
	
	г) В правильную треугольную усеченную пирамиду вписан шар радиуса $r$, касающийся всех 5 граней пирамиды; боковое ее ребро равно стороне меньшего основания. Найти объем этой усеченной пирамиды.
	
	\item[II.] а) Пять человек выполняют некоторую работу. 1-й, 2-й и 3-й, работая вместе, могут выполнить эту работу за 7,5 часа; 1-й, 3-й и 5-й — за 5 часов; 1-й, 3-й и 4-й, — за 6 часов: 2-й, 4-й, 5-й — за 4 часа. За какое время выполнят эту работу все 5 человек, работая вместе?
	
	б) Дано, что $\displaystyle\frac{1}{a}+\frac{1}{b}+\frac{1}{c} = \frac{1}{a+b+c}$. Доказать, что тогда сумма некоторых двух чисел из $a$, $b$, $c$ обязательно равна нулю.
	
	в) Точка $A$, $B$, $C$ и $D$ лежат на некоторой окружности (в порядке обхода против часовой стрелки). Найти геометрическое место точек касания окружностей, проходящих через $A$, $B$ и соответственно $C$ и $D$.
	
	г) Найти сторону куба, вписанного в правильную треугольную пирамиду, сторона основания которой равна $a$, а боковое ребро — $b$. Четыре вершины куба лежат на основании пирамиды, четыре другие — на боковых гранях.
	
	\item[III.] а) Сосуд снабжен 4 кранами. Если открыты все 4 крана, сосуд заполняется жидкостью за 4 часа, 1-й, 2-й и 4-й — за 5 часов, 2-й, 3-й и 4-й — за 6 часов. За какое время заполнят сосуд 1-й и 3-й краны?
	
	б) Найти такое трехзначное число $abc$, чтобы четырехзначные числа $abc$ 1 и 2 $abc$ удовлетворяли равенству
	
	$$abc1 = 3\cdot 2abc$$
	
	в) Дана прямоугольная трапеция с высотой $H$. На наклонной боковой стороне, как на диаметре, строится полуокружность и оказывается, что она касается вертикальной боковой стороны. Вычислить площадь прямоугольного треугольника с катетами, равными основаниям трапеции.
	
	г) От правильной треугольной призмы $ABCA^1B^1C^1$ плоскостью $A^1BC$ отрезана пирамида. В оставшееся тело вписан шар, касающийся всех его пяти граней. Радиус шара равен $r$. Найти объем призмы.
	
	\item[IV.]  а) Первый раствор содержит 6\% (по весу) вещества $A$, 16\% вещества $B$ и 4\% вещества $C$, второй раствор соответственно — 15, 9 и 10\%,
	третий — 3, 5 и 9\%. В каком отношении надо смешать эти растворы, чтобы получить раствор, содержащий 12\% вещества $A$, 10\% вещества $B$ и 8\% вещества $C$?
	
	6) Решить уравнение:

$$ \sin^2 x + \sin^2x \sin4x + \dots + \sin{nx} \sin^2{nx}=1.$$

	в) Стороны $a$, $b$ и $c$ треугольника лежат соответственно против углов $A$, $B$ и $C$. Доказать, что биссектриса угла $A$
	
	
	$$ b_a = \frac{2bc \cos{\displaystyle\frac{A}{2}}}{b+c}$$
	
	
	Пользуясь этой формулой, доказать, что треугольник с двумя равными биссектрисами равнобедренный.

	г) В треугольной пирамиде боковые ребра равны $a$, $b$ и $c$, а все плоские углы при вершине прямые. Найти сторону куба, вписанного в пирамиду так, что одна из его вершин совпадает с вершиной пирамиды, противоположная вершина лежит на основании.
	
	\item[V.]   а) Поместить. внутри правильного шестиугольника со стороной $a$ квадрат возможно б\'{о}льших размеров. Найти сторону этого квадрата.
	\end{description}
	
		\subsection[3. Из задач, предлагавшихся на математических олимпиадах]{\center \textls{ПРИЛОЖЕНИЕ 3}}				
	\subsection*{\center ИЗ ЗАДАЧ, ПРЕДЛАГАВШИХСЯ НА МАТЕМАТИЧЕСКИХ ОЛИМПИАДАХ	}		
	

\begin{enumerate}
	\item Разложить на множители: $x^5+x+1$. 
	
	\rightline{\textit{Ленинград, 1951, для 8-х классов.}}

	\item Разложить на множители: $a^{10}+a^5+1$. 
	
		\rightline{\textit{Львов, 1946, для 9-10-х классов.}}

\item Решить систему уравнений:
$$ xy (x+y)=30;$$
$$x^3+y^3=35.$$

		\rightline{\textit{Ленинград, 1951, для 9-х классов.}}

\item В куб вложены два правильных тетраэдра так, что четыре вершины куба служат вершинами одного из них, а остальные четыре вершины куба — вершинами другого. Какую долю объема куба составляет объем общей части этих тетраэдров?

		\rightline{\textit{Иваново, 1951, для 9—10-х классов.}}

\item Около сферы описан пространственный четырехугольник. Доказать, что точки касания лежат в одной плоскости.

		\rightline{\textit{Москва, 1950, для 9—10-х классов.}}

\item Доказать, что сумма расстояний от произвольной внутренней точки правильного тетраэдра до его граней есть величина постоянная.

		\rightline{\textit{Сталинград, 1950, для 10-х классов.}}

\item Доказать, что прямые, соединяющие середину высоты правильного тетраэдра с вершинами основания, взаимно-перпендикулярны.

		\rightline{\textit{Казань, 1947, для 9—10-х классов.}}

\item В 500 ящиках лежат яблоки. Известно, что ящик не может вместить более 240 яблок. Доказать, что по крайней мере 3 ящика содержат по одинаковому числу яблок.

\rightline{\textit{Киев, 1950, для 7—8-х классов.}}

\item Сколько нулей имеет число $50!=1\cdot2\cdot3\cdot4\cdot\dots\cdot49\cdot50$?

\rightline{\textit{Львов, 1950, для 7—8-х классов.}}


\item Сколько раз в сутки стрелки часов перпендикулярны друг другу?

\rightline{\textit{Киев, 1950, для 7—8-х классов.}}

\item Какое наибольшее число острых углов может иметь выпуклый многоугольник, имеющий $n$ сторон?

\rightline{\textit{Киев, 1950, для 7—8-х классов.}}

\item Доказать, что выпуклый 13-угольник нельзя разрезать на параллелограммы.

\rightline{\textit{Москва, 1947, для 7—8-х классов.}}

\item Доказать, что при любом целом $a$ число $a^7-a$ делится на 42.

\item Показать, что комплексное число $a+bi$ $(i=\sqrt{-1}$, $a$ и $b$ — действительны), модуль которого равен единице, причем $b\neq0$, можно представить в виде:

$$ a+bi = \frac{c+i}{c-i} ,$$

где $c$ - вещественное число. 

\rightline{\textit{Казань, 1947, для 10-х классов.}}


\item Числа 1, 2, 3... 101 (всего 101 число) расположены в ряд в каком-то неизвестном порядке. Показать, что из них можно вычеркнуть 90 чисел так, что оставшиеся 11 окажутся расположенными либо в порядке возрастания, либо в порядке убывания.

\rightline{\textit{Москва, 1950, для 9—10-х классов.}}

\item Доказать, что не существует многогранника, имеющего семь ребер.

\rightline{\textit{Киев, 1952, для 9—10-х классов.}}

\item Доказать, что $ \displaystyle\frac{n}{2}+\frac{n^2}{8}+\frac{n^3}{24} $ является целым числом при любом четном $n$.

\rightline{\textit{Республиканская олимпиада Литовской ССР, 1952, для 8—9-х классов.}}

\item Решить систему пятнадцати уравнений с пятнадцатью  неизвестными:

$$ 1-x_1x_2=0$$
$$ 1-x_2x_3=0$$
$$ \dots \dots \dots \dots $$
$$ 1-x_{14}x_{15}=0$$
$$ 1-x_{15}x_1=0$$

\rightline{\textit{Москва, 1952, для 7-х классов.}}

\item Доказать, что при $n>1$ имеют место неравенства

$$  2<(1+\frac{1}{n})^n<3. $$

\rightline{\textit{Орджоникидзе, 1958, для 10 классов.}}

\item Известно, что две смежные стороны параллелограмма равны $a$ и $b$. Найти отношение объемов фигур, получаемых при вращении параллелограмма вокруг этих сторон.

\rightline{\textit{Москва, 1958, для 10-х классов.}}

\item Решить в целых положительных числах уравнение:

$$ x^{2y}+(x+1)^{2y}=(x+2)^{2y} $$

\rightline{\textit{Москва, 1958, для 10-х классов.}}

\item Доказать, что при целом $n>2$ имеет место неравенство

$$(1\cdot2\cdot3\cdot\dots\cdot n)^2>n^n.$$

\rightline{\textit{Москва, 1958, для 8-х классов.}}

\item Из точки $O$ на плоскость проведено $n$ лучей. Попарно они образуют $\displaystyle\frac{n(n-1)}{2}$ углов (каждый раз берется угол, не превышающий 180°). Каково наибольшее возможное значение суммы этих углов?


\rightline{\textit{Москва, 1958, для 9-х классов.}}

\item Решить систему уравнений:

$$ \frac{2x^2}{1+x^2}=y, \ \ \  \frac{2y^2}{1+y^2}=z, \ \  \ \frac{2z^2}{1+z^2}=x.$$

\rightline{\textit{Москва, 1957, для 8-х классов.}}

\item Найти все действительные решения системы:

$$ 1- x^2_1=x_2 $$
$$ 1- x^2_2=x_3 $$
$$ \dots \dots \dots \dots $$
$$ 1- x^2_{n-1}=x_n $$
$$ 1- x^2_n=x_1 $$

\rightline{\textit{Москва, 1957, для 10-х классов.}}

\end{enumerate}

		\subsection[4. Список литературы по математике]{\center \textls{ПРИЛОЖЕНИЕ 4}}			
\subsection*{\center СПИСОК ЛИТЕРАТУРЫ ПО МАТЕМАТИКЕ}		

Доступная людям со средним образованием и учащимся старших классов средней школы литература по математике в настоящее время достаточно обширна. Трудность скорее состоит в том, чтобы выбрать себе из нее книги по вкусу и по силам.

Для интересующихся тренировкой в решении задач того типа, которые предлагаются на вступительных экзаменах в вузы, полезны сборники:

\begin{enumerate}
	\item \textls{Моденов} П. С. Сборник задач по математике, изд. 5. Изд-во «Советская наука», М., 1954. (В сборнике собраны задачи, предлагавшиеся на конкурсных экзаменах в Московском государственном университете. Особое внимание уделено разбору типичных ошибок, делаемых многими экзаменующимися. Сборник страдает некоторым уклоном в сторону искусственно запутанных задач).
	\item \textls{Антонов} Н. П., \textls{Выгодский} М. Я., \textls{Никитин} В. В., \textls{Санкин} А. И. Сборник задач по элементарной математике, изд. 4. Гостехиздат, М., 1958.
\end{enumerate}

	Следующие два сборника содержат по преимуществу трудные задачи, но они интереснее в смысле возможности на ряде задач познакомиться с дополнительным теоретическим материалом и общими методами решения задач.

 
\begin{enumerate}
		\setcounter{enumi}{2}
	\item  \textls{Делоне} Б. Н. и \textls{Житомирский} О. К. Задачник по геометрии, изд. 6. Гостехиздат, М.—Л., 1952. (Сборник содержит 506 геометрических задач, выясняющих те или иные существенные свойства геометрических фигур. Наборы задач по геометрии кругов, теории выпуклых многогранников, теории параллелоэдров и теории перспективы представляют собой введения в интересные специальные разделы  геометрии.)
	
	\item \textls{Кречмар} В. А. Задачник по алгебре, изд. 2. Гостехиздат, М.—Л., 1950. (Большой сборник сравнительно трудных алгебраических задач, включая задачи на обратные тригонометрические функции. Многие задачи приводят читателя к установлению интересных и находящих большое применение в математике соотношений, неравенств и т. п.)
	
	\item Библиотека школьного математического кружка, издается Гостехиздатом начиная с 1950 г. (Содержит задачи, предлагавшиеся на олимпиадах и в школьном математическом кружке при Московском государственном университете. Много очень трудных задач. Некоторые циклы задач представляют собой введение в сложные разделы современной математики. В частности, в последнем разделе выпуска 6 можно получить некоторые сведения о предмете и практических применениях теории вероятностей.)
	
	\textit{Выпуск 1.} \textls{Шклярский Д. О.} и др. Избранные задачи и теоремы элементарной математики, ч. 1. Арифметика и алгебра. 1950.
	
	\textit{Выпуск 2.} \textls{Шклярский} Д. О., \textls{Ченцов} Н. Н., \textls{Яглом} И. М. Избранные задачи и теоремы элементарной математики, ч. 2. Планиметрия. 1952.
	
	\textit{Выпуск 3.} \textls{Шклярский} Д. О., \textls{Ченцов} Н. Н., \textls{Яглом} И. М. Избранные задачи и теоремы элементарной математики, ч. 3. Стереометрия. 1954.
	
	\textit{Выпуск 4.} \textls{Яглом} И.М., \textls{Болтянский} В. Г. Выпуклые фигуры. 1951.
	
	\textit{Выпуск 5}. \textls{Яглом} А. М., \textls{Яглом} И. М. Неэлементарные задачи в элементарном изложении. 1954.
	
	\textit{Выпуск 6}. \textls{Дынкин} Е. Б., \textls{Успенский} В. А. Математические беседы. Задачи о многоцветной раскраске. Задачи из теории чисел. Случайные блуждания, 1952.
	
	\textit{Выпуск 7}. \textls{Яглом} И. М. Геометрические преобразования. Движения и преобразования подобия. [Т] 1, 1955.
	
	\textit{Выпуск 8}. \textls{Яглом} И. М. Геометрические преобразования, линейные и круговые преобразования. [Т] 2, 1956.
	
\end{enumerate}
	
	Книжки библиотеки школьного математического кружка Довольно трудны. Кроме того, они возникли в обстановке работы кружков и подготовки к олимпиаде, где центр тяжести лежит в развитии изобретательности в решении задач. По материалам этой библиотеки лишь с трудом можно получить представление об основных отделах математики и ее роли в естествознании и технике. Книжки библиотеки сознательно не касаются основ математического анализа (ср. раздел 5 этой брошюры).
	
	Желающие получить широкое понимание строения математики и ее места в естествознании и технике могут обратиться к первым главам
	монографий:
	
\begin{enumerate}
	\setcounter{enumi}{5}
	\item Математика, ее содержание, методы и значение. Изд-во АН СССР. М., 1956.
	
	\item \textls{Курант} Р. и \textls{Роббинс} Г. Что такое математика. Элементарный очерк идей и метолов. Пер. с англ. Гостехиздат, М.—Л., 1947.
	
	\item Со многими принципиально важными вопросами математики в доступной форме знакомят небольшие книжки серии Гостехиздата — популярные лекции по математике.
	
	\textit{Выпуск 1}. \textls{Маркушевич} А. И. Возвратные последовательности, 1951.
	
	\textit{Выпуск 2}. \textls{Натансон} И. П. Простейшие задачи на максимум и минимум, 1951.
	
	\textit{Выпуск 3}. \textls{Соминский} И. С. Метод математической индукции, 1951.
	
	\textit{Выпуск 4}. \textls{Маркушевич} А. И. Замечательные кривые, 1951.
	
	\textit{Выпуск 5}. \textls{Коровкин} П. П. Неравенства, 1951.
	
	\textit{Выпуск 6}. \textls{Воробьев} Н. Н. Числа фибоначчи, 1951.
	
	\textit{Выпуск 7}. \textls{Курош} А. Г. Алгебраические уравнения произвольных степеней, 1951.
	
	\textit{Выпуск 8}. \textls{Гельфонд} А. О. Решение уравнений в целых числах, 1956.
	\textit{Выпуск 9}. \textls{Маркушевич} А. И. Площади и логарифмы, 1955.
	\textit{Выпуск 10}. \textls{Смогоржевский} А. С. Метод координат, 1959.
	
	\textit{Выпуск 11}. \textls{Дубнов} Я. С. Ошибки в геометрических доказательствах, 1953.
	
	\textit{Выпуск 12}. \textls{Натансон} И. П. Суммирование бесконечно малых величин, 1953.
	
	\textit{Выпуск 13}. \textls{Маркушевич} А. И. Комплексные числа и конформные отображения, 1954.
	
	\textit{Выпуск 14}. \textls{Фетисов} А. И. О доказательствах в геометрии, 1954.
	
	\textit{Выпуск 15}. \textls{Шафаревич} И. Р. О решении уравнений высших степеней. 1954.
	
	\textit{Выпуск 16}. \textls{Шерватов} В. Г. Гиперболистические функции, 1954. 
	
	\textit{Выпуск 17}. \textls{Болтянский} В. Г. Что такое дифференцирование? 1955.
	
	\textit{Выпуск 18}. \textls{Миракьян} Г. М. Прямой круговой цилиндр, 1955.
	
	\textit{Выпуск 19}. \textls{Люстерник} Л. А. Кратчайшие линии, 1955.
	
	\textit{Выпуск 20}. \textls{Лопщиц} А. М. Вычисление площадей ориентированных фигур, 1956.
	
	\textit{Выпуск 21}. \textls{Головина} Л. И. и \textls{Яглом} И. М. Индукция в геометрии, 1956.
	
	\textit{Выпуск 22}. \textls{Болтянский} В. Г. Равновеликие и равносоставленные фигуры, 1956.
	
	\textit{Выпуск 23}. \textls{Смогоржевский} А. С. О геометрии Лобачевского, 1957.
	
	\textit{Выпуск 24}. \textls{Аргунов} Б. И. и \textls{Скорняков} Л. А. Конфигурационные теоремы, 1957.
	
	\textit{Выпуск 25}. \textls{Смогоржевский} А. С. Линейка в геометрических построениях, 1957.
	
	\textit{Выпуск 26}. \textls{Трахтенброт} Б. А. Алгоритмы и машинное решение задач, 1957.
	
	\textit{Выпуск 27}. \textls{Успенский} В. А. Некоторые приложения механики к математике, 1958.

\end{enumerate}

В частности, книга В. Г. Болтянского может служить первым введением в основные понятия дифференциального и интегрального исчисления. Более подробное изложение можно найти в книге:

\begin{enumerate}
	\setcounter{enumi}{8}
	
	\item \textls{Привалов} И. И. и \textls{Гальперин} С. А. Основы анализа бесконечно малых (пособие для самообразования), изд. 2. Гостехиздат, М.—Л., 1949. (Доступное, но вполне строгое изложение начал дифференциального и интегрального исчислений. Следует, впрочем, заметить, что познакомиться с началами высшей математики можно самостоятельно и по многим учебникам для вузов и техникумов). По стилю примыкают к «Популярным лекциям по математике», но несколько больше их по объему книжки, указанные под номерами 10—15.
	
	\item \textls{Александров} П. С. Введение в теорию групп, изд. 2. Учпедгиз, М., 1951.
	
	\item \textls{Люстерник} Л. А. Выпуклые тела, изд. 2. Гостехиздат, М.—Л., 1941.
	\item \textls{Шнирельман} Л. Г. Простые числа. Гостехиздат, М.—Л., 1940.
	
	\item \textls{Маркушевич} А И. Ряды, изд. 3, испр. и дополн. Гостехиздат, М., 1957.
	
	\item \textls{Хинчин} А. Я. Три жемчужины теории чисел, изд. 2. Гостехиздат, М.—Л., 1948.
	
	\item \textls{Вольберг} О. А. Основные идеи проективной геометрии, изд. 3. Учпедгиз. М.—Л., 1949.
	
	\item \textls{Радемахер} Г. и \textls{Теплиц} О. Числа и фигуры. Гл. ред. научно-попул. и юношеской литературы, изд. 2, М.—Л., 1938 (ОНТИ). (Очерки, показывающие на примерах из различных разделов математики пути, которыми были найдены некоторые замечательные математические предложения.) 
	
	\item \textls{Штейнгауз} Г. Математический калейдоскоп. Гостехиздат, М.—Л., 1949. (Ряд математических фактов сообщается в наглядной форме без всяких доказательств. Книжка может служить по преимуществу лишь для возбуждения интереса к более основательным занятиям математикой. Очень хорошо выполнены иллюстрации.)
	
	\item \textls{Адамар} Ж. Элементарная геометрия, ч. 1, изд. 4. Учпедгиз, М., 1957. (Систематическое изложение элементарной геометрии, выходящее за рамки школьных программ. Много интересных задач.)
	
	\item \textls{Александров} П С. Что такое неэвклидова геометрия. Изд-во Акад. пед. наук РСФСР, М., 1950.
	
	\item \textls{Делоне} Б. Н. Краткое изложение доказательства непротиворечивости планиметрии Лобачевского. Изд-во АН СССР, М., 1953.
	
	\item \textls{Гнеденко} Б. В. Очерки по истории математики в России. Гостехиздат, М.—Л., 1946. (Книга содержит много интересных исторических сведений. Изложение конкретных математических фактов. иногда слишком бегло, чтобы в нем можно было разобраться.)
	
	\item \textls{Китов} А. И. Электронные цифровые машины. Изд-во «Советское радио», М., 1956. (Книга по замыслу издательства рассчитана на инженеров и научных работников, но в большей своей части доступна школьникам старших классов.)
	
	\item \textls{Полетаев} И. А. Сигнал. Изд-во «Советское радио», М., 1958. (Увлекательно написанная книга о кибернетике. Содержит, кроме того, элементарные сведения по теории вероятностей, теории информации и теории игр.)
	
	\item Много интересного материала можно найти в журнале «Математика в школе», издание Учпедгиза, шесть номеров в год, и в журнале «Математические просвещения».

\end{enumerate}
	
Материалы математических школьных олимпиад печатаются также в журнале «Успехи математических наук» (Гостехиздат), остальное содержание которого, впрочем, мало доступно начинающим математикам.
	
В Большой Советской Энциклопедии (второе изд.) интересный и вполне доступный материал имеется во многих статьях (обычно в начале статей, конец которых предназначен для более подготовленного читателя). Таковы, например, статьи Алгебра, Алгорифм Евклида, Аналитическая геометрия, Арифметика, Бесконечно малые, Бесконечно удаленные элементы, Геометрия, Геометрия окружностей и сфер, Дифференциальное исчисление. Интегральное исчисление (первые параграфы двух последних статей рассчитаны именно на то, чтобы дать общее представление о предмете дифференциального и интегрального исчислений читателю, владеющему лишь курсом математики средней школы.)

	
	\newpage
	\tableofcontents
	
	\thispagestyle{empty} % 

	\newpage

\setcounter{secnumdepth}{0}  

\phantomsection	
	\begin{center}
				\ \\
	\ \\
	\ \\
	\ \\
	\textbf{Колмогоров Андрей Николаевич}
	\ \\
	О ПРОФЕССИИ МАТЕМАТИКА
		\ \\
			\ \\
	Редактор \textit{С. Ф. Кондрашкова}
	
Техн. редактор \textit{Г. И. Георгиева}
	\ \\
		\ \\
\parbox{7,2cm}{%
	\sloppy\setlength\parfillskip{0pt}
	Сдано в производство 15. VI 1959 г.
	
	Подписано к печати 14. III 60 г.
	
	Л-90183 \ \ Формат бум. 60$\times$92$\rfrac{1}{16}$
	
	
	Печ. л. 2,0 \ \ Бум. д. 1,0 \ \  Уч-изд. д. 1,88
}

Заказ 783

Тираж 40 000 (2 $\cdot$ 15 000) Цена 55 к.

Издательство Московского

универсистета

Москва, Ленинские горы

Административный корпус

	\ \\
		\ \\

1 типография Издательства МГУ

Москва, Моховая, 9





	\end{center}
			\thispagestyle{empty} % выключаем отображение номера для этой страницы


	\newpage

\setcounter{secnumdepth}{0}  

\phantomsection	

	\ \\
	\ \\
	\ \\
	\ \\
	
\hangindent=2cm \hangafter=0  	\textbf{ИНФОРМАЦИЯ ОБ ОЦИФРОВКЕ}

\hangindent=2cm \hangafter=0  	\textbf{Источник}:  \href{https://rutracker.org/forum/viewtopic.php?t=3908219}{https://rutracker.org/forum/viewtopic.php?t=3908219} 

\hangindent=2cm \hangafter=0  	\textbf{Версия}: 1.0 - от 16 ноября 2025
	
\hangindent=2cm \hangafter=0  	\textbf{Автор}: lord199 (\href{mailto:lord199@mail.ru}{lord199@mail.ru})

\hangindent=2cm \hangafter=0  Если вы нашли какие-то опечатки и другие неточности, просьба связаться по электронному адресу.

	
	

\thispagestyle{empty} % выключаем отображение номера для этой страницы
	
	
\end{document}

\printindex
